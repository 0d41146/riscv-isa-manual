\chapter{Introduction}

RISC-V (pronounced ``risk-five'') is a new instruction set
architecture (ISA) that was originally designed to support computer
architecture research and education, but which we now hope will also
become a standard free and open architecture for industry
implementations.  Our goals in defining RISC-V include:
\vspace{-0.1in}
\begin{itemize}
\parskip 0pt
\itemsep 1pt
\item A completely {\em open} ISA that is freely available to
  academia and industry.
\item A {\em real} ISA suitable for direct native hardware implementation,
  not just simulation or binary translation.
\item An ISA that avoids ``over-architecting'' for a particular
  microarchitecture style (e.g., microcoded, in-order, decoupled,
  out-of-order) or implementation technology (e.g., full-custom, ASIC,
  FPGA), but which allows efficient implementation in any of these.
\item An ISA separated into a {\em small} base integer ISA, usable by
  itself as a base for customized accelerators or for educational
  purposes, and optional standard extensions, to support
  general-purpose software development.
\item Support for the revised 2008 IEEE-754 floating-point standard~\cite{ieee754-2008}.
\item An ISA supporting extensive user-level ISA extensions and
  specialized variants.
\item Both 32-bit and 64-bit address space variants for
  applications, operating system kernels, and hardware implementations.
\item An ISA with support for highly-parallel multicore
  or manycore implementations, including heterogeneous multiprocessors.
\item Optional {\em variable-length instructions} to both expand available
  instruction encoding space and to support an optional {\em dense
  instruction encoding} for improved performance, static code size,
  and energy efficiency.
\item A fully virtualizable ISA to ease hypervisor development.
\item An ISA that simplifies experiments with new supervisor-level and
  hypervisor-level ISA designs.
\end{itemize}
\vspace{-0.1in}

\begin{commentary}
Commentary on our design decisions is formatted as in this paragraph,
and can be skipped if the reader is only interested in the
specification itself.
\end{commentary}
\begin{commentary}
The name RISC-V was chosen to represent the fifth major RISC ISA
design from UC Berkeley (RISC-I~\cite{riscI-isca1981},
RISC-II~\cite{Katevenis:1983}, SOAR~\cite{Ungar:1984}, and
SPUR~\cite{spur-jsscc1989} were the first four).  We also pun on the
use of the Roman numeral ``V'' to signify ``variations'' and
``vectors'', as support for a range of architecture research,
including various data-parallel accelerators, is an explicit goal of
the ISA design.
\end{commentary}

\begin{commentary}
We developed RISC-V to support our own needs in research and
education, where our group is particularly interested in actual
hardware implementations of research ideas (we have completed eleven
different silicon fabrications of RISC-V since the first edition of
this specification), and in providing real implementations for
students to explore in classes (RISC-V processor RTL designs have been
used in multiple undergraduate and graduate classes at Berkeley).  In
our current research, we are especially interested in the move towards
specialized and heterogeneous accelerators, driven by the power
constraints imposed by the end of conventional transistor scaling.  We
wanted a highly flexible and extensible base ISA around which to build
our research effort.

A question we have been repeatedly asked is ``Why develop a new ISA?''
The biggest obvious benefit of using an existing commercial ISA is the
large and widely supported software ecosystem, both development tools
and ported applications, which can be leveraged in research and
teaching.  Other benefits include the existence of large amounts of
documentation and tutorial examples.  However, our experience of using
commercial instruction sets for research and teaching is that these
benefits are smaller in practice, and do not outweigh the
disadvantages:

\begin{itemize}
\item {\bf Commercial ISAs are proprietary.}  Except for SPARC V8,
  which is an open IEEE standard~\cite{sparcieee1994}, most owners of
  commercial ISAs carefully guard their intellectual property and do
  not welcome freely available competitive implementations.  This is
  much less of an issue for academic research and teaching using only
  software simulators, but has been a major concern for groups wishing
  to share actual RTL implementations.  It is also a major concern for
  entities who do not want to trust the few sources of commercial ISA
  implementations, but who are prohibited from creating their own
  clean room implementations.  We cannot guarantee that all RISC-V
  implementations will be free of third-party patent infringements,
  but we can guarantee we will not attempt to sue a RISC-V
  implementor.

\item {\bf Commercial ISAs are only popular in certain market
  domains.}  The most obvious examples at time of writing are that
  the ARM architecture is not well supported in the server space, and
  the Intel x86 architecture (or for that matter, almost every other
  architecture) is not well supported in the mobile space, though both
  Intel and ARM are attempting to enter each other's market segments.
  Another example is ARC and Tensilica, which provide extensible cores
  but are focused on the embedded space.  This market segmentation
  dilutes the benefit of supporting a particular commercial ISA as in
  practice the software ecosystem only exists for certain domains, and
  has to be built for others.

\item {\bf Commercial ISAs come and go.}  Previous research
  infrastructures have been built around commercial ISAs that are no
  longer popular (SPARC, MIPS) or even no longer in production
  (Alpha).  These lose the benefit of an active software ecosystem,
  and the lingering intellectual property issues around the ISA and
  supporting tools interfere with the ability of interested third
  parties to continue supporting the ISA.  An open ISA might also lose
  popularity, but any interested party can continue using and
  developing the ecosystem.

\item {\bf Popular commercial ISAs are complex.}  The dominant
  commercial ISAs (x86 and ARM) are both very complex to implement in
  hardware to the level of supporting common software stacks and
  operating systems.  Worse, nearly all the complexity is due to bad,
  or at least outdated, ISA design decisions rather than features that
  truly improve efficiency.

\item {\bf Commercial ISAs alone are not enough to bring up
  applications.}  Even if we expend the effort to implement a
  commercial ISA, this is not enough to run existing applications for
  that ISA.  Most applications need a complete ABI (application binary
  interface) to run, not just the user-level ISA.  Most ABIs rely on
  libraries, which in turn rely on operating system support.  To run an
  existing operating system requires implementing the supervisor-level
  ISA and device interfaces expected by the OS.  These are usually
  much less well-specified and considerably more complex to
  implement than the user-level ISA.

\item {\bf Popular commercial ISAs were not designed for extensibility.}  The
  dominant commercial ISAs were not particularly designed for
  extensibility, and as a consequence have added considerable
  instruction encoding complexity as their instruction sets have
  grown.  Companies such as Tensilica (acquired by Cadence) and ARC
  (acquired by Synopsys) have built ISAs and toolchains around
  extensibility, but have focused on embedded applications rather than
  general-purpose computing systems.

\item {\bf A modified commercial ISA is a new ISA.} One of our main
  goals is to support architecture research, including major ISA
  extensions.  Even small extensions diminish the benefit of using a
  standard ISA, as compilers have to be modified and applications
  rebuilt from source code to use the extension.  Larger extensions
  that introduce new architectural state also require modifications to
  the operating system.  Ultimately, the modified commercial ISA
  becomes a new ISA, but carries along all the legacy baggage of the
  base ISA.
\end{itemize}

Our position is that the ISA is perhaps the most important interface
in a computing system, and there is no reason that such an important
interface should be proprietary.  The dominant commercial ISAs are
based on instruction set concepts that were already well known over 30
years ago.  Software developers should be able to target an open
standard hardware target, and commercial processor designers should
compete on implementation quality.

We are far from the first to contemplate an open ISA design suitable
for hardware implementation.  We also considered other existing open
ISA designs, of which the closest to our goals was the OpenRISC
architecture~\cite{openriscarch}.  We decided against adopting the
OpenRISC ISA for several technical reasons:

\begin{itemize}
\item OpenRISC has condition codes and branch delay slots, which
  complicate higher performance implementations.
\item OpenRISC uses a fixed 32-bit encoding and 16-bit immediates,
  which precludes a denser instruction encoding and limits space for
  later expansion of the ISA.
\item OpenRISC does not support the 2008 revision to the IEEE 754
  floating-point standard.
\item The OpenRISC 64-bit design had not been completed when we began.
\end{itemize}

By starting from a clean slate, we could design an ISA that met all of
our goals, though of course, this took far more effort than we had
planned at the outset.  We have now invested considerable effort in
building up the RISC-V ISA infrastructure, including documentation,
compiler tool chains, operating system ports, reference ISA
simulators, FPGA implementations, efficient ASIC implementations,
architecture test suites, and teaching materials. Since the last
edition of this manual, there has been considerable uptake of the
RISC-V ISA in both academia and industry, and we have created the
non-profit RISC-V Foundation to protect and promote the standard.  The
RISC-V Foundation website at \url{http://riscv.org} contains the latest
information on the Foundation membership and various open-source
projects using RISC-V.
\end{commentary}

The RISC-V manual is structured in two volumes.  This volume covers
the user-level ISA design, including optional ISA extensions.  The
second volume provides the privileged architecture.

\begin{commentary}
In this user-level manual, we aim to remove any dependence on
particular microarchitectural features or on privileged architecture
details.  This is both for clarity and to allow maximum flexibility
for alternative implementations.
\end{commentary}

\section{RISC-V ISA Overview}

The RISC-V ISA is defined as a base integer ISA, which must be present
in any implementation, plus optional extensions to the base ISA.  The
base integer ISA is very similar to that of the early RISC processors
except with no branch delay slots and with support for optional
variable-length instruction encodings.  The base is carefully
restricted to a minimal set of instructions sufficient to provide a
reasonable target for compilers, assemblers, linkers, and operating
systems (with additional supervisor-level operations), and so provides
a convenient ISA and software toolchain ``skeleton'' around which more
customized processor ISAs can be built.

Each base integer instruction set is characterized by the width of the
integer registers and the corresponding size of the user address
space.  There are two primary base integer variants, RV32I and RV64I,
described in Chapters~\ref{rv32} and \ref{rv64}, which provide 32-bit
or 64-bit user-level address spaces respectively.  Hardware
implementations and operating systems might provide only one or both
of RV32I and RV64I for user programs.  Chapter~\ref{rv32e} describes
the RV32E subset variant of the RV32I base instruction set, which has
been added to support small microcontrollers.  Chapter~\ref{rv128}
describes a future RV128I variant of the base integer instruction set
supporting a flat 128-bit user address space.

\begin{commentary}
Although 64-bit address spaces are a requirement for larger systems,
we believe 32-bit address spaces will remain adequate for many
embedded and client devices for decades to come and will be desirable
to lower memory traffic and energy consumption.  In addition, 32-bit
address spaces are sufficient for educational purposes.  A larger flat
128-bit address space might eventually be required, so we ensured this
could be accommodated within the RISC-V ISA framework.
\end{commentary}

The base integer ISA may be subset by a hardware implementation, but
opcode traps and software emulation by a more privileged layer must
then be used to implement functionality not provided by hardware.

\begin{commentary}
Subsets of the base integer ISA might be useful for pedagogical
purposes, but the base has been defined such that there should be
little incentive to subset a real hardware implementation beyond
omitting support for misaligned memory accesses and treating all SYSTEM
instructions as a single trap.
\end{commentary}

RISC-V has been designed to support extensive customization and
specialization.  The base integer ISA can be extended with one or more
optional instruction-set extensions, but the base integer instructions
cannot be redefined.  We divide RISC-V instruction-set extensions
into {\em standard} and {\em non-standard} extensions.  Standard
extensions should be generally useful and should not conflict with
other standard extensions.  Non-standard extensions may be highly
specialized, or may conflict with other standard or non-standard
extensions.  Instruction-set extensions may provide slightly different
functionality depending on the width of the base integer instruction
set.  Chapter~\ref{extensions} describes various ways of extending the
RISC-V ISA.  We have also developed a naming convention for RISC-V
base instructions and instruction-set extensions, described in detail
in Chapter~\ref{naming}.

To support more general software development, a set of standard
extensions are defined to provide integer multiply/divide, atomic
operations, and single and double-precision floating-point arithmetic.
The base integer ISA is named ``I'' (prefixed by RV32 or RV64
depending on integer register width), and contains integer
computational instructions, integer loads, integer stores, and
control-flow instructions, and is mandatory for all RISC-V
implementations.  The standard integer multiplication and division
extension is named ``M'', and adds instructions to multiply and divide
values held in the integer registers.  The standard atomic instruction
extension, denoted by ``A'', adds instructions that atomically read,
modify, and write memory for inter-processor synchronization.  The
standard single-precision floating-point extension, denoted by ``F'',
adds floating-point registers, single-precision computational
instructions, and single-precision loads and stores.  The standard
double-precision floating-point extension, denoted by ``D'', expands
the floating-point registers, and adds double-precision computational
instructions, loads, and stores.  An integer base plus these four
standard extensions (``IMAFD'') is given the abbreviation ``G'' and
provides a general-purpose scalar instruction set.  RV32G and RV64G
are currently the default target of our compiler toolchains.  Later
chapters describe these and other planned standard RISC-V extensions.

Beyond the base integer ISA and the standard extensions, it is rare
that a new instruction will provide a significant benefit for all
applications, although it may be very beneficial for a certain domain.
As energy efficiency concerns are forcing greater specialization, we
believe it is important to simplify the required portion of an ISA
specification.  Whereas other architectures usually treat their ISA as
a single entity, which changes to a new version as instructions are
added over time, RISC-V will endeavor to keep the base and each
standard extension constant over time, and instead layer new
instructions as further optional extensions.  For example, the base
integer ISAs will continue as fully supported standalone ISAs,
regardless of any subsequent extensions.
\begin{commentary}
With the 2.0 release of the user ISA specification, we intend the
``RV32IMAFD'' and ``RV64IMAFD''base and standard extensions
(aka. ``RV32G'' and ``RV64G'') to remain constant for future
development.
\end{commentary}

\section{Instruction Length Encoding}

The base RISC-V ISA has fixed-length 32-bit instructions that must be
naturally aligned on 32-bit boundaries.  However, the standard RISC-V
encoding scheme is designed to support ISA extensions with
variable-length instructions, where each instruction can be any number
of 16-bit instruction {\em parcels} in length and parcels are
naturally aligned on 16-bit boundaries.  The standard compressed ISA
extension described in Chapter~\ref{compressed} reduces code size by
providing compressed 16-bit instructions and relaxes the alignment
constraints to allow all instructions (16 bit and 32 bit) to be
aligned on any 16-bit boundary to improve code density.

Figure~\ref{instlengthcode} illustrates the standard RISC-V
instruction-length encoding convention.  All the 32-bit instructions
in the base ISA have their lowest two bits set to {\tt 11}.  The
optional compressed 16-bit instruction-set extensions have their
lowest two bits equal to {\tt 00}, {\tt 01}, or {\tt 10}.  Standard
instruction-set extensions encoded with more than 32 bits have
additional low-order bits set to {\tt 1}, with the conventions for
48-bit and 64-bit lengths shown in Figure~\ref{instlengthcode}.
Instruction lengths between 80 bits and 176 bits are encoded using a
3-bit field in bits [14:12] giving the number of 16-bit words in
addition to the first 5$\times$16-bit words.  The encoding with bits
[14:12] set to {\tt 111} is reserved for future longer instruction
encodings.

\begin{figure}[hb]
{
\begin{center}
\begin{tabular}{ccccl}
\cline{4-4}
& & & \multicolumn{1}{|c|}{\tt xxxxxxxxxxxxxxaa} & 16-bit ({\tt aa}
$\neq$ {\tt 11})\\
\cline{4-4}
\\
\cline{3-4}
& & \multicolumn{1}{|c|}{\tt xxxxxxxxxxxxxxxx}
& \multicolumn{1}{c|}{\tt xxxxxxxxxxxbbb11} & 32-bit ({\tt bbb}
$\neq$ {\tt 111}) \\
\cline{3-4}
\\
\cline{2-4}
\hspace{0.1in} 
& \multicolumn{1}{c|}{$\cdot\cdot\cdot${\tt xxxx} }
& \multicolumn{1}{c|}{\tt xxxxxxxxxxxxxxxx}
& \multicolumn{1}{c|}{\tt xxxxxxxxxx011111} & 48-bit \\
\cline{2-4}
\\
\cline{2-4}
\hspace{0.1in} 
& \multicolumn{1}{c|}{$\cdot\cdot\cdot${\tt xxxx} }
& \multicolumn{1}{c|}{\tt xxxxxxxxxxxxxxxx}
& \multicolumn{1}{c|}{\tt xxxxxxxxx0111111} & 64-bit \\
\cline{2-4}
\\
\cline{2-4}
\hspace{0.1in} 
& \multicolumn{1}{c|}{$\cdot\cdot\cdot${\tt xxxx} }
& \multicolumn{1}{c|}{\tt xxxxxxxxxxxxxxxx}
& \multicolumn{1}{c|}{\tt xnnnxxxxx1111111} & (80+16*{\tt nnn})-bit,
       {\tt nnn}$\neq${\tt 111} \\
\cline{2-4}
\\
\cline{2-4}
\hspace{0.1in} 
& \multicolumn{1}{c|}{$\cdot\cdot\cdot${\tt xxxx} }
& \multicolumn{1}{c|}{\tt xxxxxxxxxxxxxxxx}
& \multicolumn{1}{c|}{\tt x111xxxxx1111111} & Reserved for $\geq$192-bits \\
\cline{2-4}
\\
Byte Address: & \multicolumn{1}{r}{base+4} & \multicolumn{1}{r}{base+2} & \multicolumn{1}{r}{base} & \\
 \end{tabular}
\end{center}
}
\caption{RISC-V instruction length encoding.}
\label{instlengthcode}
\end{figure}

\begin{commentary}
Given the code size and energy savings of a compressed format, we
wanted to build in support for a compressed format to the ISA encoding
scheme rather than adding this as an afterthought, but to allow
simpler implementations we didn't want to make the compressed format
mandatory. We also wanted to optionally allow longer instructions to
support experimentation and larger instruction-set extensions.
Although our encoding convention required a tighter encoding of the
core RISC-V ISA, this has several beneficial effects.

An implementation of the standard G ISA need only hold the
most-significant 30 bits in instruction caches (a 6.25\% saving).  On
instruction cache refills, any instructions encountered with either
low bit clear should be recoded into illegal 30-bit instructions
before storing in the cache to preserve illegal instruction exception
behavior.

Perhaps more importantly, by condensing our base ISA into a subset of
the 32-bit instruction word, we leave more space available for custom
extensions.  In particular, the base RV32I ISA uses less than 1/8 of
the encoding space in the 32-bit instruction word.  As described in
Chapter~\ref{extensions}, an implementation that does not require
support for the standard compressed instruction extension can map 3
additional 30-bit instruction spaces into the 32-bit fixed-width
format, while preserving support for standard $>=$32-bit
instruction-set extensions.  Further, if the implementation also does
not need instructions $>$32-bits in length, it can recover a further
four major opcodes.
\end{commentary}
\begin{commentary}
We consider it a feature that any length of instruction containing all
zero bits is not legal, as this quickly traps erroneous jumps into
zeroed memory regions. Similarly, we also reserve the instruction
encoding containing all ones to be an illegal instruction, to catch
the other common pattern observed with unprogrammed non-volatile
memory devices, disconnected memory buses, or broken memory devices.
\end{commentary}

The base RISC-V ISA has a little-endian memory system, but
non-standard variants can provide a big-endian or bi-endian memory
system.  Instructions are stored in memory with each 16-bit parcel
stored in a memory halfword according to the implementation's natural
endianness. Parcels comprising one instruction are stored at
increasing halfword addresses, with the lowest addressed parcel
holding the lowest numbered bits in the instruction specification,
i.e., instructions are always stored in a little-endian sequence of
parcels regardless of the memory system endianness.  The code sequence
in Figure~\ref{fig:storeinstruction} will store a 32-bit instruction
to memory correctly regardless of memory system endianness.

\begin{figure}[ht]
\begin{verbatim}
    // Store 32-bit instruction in x2 register to location pointed to by x3.
    sh   x2, 0(x3)    // Store low bits of instruction in first parcel.
    srli x2, x2, 16   // Move high bits down to low bits, overwriting x2.
    sh   x2, 2(x3)    // Store high bits in second parcel.
\end{verbatim}
\caption{Recommended code sequence to store 32-bit instruction from register to
  memory.  Operates correctly on both big- and little-endian
  memory systems and avoids misaligned accesses when used with variable-length
  instruction-set extensions.}
\label{fig:storeinstruction}
\end{figure}

\begin{commentary}
We chose little-endian byte ordering for the RISC-V memory system
because little-endian systems are currently dominant commercially (all
x86 systems; iOS, Android, and Windows for ARM).  A minor point is
that we have also found little-endian memory systems to be more
natural for hardware designers.  However, certain application areas,
such as IP networking, operate on big-endian data structures, and so
we leave open the possibility of non-standard big-endian or bi-endian
systems.

We have to fix the order in which instruction parcels are stored in
memory, independent of memory system endianness, to ensure that the
length-encoding bits always appear first in halfword address
order. This allows the length of a variable-length instruction to be
quickly determined by an instruction fetch unit by examining only the
first few bits of the first 16-bit instruction parcel.  Once we had
decided to fix on a little-endian memory system and instruction parcel
ordering, this naturally led to placing the length-encoding bits in
the LSB positions of the instruction format to avoid breaking up
opcode fields.
\end{commentary}

\section{Exceptions, Traps, and Interrupts}

We use the term {\em exception} to refer to an unusual condition
occurring at run time associated with an instruction in the current
RISC-V thread.  We use the term {\em trap} to refer to the synchronous
transfer of control to a trap handler caused by an exceptional
condition occurring within a RISC-V thread.  Trap handlers usually
execute in a more privileged environment.

We use the term {\em interrupt} to refer to an external event that
occurs asynchronously to the current RISC-V thread. When an interrupt
that must be serviced occurs, some instruction is selected to receive
an interrupt exception and subsequently experiences a trap.

The instruction descriptions in following chapters describe conditions
that raise an exception during execution.  Whether and how these are
converted into traps is dependent on the execution environment, though
the expectation is that most environments will take a {\em precise}
trap when an exception is signaled (except for floating-point
exceptions, which, in the standard floating-point extensions, do not
cause traps).

\begin{commentary}
Our use of ``exception'' and ``trap'' matches that in the IEEE-754
floating-point standard.
\end{commentary}

