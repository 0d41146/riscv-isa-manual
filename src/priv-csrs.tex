\chapter{Control and Status Registers (CSRs)}
\label{chap:priv-csrs}

The SYSTEM major opcode is used to encode all privileged instructions
in the RISC-V ISA.
These can be divided into two main classes: those that atomically
read-modify-write control and status registers (CSRs), which are defined in
the Zicsr extension, and all other privileged instructions.
The privileged architecture requires the Zicsr extension; which other
privileged instructions are required depends on the privileged-architecture
feature set.

In addition to the unprivileged
state described in Volume I of this manual, an implementation may
contain additional CSRs, accessible by some subset of the privilege
levels using the CSR instructions described in Volume~I.
In this chapter, we map out the CSR address space.  The following
chapters describe the function of each of the CSRs according to
privilege level, as well as the other privileged instructions which
are generally closely associated with a particular privilege level.
Note that although CSRs and instructions are associated with one
privilege level, they are also accessible at all higher privilege
levels.

Standard CSRs do not have side effects on reads but may have side effects
on writes.

\section{CSR Address Mapping Conventions}

The standard RISC-V ISA sets aside a 12-bit encoding space (csr[11:0])
for up to 4,096 CSRs.  By convention, the upper 4 bits of the CSR
address (csr[11:8]) are used to encode the read and write
accessibility of the CSRs according to privilege level as shown in
Table~\ref{csrrwpriv}.  The top two bits (csr[11:10]) indicate whether
the register is read/write ({\tt 00}, {\tt 01}, or {\tt 10}) or
read-only ({\tt 11}).  The next two bits (csr[9:8]) encode the lowest
privilege level that can access the CSR.

\begin{commentary}
The CSR address convention uses the upper bits of the CSR address to
encode default access privileges.  This simplifies error checking in
the hardware and provides a larger CSR space, but does constrain the
mapping of CSRs into the address space.

Implementations might allow a more-privileged level to trap otherwise
permitted CSR accesses by a less-privileged level to allow these
accesses to be intercepted.  This change should be transparent to the
less-privileged software.
\end{commentary}

\begin{table*}[h!]
\begin{center}
\begin{tabular}{|c|c|c|c|l|}
\hline
\multicolumn{3}{|c|}{CSR Address} & Hex & \multicolumn{1}{c|}{Use and Accessibility}\\ \cline{1-3}
[11:10] & [9:8] & [7:4]                  &  & \\
\hline
\multicolumn{5}{|c|}{Unprivileged and User-Level CSRs}  \\
\hline
\tt   00   &\tt   00  &\tt   XXXX   & \tt 0x000-0x0FF & Standard read/write \\
\tt   01   &\tt   00  &\tt   XXXX   & \tt 0x400-0x4FF & Standard read/write \\
\tt   10   &\tt   00  &\tt   XXXX   & \tt 0x800-0x8FF & Custom read/write \\
\tt   11   &\tt   00  &\tt   0XXX   & \tt 0xC00-0xC7F & Standard read-only \\
\tt   11   &\tt   00  &\tt   10XX   & \tt 0xC80-0xCBF & Standard read-only \\
\tt   11   &\tt   00  &\tt   11XX   & \tt 0xCC0-0xCFF & Custom read-only \\
\hline
\multicolumn{5}{|c|}{Supervisor-Level CSRs}  \\
\hline
\tt   00   &\tt   01  &\tt   XXXX   & \tt 0x100-0x1FF & Standard read/write \\
\tt   01   &\tt   01  &\tt   0XXX   & \tt 0x500-0x57F & Standard read/write \\
\tt   01   &\tt   01  &\tt   10XX   & \tt 0x580-0x5BF & Standard read/write \\
\tt   01   &\tt   01  &\tt   11XX   & \tt 0x5C0-0x5FF & Custom read/write \\
\tt   10   &\tt   01  &\tt   0XXX   & \tt 0x900-0x97F & Standard read/write \\
\tt   10   &\tt   01  &\tt   10XX   & \tt 0x980-0x9BF & Standard read/write \\
\tt   10   &\tt   01  &\tt   11XX   & \tt 0x9C0-0x9FF & Custom read/write \\
\tt   11   &\tt   01  &\tt   0XXX   & \tt 0xD00-0xD7F & Standard read-only \\
\tt   11   &\tt   01  &\tt   10XX   & \tt 0xD80-0xDBF & Standard read-only \\
\tt   11   &\tt   01  &\tt   11XX   & \tt 0xDC0-0xDFF & Custom read-only \\
\hline
\multicolumn{5}{|c|}{Hypervisor and VS CSRs} \\
\hline
\tt   00   &\tt   10  &\tt   XXXX   & \tt 0x200-0x2FF & Standard read/write \\
\tt   01   &\tt   10  &\tt   0XXX   & \tt 0x600-0x67F & Standard read/write \\
\tt   01   &\tt   10  &\tt   10XX   & \tt 0x680-0x6BF & Standard read/write \\
\tt   01   &\tt   10  &\tt   11XX   & \tt 0x6C0-0x6FF & Custom read/write \\
\tt   10   &\tt   10  &\tt   0XXX   & \tt 0xA00-0xA7F & Standard read/write \\
\tt   10   &\tt   10  &\tt   10XX   & \tt 0xA80-0xABF & Standard read/write \\
\tt   10   &\tt   10  &\tt   11XX   & \tt 0xAC0-0xAFF & Custom read/write \\
\tt   11   &\tt   10  &\tt   0XXX   & \tt 0xE00-0xE7F & Standard read-only \\
\tt   11   &\tt   10  &\tt   10XX   & \tt 0xE80-0xEBF & Standard read-only \\
\tt   11   &\tt   10  &\tt   11XX   & \tt 0xEC0-0xEFF & Custom read-only \\
\hline
\multicolumn{5}{|c|}{Machine-Level CSRs}  \\
\hline
\tt   00   &\tt   11  &\tt   XXXX   & \tt 0x300-0x3FF & Standard read/write \\
\tt   01   &\tt   11  &\tt   0XXX   & \tt 0x700-0x77F & Standard read/write \\
\tt   01   &\tt   11  &\tt   100X   & \tt 0x780-0x79F & Standard read/write \\
\tt   01   &\tt   11  &\tt   1010   & \tt 0x7A0-0x7AF & Standard read/write debug CSRs  \\
\tt   01   &\tt   11  &\tt   1011   & \tt 0x7B0-0x7BF & Debug-mode-only CSRs \\
\tt   01   &\tt   11  &\tt   11XX   & \tt 0x7C0-0x7FF & Custom read/write \\
\tt   10   &\tt   11  &\tt   0XXX   & \tt 0xB00-0xB7F & Standard read/write \\
\tt   10   &\tt   11  &\tt   10XX   & \tt 0xB80-0xBBF & Standard read/write \\
\tt   10   &\tt   11  &\tt   11XX   & \tt 0xBC0-0xBFF & Custom read/write \\
\tt   11   &\tt   11  &\tt   0XXX   & \tt 0xF00-0xF7F & Standard read-only \\
\tt   11   &\tt   11  &\tt   10XX   & \tt 0xF80-0xFBF & Standard read-only \\
\tt   11   &\tt   11  &\tt   11XX   & \tt 0xFC0-0xFFF & Custom read-only \\
\hline
\end{tabular}
\end{center}
\caption{Allocation of RISC-V CSR address ranges.}
\label{csrrwpriv}
\end{table*}

Attempts to access a non-existent CSR raise an illegal instruction
exception.  Attempts to access a CSR without appropriate privilege
level or to write a read-only register also raise illegal instruction
exceptions.  A read/write register might also contain some bits that
are read-only, in which case writes to the read-only bits are ignored.

Table~\ref{csrrwpriv} also indicates the convention to allocate CSR
addresses between standard and custom uses.  The CSR addresses
designated for custom uses will not be redefined by future
standard extensions.

Machine-mode standard read-write CSRs {\tt 0x7A0}--{\tt 0x7BF} are reserved
for use by the debug system.  Of these CSRs, {\tt 0x7A0}--{\tt 0x7AF} are
accessible to machine mode, whereas {\tt 0x7B0}--{\tt 0x7BF} are only visible
to debug mode.  Implementations should raise illegal instruction exceptions on
machine-mode access to the latter set of registers.

\begin{commentary}
Effective virtualization requires that as many instructions run natively as
possible inside a virtualized environment, while any privileged accesses trap
to the virtual machine monitor~\cite{goldbergvm}.  CSRs that are read-only at
some lower privilege level are shadowed into separate CSR addresses if they
are made read-write at a higher privilege level.  This avoids trapping
permitted lower-privilege accesses while still causing traps on illegal
accesses.  Currently, the counters are the only shadowed CSRs.
\end{commentary}

\section{CSR Listing}

Tables~\ref{ucsrnames}--\ref{mcsrnames1} list the CSRs that have
currently been allocated CSR addresses.  The timers, counters, and
floating-point CSRs are standard unprivileged CSRs.
The other
registers are used by privileged code, as described in the following
chapters.  Note that not all registers are required on all
implementations.

\begin{table}[htb!]
\begin{center}
\begin{tabular}{|l|l|l|l|}
\hline
Number    & Privilege & Name & Description \\
\hline
\multicolumn{4}{|c|}{Unprivileged Floating-Point CSRs} \\
\hline
\tt 0x001 & URW  &\tt fflags     & Floating-Point Accrued Exceptions. \\
\tt 0x002 & URW  &\tt frm        & Floating-Point Dynamic Rounding Mode. \\
\tt 0x003 & URW  &\tt fcsr       & Floating-Point Control and Status
Register ({\tt frm} + {\tt fflags}). \\
\hline
\multicolumn{4}{|c|}{Unprivileged Counter/Timers} \\
\hline
\tt 0xC00 & URO  &\tt cycle         & Cycle counter for RDCYCLE instruction. \\
\tt 0xC01 & URO  &\tt time          & Timer for RDTIME instruction. \\
\tt 0xC02 & URO  &\tt instret       & Instructions-retired counter for RDINSTRET instruction. \\
\tt 0xC03 & URO  &\tt hpmcounter3   & Performance-monitoring counter. \\
\tt 0xC04 & URO  &\tt hpmcounter4   & Performance-monitoring counter. \\
& & \multicolumn{1}{c|}{\vdots} & \ \\
\tt 0xC1F & URO  &\tt hpmcounter31  & Performance-monitoring counter. \\
\tt 0xC80 & URO  &\tt cycleh        & Upper 32 bits of {\tt cycle}, RV32 only. \\
\tt 0xC81 & URO  &\tt timeh         & Upper 32 bits of {\tt time}, RV32 only. \\
\tt 0xC82 & URO  &\tt instreth      & Upper 32 bits of {\tt instret}, RV32 only. \\
\tt 0xC83 & URO  &\tt hpmcounter3h  & Upper 32 bits of {\tt hpmcounter3}, RV32 only. \\
\tt 0xC84 & URO  &\tt hpmcounter4h  & Upper 32 bits of {\tt hpmcounter4}, RV32 only. \\
& & \multicolumn{1}{c|}{\vdots} & \ \\
\tt 0xC9F & URO  &\tt hpmcounter31h & Upper 32 bits of {\tt hpmcounter31}, RV32 only. \\
\hline
\end{tabular}
\end{center}
\caption{Currently allocated RISC-V unprivileged CSR addresses.}
\label{ucsrnames}
\end{table}

\begin{table}[htb!]
\begin{center}
\begin{tabular}{|l|l|l|l|}
\hline
Number    & Privilege & Name & Description \\
\hline
\multicolumn{4}{|c|}{Supervisor Trap Setup} \\
\hline
\tt 0x100 & SRW  &\tt sstatus    & Supervisor status register. \\
\tt 0x104 & SRW  &\tt sie        & Supervisor interrupt-enable register. \\
\tt 0x105 & SRW  &\tt stvec      & Supervisor trap handler base address. \\
\tt 0x106 & SRW  &\tt scounteren & Supervisor counter enable. \\
\hline
\multicolumn{4}{|c|}{Supervisor Configuration} \\
\hline
\tt 0x10A & SRW  &\tt senvcfg    & Supervisor environment configuration register. \\
\hline
\multicolumn{4}{|c|}{Supervisor Trap Handling} \\
\hline
\tt 0x140 & SRW  &\tt sscratch   & Scratch register for supervisor trap handlers. \\
\tt 0x141 & SRW  &\tt sepc       & Supervisor exception program counter. \\
\tt 0x142 & SRW  &\tt scause     & Supervisor trap cause. \\
\tt 0x143 & SRW  &\tt stval      & Supervisor bad address or instruction. \\
\tt 0x144 & SRW  &\tt sip        & Supervisor interrupt pending. \\
\hline
\multicolumn{4}{|c|}{Supervisor Protection and Translation} \\
\hline
\tt 0x180 & SRW  &\tt satp       & Supervisor address translation and protection. \\
\hline
\multicolumn{4}{|c|}{Debug/Trace Registers} \\
\hline
\tt 0x5A8 & SRW &\tt scontext & Supervisor-mode context register. \\
\hline
\end{tabular}
\end{center}
\caption{Currently allocated RISC-V supervisor-level CSR addresses.}
\label{scsrnames}
\end{table}

\begin{table}[htb!]
\begin{center}
\begin{tabular}{|l|l|l|l|}
\hline
Number    & Privilege & Name & Description \\
\hline
\multicolumn{4}{|c|}{Hypervisor Trap Setup} \\
\hline
\hline
\tt 0x600 & HRW  &\tt hstatus    & Hypervisor status register. \\
\tt 0x602 & HRW  &\tt hedeleg    & Hypervisor exception delegation register. \\
\tt 0x603 & HRW  &\tt hideleg    & Hypervisor interrupt delegation register. \\
\tt 0x604 & HRW  &\tt hie        & Hypervisor interrupt-enable register. \\
\tt 0x606 & HRW  &\tt hcounteren & Hypervisor counter enable. \\
\tt 0x607 & HRW  &\tt hgeie      & Hypervisor guest external interrupt-enable register. \\
\hline
\multicolumn{4}{|c|}{Hypervisor Trap Handling} \\
\hline
\tt 0x643 & HRW  &\tt htval      & Hypervisor bad guest physical address. \\
\tt 0x644 & HRW  &\tt hip        & Hypervisor interrupt pending. \\
\tt 0x645 & HRW  &\tt hvip       & Hypervisor virtual interrupt pending. \\
\tt 0x64A & HRW  &\tt htinst     & Hypervisor trap instruction (transformed). \\
\tt 0xE12 & HRO  &\tt hgeip      & Hypervisor guest external interrupt pending. \\
\hline
\multicolumn{4}{|c|}{Hypervisor Configuration} \\
\hline
\tt 0x60A & HRW  &\tt henvcfg    & Hypervisor environment configuration register. \\
\tt 0x61A & HRW  &\tt henvcfgh   & Additional hypervisor env. conf. register, RV32 only. \\
\hline
\multicolumn{4}{|c|}{Hypervisor Protection and Translation} \\
\hline
\tt 0x680 & HRW  &\tt hgatp      & Hypervisor guest address translation and protection. \\
\hline
\multicolumn{4}{|c|}{Debug/Trace Registers} \\
\hline
\tt 0x6A8 & HRW &\tt hcontext & Hypervisor-mode context register. \\
\hline
\multicolumn{4}{|c|}{Hypervisor Counter/Timer Virtualization Registers} \\
\hline
\tt 0x605 & HRW  &\tt htimedelta   & Delta for VS/VU-mode timer. \\
\tt 0x615 & HRW  &\tt htimedeltah  & Upper 32 bits of {\tt htimedelta}, HSXLEN=32 only. \\
\hline
\multicolumn{4}{|c|}{Virtual Supervisor Registers} \\
\hline
\tt 0x200 & HRW  &\tt vsstatus   & Virtual supervisor status register. \\
\tt 0x204 & HRW  &\tt vsie       & Virtual supervisor interrupt-enable register. \\
\tt 0x205 & HRW  &\tt vstvec     & Virtual supervisor trap handler base address. \\
\tt 0x240 & HRW  &\tt vsscratch  & Virtual supervisor scratch register. \\
\tt 0x241 & HRW  &\tt vsepc      & Virtual supervisor exception program counter. \\
\tt 0x242 & HRW  &\tt vscause    & Virtual supervisor trap cause. \\
\tt 0x243 & HRW  &\tt vstval     & Virtual supervisor bad address or instruction. \\
\tt 0x244 & HRW  &\tt vsip       & Virtual supervisor interrupt pending. \\
\tt 0x280 & HRW  &\tt vsatp      & Virtual supervisor address translation and protection. \\
\hline
\end{tabular}
\end{center}
\caption{Currently allocated RISC-V hypervisor and VS CSR addresses.}
\label{hcsrnames}
\end{table}


\begin{table}[htb!]
\begin{center}
\begin{tabular}{|l|l|l|l|}
\hline
Number    & Privilege & Name & Description \\
\hline
\multicolumn{4}{|c|}{Machine Information Registers} \\
\hline
\tt 0xF11 & MRO &\tt mvendorid   & Vendor ID. \\
\tt 0xF12 & MRO &\tt marchid     & Architecture ID. \\
\tt 0xF13 & MRO &\tt mimpid      & Implementation ID. \\
\tt 0xF14 & MRO &\tt mhartid     & Hardware thread ID. \\
\tt 0xF15 & MRO &\tt mconfigptr  & Pointer to configuration data structure. \\
\hline
\multicolumn{4}{|c|}{Machine Trap Setup} \\
\hline
\tt 0x300 & MRW  &\tt mstatus    & Machine status register. \\
\tt 0x301 & MRW  &\tt misa       & ISA and extensions \\
\tt 0x302 & MRW  &\tt medeleg    & Machine exception delegation register. \\
\tt 0x303 & MRW  &\tt mideleg    & Machine interrupt delegation register. \\
\tt 0x304 & MRW  &\tt mie        & Machine interrupt-enable register. \\
\tt 0x305 & MRW  &\tt mtvec      & Machine trap-handler base address. \\
\tt 0x306 & MRW  &\tt mcounteren & Machine counter enable. \\
\tt 0x310 & MRW  &\tt mstatush   & Additional machine status register, RV32 only. \\
\hline
\multicolumn{4}{|c|}{Machine Trap Handling} \\
\hline
\tt 0x340 & MRW  &\tt mscratch   & Scratch register for machine trap handlers. \\
\tt 0x341 & MRW  &\tt mepc       & Machine exception program counter. \\
\tt 0x342 & MRW  &\tt mcause     & Machine trap cause. \\
\tt 0x343 & MRW  &\tt mtval      & Machine bad address or instruction. \\
\tt 0x344 & MRW  &\tt mip        & Machine interrupt pending. \\
\tt 0x34A & MRW  &\tt mtinst     & Machine trap instruction (transformed). \\
\tt 0x34B & MRW  &\tt mtval2     & Machine bad guest physical address. \\
\hline
\multicolumn{4}{|c|}{Machine Configuration} \\
\hline
\tt 0x30A & MRW  &\tt menvcfg    & Machine environment configuration register. \\
\tt 0x31A & MRW  &\tt menvcfgh   & Additional machine env. conf. register, RV32 only. \\
\tt 0x747 & MRW  &\tt mseccfg    & Machine security configuration register. \\
\tt 0x757 & MRW  &\tt mseccfgh   & Additional machine security conf. register, RV32 only. \\
\hline
\multicolumn{4}{|c|}{Machine Memory Protection} \\
\hline
%\tt 0x380 & MRW  &\tt mbase      & Base register. \\
%\tt 0x381 & MRW  &\tt mbound     & Bound register. \\
%\tt 0x382 & MRW  &\tt mibase     & Instruction base register. \\
%\tt 0x383 & MRW  &\tt mibound    & Instruction bound register. \\
%\tt 0x384 & MRW  &\tt mdbase     & Data base register. \\
%\tt 0x385 & MRW  &\tt mdbound    & Data bound register. \\
\tt 0x3A0 & MRW  &\tt pmpcfg0    & Physical memory protection configuration. \\
\tt 0x3A1 & MRW  &\tt pmpcfg1    & Physical memory protection configuration, RV32 only. \\
\tt 0x3A2 & MRW  &\tt pmpcfg2    & Physical memory protection configuration. \\
\tt 0x3A3 & MRW  &\tt pmpcfg3    & Physical memory protection configuration, RV32 only. \\
& & \multicolumn{1}{c|}{\vdots} & \ \\
\tt 0x3AE & MRW  &\tt pmpcfg14   & Physical memory protection configuration. \\
\tt 0x3AF & MRW  &\tt pmpcfg15   & Physical memory protection configuration, RV32 only. \\
\tt 0x3B0 & MRW  &\tt pmpaddr0   & Physical memory protection address register. \\
\tt 0x3B1 & MRW  &\tt pmpaddr1   & Physical memory protection address register. \\
& & \multicolumn{1}{c|}{\vdots} & \ \\
\tt 0x3EF & MRW  &\tt pmpaddr63  & Physical memory protection address register. \\
\hline
\end{tabular}
\end{center}
\caption{Currently allocated RISC-V machine-level CSR addresses.}
\label{mcsrnames0}
\end{table}

\begin{table}[htb!]
\begin{center}
\begin{tabular}{|l|l|l|l|}
\hline
Number    & Privilege & Name & Description \\
\hline
\multicolumn{4}{|c|}{Machine Counter/Timers} \\
\hline
\tt 0xB00 & MRW  &\tt mcycle         & Machine cycle counter. \\
\tt 0xB02 & MRW  &\tt minstret       & Machine instructions-retired counter. \\
\tt 0xB03 & MRW  &\tt mhpmcounter3   & Machine performance-monitoring counter. \\
\tt 0xB04 & MRW  &\tt mhpmcounter4   & Machine performance-monitoring counter. \\
& & \multicolumn{1}{c|}{\vdots} & \ \\
\tt 0xB1F & MRW  &\tt mhpmcounter31  & Machine performance-monitoring counter. \\
\tt 0xB80 & MRW  &\tt mcycleh        & Upper 32 bits of {\tt mcycle}, RV32 only. \\
\tt 0xB82 & MRW  &\tt minstreth      & Upper 32 bits of {\tt minstret}, RV32 only. \\
\tt 0xB83 & MRW  &\tt mhpmcounter3h  & Upper 32 bits of {\tt mhpmcounter3}, RV32 only. \\
\tt 0xB84 & MRW  &\tt mhpmcounter4h  & Upper 32 bits of {\tt mhpmcounter4}, RV32 only. \\
& & \multicolumn{1}{c|}{\vdots} & \ \\
\tt 0xB9F & MRW  &\tt mhpmcounter31h & Upper 32 bits of {\tt mhpmcounter31}, RV32 only. \\
\hline
\multicolumn{4}{|c|}{Machine Counter Setup} \\
\hline
\tt 0x320 & MRW  &\tt mcountinhibit  & Machine counter-inhibit register. \\
\tt 0x323 & MRW  &\tt mhpmevent3     & Machine performance-monitoring event selector. \\
\tt 0x324 & MRW  &\tt mhpmevent4     & Machine performance-monitoring event selector. \\
& & \multicolumn{1}{c|}{\vdots} & \ \\
\tt 0x33F & MRW  &\tt mhpmevent31    & Machine performance-monitoring event selector. \\
\hline
\multicolumn{4}{|c|}{Debug/Trace Registers (shared with Debug Mode)} \\
\hline
\tt 0x7A0 & MRW &\tt tselect & Debug/Trace trigger register select. \\
\tt 0x7A1 & MRW &\tt tdata1 & First Debug/Trace trigger data register. \\
\tt 0x7A2 & MRW &\tt tdata2 & Second Debug/Trace trigger data register. \\
\tt 0x7A3 & MRW &\tt tdata3 & Third Debug/Trace trigger data register. \\
\tt 0x7A8 & MRW &\tt mcontext & Machine-mode context register. \\
\hline
\multicolumn{4}{|c|}{Debug Mode Registers } \\
\hline
\tt 0x7B0 & DRW &\tt dcsr & Debug control and status register. \\
\tt 0x7B1 & DRW &\tt dpc & Debug PC. \\
\tt 0x7B2 & DRW &\tt dscratch0 & Debug scratch register 0. \\
\tt 0x7B3 & DRW &\tt dscratch1 & Debug scratch register 1. \\
\hline
\end{tabular}
\end{center}
\caption{Currently allocated RISC-V machine-level CSR addresses.}
\label{mcsrnames1}
\end{table}

\clearpage

\section{CSR Field Specifications}


The following definitions and abbreviations are used in specifying the
behavior of fields within the CSRs.

\subsection*{Reserved Writes Preserve Values, Reads Ignore Values (WPRI)}

Some whole read/write fields are reserved for future use.  Software
should ignore the values read from these fields, and should preserve
the values held in these fields when writing values to other fields of
the same register.
For forward compatibility, implementations that do not furnish these fields
must make them read-only zero.
These fields are labeled \wpri\ in the register descriptions.

\begin{commentary}
To simplify the software model, any backward-compatible future
definition of previously reserved fields within a CSR must cope with
the possibility that a non-atomic read/modify/write sequence is used
to update other fields in the CSR.  Alternatively, the original CSR
definition must specify that subfields can only be updated atomically,
which may require a two-instruction clear bit/set bit sequence in
general that can be problematic if intermediate values are not legal.
\end{commentary}

\subsection*{Write/Read Only Legal Values (WLRL)}

Some read/write CSR fields specify behavior for only a subset of
possible bit encodings, with other bit encodings reserved.  Software
should not write anything other than legal values to such a field, and
should not assume a read will return a legal value unless the last
write was of a legal value, or the register has not been written since
another operation (e.g., reset) set the register to a legal value.
These fields are labeled \wlrl\ in the register descriptions.

\begin{commentary}
Hardware implementations need only implement enough state bits to
differentiate between the supported values, but must always return the
complete specified bit-encoding of any supported value when read.
\end{commentary}

Implementations are permitted but not required to raise an illegal
instruction exception if an instruction attempts to write a
non-supported value to a \wlrl\ field.  Implementations can
return arbitrary bit patterns on the read of a \wlrl\ field when the last
write was of an illegal value, but the value returned should
deterministically depend on the illegal written value and
the value of the field prior to the write.

\subsection*{Write Any Values, Reads Legal Values (WARL)}

Some read/write CSR fields are only defined for a subset of bit
encodings, but allow any value to be written while guaranteeing to
return a legal value whenever read.  Assuming that writing the CSR has
no other side effects, the range of supported values can be determined
by attempting to write a desired setting then reading to see if the
value was retained.  These fields are labeled \warl\ in the register
descriptions.

Implementations will not raise an exception on writes of unsupported
values to a \warl\ field.  Implementations can
return any legal value on the read of a \warl\ field when the last
write was of an illegal value, but the legal value returned should
deterministically depend on the illegal written value and
the architectural state of the hart.

\section{CSR Width Modulation}
\label{sec:csrwidthmodulation}

If the width of a CSR is changed (for example, by changing MXLEN or UXLEN, as
described in Section~\ref{xlen-control}), the values of the {\em writable}
fields and bits of the new-width CSR are, unless specified otherwise,
determined from the previous-width CSR as though by this algorithm:

\begin{enumerate}

\item The value of the previous-width CSR is copied to a temporary register of
the same width.

\item For the read-only bits of the previous-width CSR, the bits at the same
positions in the temporary register are set to zeros.

\item The width of the temporary register is changed to the new width. If the
new width $W$ is narrower than the previous width, the least-significant $W$
bits of the temporary register are retained and the more-significant bits are
discarded. If the new width is wider than the previous width, the temporary
register is zero-extended to the wider width.

\item Each writable field of the new-width CSR takes the value of the bits at
the same positions in the temporary register.

\end{enumerate}

Changing the width of a CSR is not a read or write of the CSR and thus
does not trigger any side effects.
