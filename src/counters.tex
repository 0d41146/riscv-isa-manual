\chapter{Counters}
\label{counters}

RISC-V ISAs provide a set of up to 32$\times$64-bit performance counters and
timers that are accessible via unprivileged XLEN read-only CSR
registers {\tt 0xC00}--{\tt 0xC1F} (with the upper 32 bits accessed
via CSR registers {\tt 0xC80}--{\tt 0xC9F} on RV32).  The first three
of these (CYCLE, TIME, and INSTRET) have dedicated functions (cycle
count, real-time clock, and instructions-retired respectively), while
the remaining counters, if implemented, provide programmable event
counting.

\section{Base Counters and Timers}

\vspace{-0.2in}
\begin{center}
\begin{tabular}{M@{}R@{}F@{}R@{}S}
\\
\instbitrange{31}{20} &
\instbitrange{19}{15} &
\instbitrange{14}{12} &
\instbitrange{11}{7} &
\instbitrange{6}{0} \\
\hline
\multicolumn{1}{|c|}{csr} &
\multicolumn{1}{c|}{rs1} &
\multicolumn{1}{c|}{funct3} &
\multicolumn{1}{c|}{rd} &
\multicolumn{1}{c|}{opcode} \\
\hline
12 & 5 & 3 & 5 & 7 \\
RDCYCLE[H]   & 0 & CSRRS  & dest & SYSTEM \\
RDTIME[H]    & 0 & CSRRS  & dest & SYSTEM \\
RDINSTRET[H] & 0 & CSRRS  & dest & SYSTEM \\
\end{tabular}
\end{center}

RV32I provides a number of 64-bit read-only user-level counters, which
are mapped into the 12-bit CSR address space and accessed in 32-bit
pieces using CSRRS instructions.  In RV64I, the CSR instructions can
manipulate 64-bit CSRs.  In particular, the RDCYCLE, RDTIME, and
RDINSTRET pseudoinstructions read the full 64 bits of the {\tt cycle},
{\tt time}, and {\tt instret} counters.  Hence, the RDCYCLEH, RDTIMEH,
and RDINSTRETH instructions are not required in RV64I.

\begin{commentary}
Some execution environments might prohibit access to counters to
impede timing side-channel attacks.
\end{commentary}

The RDCYCLE pseudoinstruction reads the low XLEN bits of the {\tt
  cycle} CSR which holds a count of the number of clock cycles
executed by the processor core on which the hart is running from an
arbitrary start time in the past.  RDCYCLEH is an RV32I-only
instruction that reads bits 63--32 of the same cycle counter.  The
underlying 64-bit counter should never overflow in practice.  The rate
at which the cycle counter advances will depend on the implementation
and operating environment.  The execution environment should provide a
means to determine the current rate (cycles/second) at which the cycle
counter is incrementing.

\begin{commentary}
RDCYCLE is intended to return the number of cycles executed by the
processor core, not the hart.  Precisely defining what is a ``core'' is
difficult given some implementation choices (e.g., AMD Bulldozer).
Precisely defining what is a ``clock cycle'' is also difficult given the
range of implementations (including software emulations), but the
intent is that RDCYCLE is used for performance monitoring along with the
other performance counters.  In particular, where there is one
hart/core, one would expect cycle-count/instructions-retired to
measure CPI for a hart.

Cores don't have to be exposed to software at all, and an implementor
might choose to pretend multiple harts on one physical core are
running on separate cores with one hart/core, and provide separate
cycle counters for each hart.  This might make sense in a simple
barrel processor (e.g., CDC 6600 peripheral processors) where
inter-hart timing interactions are non-existent or minimal.

Where there is more than one hart/core and dynamic multithreading, it
is not generally possible to separate out cycles per hart (especially
with SMT).  It might be possible to define a separate performance
counter that tried to capture the number of cycles a particular hart
was running, but this definition would have to be very fuzzy to cover
all the possible threading implementations.  For example, should we
only count cycles for which any instruction was issued to execution
for this hart, and/or cycles any instruction retired, or include
cycles this hart was occupying machine resources but couldn't execute
due to stalls while other harts went into execution? Likely, ``all of
the above'' would be needed to have understandable performance stats.
This complexity of defining a per-hart cycle count, and also the need
in any case for a total per-core cycle count when tuning multithreaded
code led to just standardizing the per-core cycle counter, which also
happens to work well for the common single hart/core case.

Standardizing what happens during ``sleep'' is not practical given
that what ``sleep'' means is not standardized across execution
environments, but if the entire core is paused (entirely clock-gated
or powered-down in deep sleep), then it is not executing clock cycles,
and the cycle count shouldn't be increasing per the spec.  There are
many details, e.g., whether clock cycles required to reset a processor
after waking up from a power-down event should be counted, and these
are considered execution-environment-specific details.

Even though there is no precise definition that works for all
platforms, this is still a useful facility for most platforms, and an
imprecise, common, ``usually correct'' standard here is better than no
standard.  The intent of RDCYCLE was primarily performance
monitoring/tuning, and the specification was written with that goal in
mind.
\end{commentary}

The RDTIME pseudoinstruction reads the low XLEN bits of the {\tt
  time} CSR, which counts wall-clock real time that has passed from an
arbitrary start time in the past.  RDTIMEH is an RV32I-only instruction
that reads bits 63--32 of the same real-time counter.  The underlying 64-bit
counter should never overflow in practice.  The execution environment
should provide a means of determining the period of the real-time
counter (seconds/tick).  The period must be constant.  The
real-time clocks of all harts in a single user application
should be synchronized to within one tick of the real-time clock.  The
environment should provide a means to determine the accuracy of the
clock.

\begin{commentary}
On some simple platforms, cycle count might represent a valid
implementation of RDTIME, but in this case, platforms should implement
the RDTIME instruction as an alias for RDCYCLE to make code more
portable, rather than using RDCYCLE to measure wall-clock time.
\end{commentary}

The RDINSTRET pseudoinstruction reads the low XLEN bits of the {\tt
  instret} CSR, which counts the number of instructions retired by
this hart from some arbitrary start point in the past.  RDINSTRETH is
an RV32I-only instruction that reads bits 63--32 of the same
instruction counter. The underlying 64-bit counter should never
overflow in practice.

The following code sequence will read a valid 64-bit cycle counter value into
{\tt x3}:{\tt x2}, even if the counter overflows its lower half between reading its upper
and lower halves.

\begin{figure}[h!]
\begin{center}
\begin{verbatim}
    again:
        rdcycleh     x3
        rdcycle      x2
        rdcycleh     x4
        bne          x3, x4, again
\end{verbatim}
\end{center}
\caption{Sample code for reading the 64-bit cycle counter in RV32.}
\label{rdcycle}
\end{figure}

\begin{commentary}
We recommend provision of these basic counters in implementations
as they are essential for basic performance analysis, adaptive and
dynamic optimization, and to allow an application to work with
real-time streams.  Additional counters should be provided to help
diagnose performance problems and these should be made accessible from
user-level application code with low overhead.

We required the counters be 64 bits wide, even on RV32, as otherwise
it is very difficult for software to determine if values have
overflowed.  For a low-end implementation, the upper 32 bits of each
counter can be implemented using software counters incremented by a
trap handler triggered by overflow of the lower 32 bits.  The sample
code described above shows how the full 64-bit width value can be
safely read using the individual 32-bit instructions.

In some applications, it is important to be able to read multiple
counters at the same instant in time.  When run under a multitasking
environment, a user thread can suffer a context switch while
attempting to read the counters.  One solution is for the user thread
to read the real-time counter before and after reading the other
counters to determine if a context switch occurred in the middle of the
sequence, in which case the reads can be retried.  We considered
adding output latches to allow a user thread to snapshot the counter
values atomically, but this would increase the size of the user
context, especially for implementations with a richer set of counters.
\end{commentary}

\section{Hardware Performance Counters}

There is CSR space allocated for 29 additional unprivileged 64-bit
hardware performance counters, {\tt hpmcounter3}--{\tt hpmcounter31}.
For RV32, the upper 32 bits of these performance counters is
accessible via additional CSRs {\tt hpmcounter3h}--{\tt
  hpmcounter31h}.  These counters count platform-specific events and
are configured via additional privileged registers.  The number and
width of these additional counters, and the set of events they count
is platform-specific.

\begin{commentary}
  The privileged architecture manual describes the privileged CSRs
  controlling access to these counters and to set the events to be
  counted.

  It would be useful to eventually standardize event settings to count
  ISA-level metrics, such as the number of floating-point instructions
  executed for example, and possibly a few common microarchitectural
  metrics, such as ``L1 instruction cache misses''.
\end{commentary}
