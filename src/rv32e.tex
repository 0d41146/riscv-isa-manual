\chapter{RV32E and RV64E Base Integer Instruction Sets, Version 1.95}
\label{rv32e}

This chapter describes a proposal for the RV32E and RV64E base integer
instruction sets, which are reduced versions of RV32I and RV64I
respectively designed for microcontrollers in embedded systems.  The
only change is to reduce the number of integer registers to 16.  This
chapter only outlines the differences between RV32E/RV64E and
RV32I/RV64I, and so should be read after Chapters~\ref{rv32} and
\ref{rv64}.

\begin{commentary}
RV32E was designed to provide an even smaller base core for embedded
microcontrollers.  There is also interest in RV64E for
microcontrollers within large SoC designs, and to reduce context state
for highly threaded 64-bit processors.

RV32E and RV64E can be combined with all current standard extensions.
\end{commentary}

\section{RV32E and RV64E Programmers' Model}

RV32E and RV64E reduce the integer register count to 16
general-purpose registers, ({\tt x0}--{\tt x15}), where {\tt x0} is a
dedicated zero register.

\begin{commentary}
We have found that in the small RV32I core implementations, the upper
16 registers consume around one quarter of the total area of the core
excluding memories, thus their removal saves around 25\% core area
with a corresponding core power reduction.
\end{commentary}

\section{RV32E and RV64E Instruction Set Encoding}

RV32E and RV64E use the same instruction-set encoding as RV32I and
RV64I respectively, except that only registers {\tt x0}--{\tt x15} are
provided.  All encodings specifying the other registers {\tt x16}--{\tt
  x31} are reserved.

\begin{commentary}
The previous draft of this chapter made all encodings using the {\tt
  x16}--{\tt x31} registers available as custom.  This version takes a
more conservative approach, making these reserved so that they can be
allocated between custom space or new standard encodings at a later
date.
\end{commentary}
