\chapter{Supervisor-Level ISA, Version \privrev}
\label{supervisor}

This chapter describes the RISC-V supervisor-level architecture, which
contains a common core that is used with various supervisor-level
address translation and protection schemes.
Supervisor-level code relies on a supervisor execution environment to
initialize the environment and enter the supervisor code at an entry
point defined by the system binary interface (SBI).  The SBI also
defines function entry points that provide supervisor environment
services for supervisor-level code.

\begin{commentary}
Supervisor mode is deliberately restricted in terms of interactions
with underlying physical hardware, such as physical memory and device
interrupts, to support clean virtualization.
\end{commentary}

\section{Supervisor CSRs}

A number of CSRs are provided for the supervisor.

\begin{commentary}
The supervisor should only view CSR state that should be visible to a
supervisor-level operating system.  In particular, there is no
information about the existence (or non-existence) of higher privilege
levels (hypervisor or machine) visible in the CSRs accessible by the
supervisor.

Many supervisor CSRs are a subset of the equivalent machine-mode CSR,
and the machine-mode chapter should be read first to help understand
the supervisor-level CSR descriptions.
\end{commentary}

\subsection{Supervisor Status Register (\tt sstatus)}
\label{sstatus}


The {\tt sstatus} register is an XLEN-bit read/write register
formatted as shown in Figure~\ref{sstatusreg-rv32} for RV32 and
Figure~\ref{sstatusreg} for RV64 and RV128.  The {\tt sstatus}
register keeps track of the processor's current operating state.

\begin{figure*}[h!]
{\footnotesize
\begin{center}
\setlength{\tabcolsep}{4pt}
\begin{tabular}{cWcccccWcWccWcc}
\\
\instbit{31} &
\instbitrange{30}{20} &
\instbit{19} &
\instbit{18} &
\instbit{17} &
\instbitrange{16}{15} &
\instbitrange{14}{13} &
\instbitrange{12}{9} &
\instbit{8} &
\instbitrange{7}{6} &
\instbit{5} &
\instbit{4} &
\instbitrange{3}{2} &
\instbit{1} &
\instbit{0} \\
\hline
\multicolumn{1}{|c|}{SD} &
\multicolumn{1}{c|}{\wpri} &
\multicolumn{1}{c|}{MXR} &
\multicolumn{1}{c|}{SUM} &
\multicolumn{1}{c|}{\wpri} &
\multicolumn{1}{c|}{XS[1:0]} &
\multicolumn{1}{c|}{FS[1:0]} &
\multicolumn{1}{c|}{\wpri} &
\multicolumn{1}{c|}{SPP} &
\multicolumn{1}{c|}{\wpri} &
\multicolumn{1}{c|}{SPIE} &
\multicolumn{1}{c|}{UPIE} &
\multicolumn{1}{c|}{\wpri} &
\multicolumn{1}{c|}{SIE}  &
\multicolumn{1}{c|}{UIE}
\\
\hline
1 & 11 & 1 & 1 & 1 & 2 & 2 & 4 & 1 & 2 & 1 & 1 & 2 & 1 & 1 \\
\end{tabular}
\end{center}
}
\vspace{-0.1in}
\caption{Supervisor-mode status register ({\tt sstatus}) for RV32.}
\label{sstatusreg-rv32}
\end{figure*}

\begin{figure*}[h!]
{\footnotesize
\begin{center}
\setlength{\tabcolsep}{4pt}
\begin{tabular}{cScScccc}
\\
\instbit{XLEN-1} &
\instbitrange{XLEN-2}{34} &
\instbitrange{33}{32} &
\instbitrange{31}{20} &
\instbit{19} &
\instbit{18} &
\instbit{17} &
 \\
\hline
\multicolumn{1}{|c|}{SD} &
\multicolumn{1}{c|}{\wpri} &
\multicolumn{1}{c|}{UXL} &
\multicolumn{1}{c|}{\wpri} &
\multicolumn{1}{c|}{MXR} &
\multicolumn{1}{c|}{SUM} &
\multicolumn{1}{c|}{\wpri} &
 \\
\hline
1 & XLEN-35 & 2 & 12 & 1 & 1 & 1 & \\
\end{tabular}
\begin{tabular}{ccccccccccc}
\\
&
\instbitrange{16}{15} &
\instbitrange{14}{13} &
\instbitrange{12}{9} &
\instbit{8} &
\instbitrange{7}{6} &
\instbit{5} &
\instbit{4} &
\instbitrange{3}{2} &
\instbit{1} &
\instbit{0} \\
\hline
 &
\multicolumn{1}{c|}{XS[1:0]} &
\multicolumn{1}{|c|}{FS[1:0]} &
\multicolumn{1}{c|}{\wpri} &
\multicolumn{1}{c|}{SPP} &
\multicolumn{1}{c|}{\wpri} &
\multicolumn{1}{c|}{SPIE} &
\multicolumn{1}{c|}{UPIE} &
\multicolumn{1}{c|}{\wpri} &
\multicolumn{1}{c|}{SIE} &
\multicolumn{1}{c|}{UIE} \\
\hline
 & 2 & 2 & 4 & 1 & 2 & 1 & 1 & 2 & 1 & 1 \\
\end{tabular}
\end{center}
}
\vspace{-0.1in}
\caption{Supervisor-mode status register ({\tt sstatus}) for RV64 and RV128.}
\label{sstatusreg}
\end{figure*}

The SPP bit indicates the privilege level at which a hart was executing before
entering supervisor mode.  When a trap is taken, SPP is set to 0 if the trap
originated from user mode, or 1 otherwise.  When an SRET instruction
(see Section~\ref{otherpriv}) is executed to return from the trap handler, the
privilege level is set to user mode if the SPP bit is 0, or supervisor mode if
the SPP bit is 1; SPP is then set to 0.

The SIE bit enables or disables all interrupts in supervisor mode.
When SIE is clear, interrupts are not taken while in supervisor mode.
When the hart is running in user-mode, the value in SIE is ignored, and
supervisor-level interrupts are enabled.  The supervisor can disable
individual interrupt sources using the {\tt sie} register.

The SPIE bit indicates whether supervisor interrupts were enabled prior to
trapping into supervisor mode.  When a trap is taken into supervisor
mode, SPIE is set to SIE, and SIE is set to 0.  When an SRET instruction is
executed, SIE is set to SPIE, then SPIE is set to 1.

The UIE bit enables or disables user-mode interrupts.  User-level interrupts
are enabled only if UIE is set and the hart is running in user-mode.  The UPIE
bit indicates whether user-level interrupts were enabled prior to taking
a user-level trap.  When a URET instruction is executed, UIE is
set to UPIE, and UPIE is set to 1.  User-level interrupts are optional.  If
omitted, the UIE and UPIE bits are hardwired to zero.

\begin{commentary}
The {\tt sstatus} register is a subset of the {\tt mstatus} register.  In
a straightforward implementation, reading or writing any field in {\tt
sstatus} is equivalent to reading or writing the homonymous field in
{\tt mstatus}.
\end{commentary}

\subsection{Base ISA Control in {\tt sstatus} Register}

The UXL field controls the value of XLEN for U-mode, termed {\em U-XLEN},
which may differ from the value of XLEN for S-mode, termed {\em S-XLEN}.  The
encoding of UXL is the same as that of the MXL field of {\tt misa}, shown in
Table~\ref{misabase}.

For RV32 systems, the UXL field does not exist, and U-XLEN~=~32.  For RV64 and
RV128 systems, it is a \warl\ field that encodes the current value of U-XLEN.
In particular, the implementation may hardwire UXL so that U-XLEN~=~S-XLEN.

If U-XLEN~$\ne$~S-XLEN, instructions executed in the narrower mode must ignore
source register operand bits above the configured XLEN, and must sign-extend
results to fill the widest supported XLEN in the destination register.

\subsection{Memory Privilege in {\tt sstatus} Register}
\label{sec:sum}

The MXR (Make eXecutable Readable) bit modifies the privilege with which loads
access virtual memory.  When MXR=0, only loads from pages marked readable (R=1
in Figure~\ref{sv32pte}) will succeed.  When MXR=1, loads from pages marked
either readable or executable (R=1 or X=1) will succeed.  MXR has no effect
when page-based virtual memory is not in effect.

The SUM (permit Supervisor User Memory access) bit modifies the privilege with
which S-mode loads and stores access virtual memory.
When SUM=0, S-mode memory accesses to pages that are accessible by U-mode (U=1
in Figure~\ref{sv32pte}) will fault.  When SUM=1, these accesses are permitted.
SUM has no effect when page-based virtual memory is not in effect, nor when
executing in U-mode.  Note that S-mode can never execute instructions from user
pages, regardless of the state of SUM.

\begin{commentary}
The SUM mechanism prevents supervisor software from inadvertently accessing
user memory.  Operating systems can execute the majority of code with SUM clear;
the few code segments that should access user memory can temporarily set
SUM.

The SUM mechanism does not avail S-mode software of permission to execute
instructions in user code pages.  Legitimate uses cases for execution from
user memory in supervisor context are rare in general and nonexistent in POSIX
environments.  However, bugs in supervisors that lead to arbitrary code
execution are much easier to exploit if the supervisor exploit code can be
stored in a user buffer at a virtual address chosen by an attacker.

Some non-POSIX single address space operating systems do allow certain
privileged software to partially execute in supervisor mode, while most
programs run in user mode, all in a shared address space.  This use case can
be realized by mapping the physical code pages at multiple virtual addresses
with different permissions, possibly with the assistance of the
instruction page-fault handler to direct supervisor software to use the
alternate mapping.
\end{commentary}

\subsection{Supervisor Trap Vector Base Address Register ({\tt stvec})}

The {\tt stvec} register is an XLEN-bit read/write register that holds
trap vector configuration, consisting of a vector base address (BASE) and a
vector mode (MODE).

\begin{figure*}[h!]
{\footnotesize
\begin{center}
\begin{tabular}{J@{}R}
\instbitrange{XLEN-1}{2} &
\instbitrange{1}{0} \\
\hline
\multicolumn{1}{|c|}{BASE[XLEN-1:2] (\warl)} & 
\multicolumn{1}{c|}{MODE (\warl)} \\
\hline
XLEN-2 & 2 \\
\end{tabular}
\end{center}
}
\vspace{-0.1in}
\caption{Supervisor trap vector base address register ({\tt stvec}).}
\label{stvecreg}
\end{figure*}

The BASE field in {\tt stvec} is a \warl\ field that can hold any valid virtual
or physical address, subject to the following alignment constraints: the
address must always be at least 4-byte aligned, and the MODE
setting may impose additional alignment constraints on the value in the BASE
field.

\begin{table*}[h!]
\begin{center}
\begin{tabular}{|r|c|l|}
\hline
Value & Name & Description \\
\hline	 
0      & Direct   & All exceptions set {\tt pc} to BASE. \\
1      & Vectored & Asynchronous interrupts set {\tt pc} to BASE+4$\times$cause. \\
$\ge$2 & --- & {\em Reserved} \\
\hline
\end{tabular}
\end{center}
\caption{Encoding of {\tt stvec} MODE field.}
\label{stvec-mode}
\end{table*}

The encoding of the MODE field is shown in Table~\ref{stvec-mode}.  When
MODE=Direct, all traps into supervisor mode cause the {\tt pc} to be set to the
address in the BASE field.  When MODE=Vectored, all synchronous exceptions
into supervisor mode cause the {\tt pc} to be set to the address in the BASE
field, whereas interrupts cause the {\tt pc} to be set to the address in
the BASE field plus four times the interrupt cause number.  For example,
a supervisor-mode timer interrupt (see Table~\ref{scauses}) causes the {\tt pc}
to be set to BASE+{\tt 0x14}.
Setting MODE=Vectored may impose an additional alignment constraint on BASE,
requiring up to $4\times$XLEN-byte alignment.

\begin{commentary}
When vectored interrupts are enabled, interrupt cause 0, which corresponds to
user-mode software interrupts, are vectored to the same location as
synchronous exceptions.  This ambiguity does not arise in practice for
supervisor software, since user-mode software interrupts are either disabled
or delegated to user mode.
\end{commentary}

\subsection{Supervisor Interrupt Registers ({\tt sip} and {\tt sie})}

The {\tt sip} register is an XLEN-bit read/write register containing
information on pending interrupts, while {\tt sie} is the corresponding
XLEN-bit read/write register containing interrupt enable bits.

\begin{figure*}[h!]
{\footnotesize
\begin{center}
\setlength{\tabcolsep}{4pt}
\begin{tabular}{EccFccFcc}
\instbitrange{XLEN-1}{10} &
\instbit{9} &
\instbit{8} &
\instbitrange{7}{6} &
\instbit{5} &
\instbit{4} &
\instbitrange{3}{2} &
\instbit{1} &
\instbit{0} \\
\hline
\multicolumn{1}{|c|}{\wiri} &
\multicolumn{1}{c|}{SEIP} &
\multicolumn{1}{c|}{UEIP} &
\multicolumn{1}{c|}{\wiri} &
\multicolumn{1}{c|}{STIP} &
\multicolumn{1}{c|}{UTIP} &
\multicolumn{1}{c|}{\wiri} &
\multicolumn{1}{c|}{SSIP} &
\multicolumn{1}{c|}{USIP} \\
\hline
XLEN-10 & 1 & 1 & 2 & 1 & 1 & 2 & 1 & 1 \\
\end{tabular}
\end{center}
}
\vspace{-0.1in}
\caption{Supervisor interrupt-pending register ({\tt sip}).}
\label{sipreg}
\end{figure*}

\begin{figure*}[h!]
{\footnotesize
\begin{center}
\setlength{\tabcolsep}{4pt}
\begin{tabular}{EccFccFcc}
\instbitrange{XLEN-1}{10} &
\instbit{9} &
\instbit{8} &
\instbitrange{7}{6} &
\instbit{5} &
\instbit{4} &
\instbitrange{3}{2} &
\instbit{1} &
\instbit{0} \\
\hline
\multicolumn{1}{|c|}{\wpri} &
\multicolumn{1}{c|}{SEIE} &
\multicolumn{1}{c|}{UEIE} &
\multicolumn{1}{c|}{\wpri} &
\multicolumn{1}{c|}{STIE} &
\multicolumn{1}{c|}{UTIE} &
\multicolumn{1}{c|}{\wpri} &
\multicolumn{1}{c|}{SSIE} &
\multicolumn{1}{c|}{USIE} \\
\hline
XLEN-10 & 1 & 1 & 2 & 1 & 1 & 2 & 1 & 1 \\
\end{tabular}
\end{center}
}
\vspace{-0.1in}
\caption{Supervisor interrupt-enable register ({\tt sie}).}
\label{siereg}
\end{figure*}

Three types of interrupts are defined: software interrupts, timer interrupts,
and external interrupts.  A supervisor-level software interrupt is triggered
on the current hart by writing 1 to its supervisor software interrupt-pending
(SSIP) bit in the {\tt sip} register.  A pending supervisor-level software
interrupt can be cleared by writing 0 to the SSIP bit in {\tt sip}.
Supervisor-level software interrupts are disabled when the SSIE bit in the
{\tt sie} register is clear.

Interprocessor interrupts are sent to other harts by means of SBI
calls, which will ultimately cause the SSIP bit to be set in the
recipient hart's {\tt sip} register.

A user-level software interrupt is triggered on the current hart by writing
1 to its user software interrupt-pending (USIP) bit in the {\tt sip} register.
A pending user-level software interrupt can be cleared by writing 0 to the
USIP bit in {\tt sip}.  User-level software interrupts are disabled when the
USIE bit in the {\tt sie} register is clear.  If user-level interrupts are not
supported, USIP and USIE are hardwired to zero.

All bits besides SSIP, USIP, and UEIP in the {\tt sip} register are read-only.

A supervisor-level timer interrupt is pending if the STIP bit in the {\tt sip}
register is set.  Supervisor-level timer interrupts are disabled when the STIE
bit in the {\tt sie} register is clear.  An SBI call to the SEE may be used to
clear the pending timer interrupt.

A user-level timer interrupt is pending if the UTIP bit in the {\tt sip}
register is set.  User-level timer interrupts are disabled when the UTIE bit
in the {\tt sie} register is clear.  If user-level interrupts are supported,
the ABI should provide a facility for scheduling timer interrupts in terms of
real-time counter values.  If user-level interrupts are not supported, UTIP
and UTIE are hardwired to zero.

A supervisor-level external interrupt is pending if the SEIP bit in the
{\tt sip} register is set.  Supervisor-level external interrupts are disabled
when the SEIE bit in the {\tt sie} register is clear.  The SBI should provide
facilities to mask, unmask, and query the cause of external interrupts.

The UEIP field in {\tt sip} contains a single read-write bit.  UEIP
may be written by S-mode software to indicate to U-mode that an
external interrupt is pending.  Additionally, the platform-level
interrupt controller may generate user-level external interrupts.  The
logical-OR of the software-writeable bit and the signal from the
external interrupt controller are used to generate external interrupts
for user mode.  When the UEIP bit is read with a CSRRW, CSRRS, or
CSRRC instruction, the value returned in the {\tt rd} destination
register contains the logical-OR of the software-writable bit and the
interrupt signal from the interrupt controller.  However, the value
used in the read-modify-write sequence of a CSRRS or CSRRC instruction
is only the software-writable UEIP bit, ignoring the interrupt value
from the external interrupt controller.

\begin{commentary}
  Analogous to SEIP, the UEIP field behavior is designed to allow a
  higher privilege layer to mimic external interrupts without losing
  any real external interrupts.
\end{commentary}

User-level external interrupts are disabled when the UEIE bit in the {\tt sie}
register is clear.  If the N extension for user-level interrupts is not
implemented, UEIP and UEIE are hardwired to zero.

\begin{commentary}
The {\tt sip} and {\tt sie} registers are subsets of the {\tt mip} and {\tt
mie} registers.  Reading any field, or writing any writable field, of {\tt
sip}/{\tt sie} effects a read or write of the homonymous field of {\tt
mip}/{\tt mie}.
\end{commentary}

\subsection{Supervisor Timers and Performance Counters}

Supervisor software uses the same hardware performance monitoring facility
as user-mode software, including the {\tt time}, {\tt cycle}, and {\tt instret}
CSRs.  The SBI should provide a mechanism to modify the
counter values.

The SBI must provide a facility for scheduling timer interrupts in terms
of the real-time counter, {\tt time}.

\subsection{Counter-Enable Register ({\tt scounteren})}

\begin{figure*}[h!]
{\footnotesize
\begin{center}
\setlength{\tabcolsep}{4pt}
\begin{tabular}{cccMcccccc}
\instbit{31} &
\instbit{30} &
\instbit{29} &
\instbitrange{28}{6} &
\instbit{5} &
\instbit{4} &
\instbit{3} &
\instbit{2} &
\instbit{1} &
\instbit{0} \\
\hline
\multicolumn{1}{|c|}{HPM31} &
\multicolumn{1}{c|}{HPM30} &
\multicolumn{1}{c|}{HPM29} &
\multicolumn{1}{c|}{...} &
\multicolumn{1}{c|}{HPM5} &
\multicolumn{1}{c|}{HPM4} &
\multicolumn{1}{c|}{HPM3} &
\multicolumn{1}{c|}{IR} &
\multicolumn{1}{c|}{TM} &
\multicolumn{1}{c|}{CY} \\
\hline
1 & 1 & 1 & 23 & 1 & 1 & 1 & 1 & 1 & 1 \\
\end{tabular}
\end{center}
}
\vspace{-0.1in}
\caption{Counter-enable register ({\tt scounteren}).}
\label{scounteren}
\end{figure*}

The counter-enable register {\tt scounteren} controls
the availability of the
hardware performance monitoring counters to U-mode.

When the CY, TM, IR, or HPM{\em n} bit in the {\tt scounteren} register is
clear, attempts to read the {\tt cycle}, {\tt time}, {\tt instret}, or
{\tt hpmcounter{\em n}} register while executing in U-mode
will cause an illegal instruction exception.  When one of these bits is set,
access to the corresponding register is permitted.

{\tt scounteren} must be implemented.  However, any of the bits may contain
a hardwired value of zero, indicating reads to the corresponding counter will
cause an exception when executing in U-mode.
Hence, they are effectively \warl\ fields.

\subsection{Supervisor Scratch Register ({\tt sscratch})}

The {\tt sscratch} register is an XLEN-bit read/write register,
dedicated for use by the supervisor.  Typically, {\tt sscratch} is
used to hold a pointer to the hart-local supervisor context while the
hart is executing user code.  At the beginning of a trap handler, {\tt
  sscratch} is swapped with a user register to provide an initial
working register.

\begin{figure}[h!]
{\footnotesize
\begin{center}
\begin{tabular}{@{}J}
\instbitrange{XLEN-1}{0} \\
\hline
\multicolumn{1}{|c|}{\tt sscratch} \\
\hline
XLEN \\
\end{tabular}
\end{center}
}
\vspace{-0.1in}
\caption{Supervisor Scratch Register.}
\label{kregs}
\end{figure}

\subsection{Supervisor Exception Program Counter ({\tt sepc})}

{\tt sepc} is an XLEN-bit read/write register formatted as shown in
Figure~\ref{epcreg}.  The low bit of {\tt sepc} ({\tt sepc[0]}) is
always zero.  On implementations that do not support instruction-set
extensions with 16-bit instruction alignment, the two low bits ({\tt
  sepc[1:0]}) are always zero.

{\tt sepc} is a \warl\ register that must be able to hold all valid physical
and virtual addresses.  It need not be capable of holding all possible invalid
addresses.  Implementations may convert some invalid address patterns into
other invalid addresses prior to writing them to {\tt sepc}.

When a trap is taken into S-mode, {\tt sepc} is written with the virtual
address of the instruction that encountered the exception.  Otherwise,
{\tt sepc} is never written by the implementation, though it may be
explicitly written by software.

\begin{figure}[h!]
{\footnotesize
\begin{center}
\begin{tabular}{@{}J}
\instbitrange{XLEN-1}{0} \\
\hline
\multicolumn{1}{|c|}{\tt sepc} \\
\hline
XLEN \\
\end{tabular}
\end{center}
}
\vspace{-0.1in}
\caption{Supervisor exception program counter register.}
\label{epcreg}
\end{figure}

\subsection{Supervisor Cause Register ({\tt scause})}

The {\tt scause} register is an XLEN-bit read-write register formatted as
shown in Figure~\ref{scausereg}.  When a trap is taken into S-mode, {\tt
scause} is written with a code indicating the event that caused the trap.
Otherwise, {\tt scause} is never written by the implementation, though it may be
explicitly written by software.

The Interrupt bit in the {\tt scause} register is set if the
trap was caused by an interrupt. The Exception Code field
contains a code identifying the last exception.  Table~\ref{scauses}
lists the possible exception codes for the current supervisor ISAs, in
descending order of priority.  The Exception Code is an \wlrl\ field,
so is only guaranteed to hold supported exception codes.

\begin{figure*}[h!]
{\footnotesize
\begin{center}
\begin{tabular}{c@{}U}
\instbit{XLEN-1} &
\instbitrange{XLEN-2}{0} \\
\hline
\multicolumn{1}{|c|}{Interrupt} &
\multicolumn{1}{c|}{Exception Code (\wlrl)} \\
\hline
1 & XLEN-1 \\
\end{tabular}
\end{center}
}
\vspace{-0.1in}
\caption{Supervisor Cause register {\tt scause}.}
\label{scausereg}
\end{figure*}

\begin{table*}[h!]
\begin{center}
\begin{tabular}{|r|r|l|l|}

  \hline
  Interrupt & Exception Code  & Description \\
  \hline	 
  1         & 0               & User software interrupt \\
  1         & 1               & Supervisor software interrupt \\
  1         & 2--3            & {\em Reserved} \\
  1         & 4               & User timer interrupt \\
  1         & 5               & Supervisor timer interrupt \\
  1         & 6--7            & {\em Reserved} \\
  1         & 8               & User external interrupt \\
  1         & 9               & Supervisor external interrupt \\
  1         & $\ge$10          & {\em Reserved} \\ \hline
  0         & 0               & Instruction address misaligned \\
  0         & 1               & Instruction access fault \\
  0         & 2               & Illegal instruction \\   
  0         & 3               & Breakpoint \\
  0         & 4               & Load address misaligned \\
  0         & 5               & Load access fault \\
  0         & 6               & Store/AMO address misaligned \\
  0         & 7               & Store/AMO access fault \\
  0         & 8               & Environment call \\
  0         & 9--11           & {\em Reserved} \\
  0         & 12              & Instruction page fault \\
  0         & 13              & Load page fault \\
  0         & 14              & {\em Reserved} \\
  0         & 15              & Store/AMO page fault \\
  0         & $\ge$16         & {\em Reserved} \\
  \hline
\end{tabular}
\end{center}
\caption{Supervisor cause register ({\tt scause}) values after trap.}
\label{scauses}
\end{table*}

\subsection{Supervisor Trap Value ({\tt stval}) Register}

The {\tt stval} register is an XLEN-bit read-write register formatted as shown
in Figure~\ref{stvalreg}.  When a trap is taken into S-mode, {\tt stval} is
written with exception-specific information to assist software in handling the
trap.  Otherwise, {\tt stval} is never written by the implementation, though
it may be explicitly written by software.


When a hardware breakpoint is triggered, or an instruction-fetch, load, or
store address-misaligned, access, or page-fault exception occurs, {\tt stval}
is written with the faulting address.  On an illegal instruction trap, {\tt
stval} may be written with the first XLEN or ILEN bits of the faulting
instruction as described below.  For other exceptions, {\tt stval} is set to
zero, but a future standard may redefine {\tt stval}'s setting for other
exceptions.

\begin{figure}[h!]
{\footnotesize
\begin{center}
\begin{tabular}{@{}J}
\instbitrange{XLEN-1}{0} \\
\hline
\multicolumn{1}{|c|}{\tt stval} \\
\hline
XLEN \\
\end{tabular}
\end{center}
}
\vspace{-0.1in}
\caption{Supervisor Trap Value register.}
\label{stvalreg}
\end{figure}

For instruction-fetch access faults and page faults on RISC-V systems with
variable-length instructions, {\tt stval} will point to the portion
of the instruction that caused the fault while {\tt sepc} will point
to the beginning of the instruction.

The {\tt stval} register can optionally also be used to return the
faulting instruction bits on an illegal instruction exception ({\tt
  sepc} points to the faulting instruction in memory).

If this feature is not provided, then {\tt stval} is set to zero on
an illegal instruction fault.

If this feature is provided, after an illegal instruction trap, {\tt stval}
will contain the shortest of:
\begin{compactitem}
\item the actual faulting instruction
\item the first ILEN bits of the faulting instruction
\item the first XLEN bits of the faulting instruction
\end{compactitem}
The value loaded into {\tt stval} is right-justified and all unused upper
bits are cleared to zero.

{\tt stval} is a \warl\ register that must be able to hold all valid physical
and virtual addresses and the value 0.  It need not be capable of holding all
possible invalid addresses.  Implementations may convert some invalid address
patterns into other invalid addresses prior to writing them to {\tt stval}.
If the feature to return the faulting instruction bits is implemented, {\tt
stval} must also be able to hold all values less than $2^N$, where $N$ is the
smaller of XLEN and the width of the longest supported instruction.

\subsection{Supervisor Address Translation and Protection ({\tt satp}) Register}
\label{sec:satp}

The {\tt satp} register is an XLEN-bit read/write register, formatted as shown
in Figure~\ref{rv32satp} for RV32 and Figure~\ref{rv64satp}, which
controls supervisor-mode address translation and protection.
This register holds the physical page number (PPN) of the root page
table, i.e., its supervisor physical address divided by \wunits{4}{KiB};
an address space identifier (ASID), which facilitates address-translation
fences on a per-address-space basis; and the MODE field, which selects the
current address-translation scheme.

\begin{figure}[h!]
{\footnotesize
\begin{center}
\begin{tabular}{c@{}E@{}K}
\instbit{31} &
\instbitrange{30}{22} &
\instbitrange{21}{0} \\
\hline
\multicolumn{1}{|c|}{{\tt MODE} (\warl)} &
\multicolumn{1}{|c|}{{\tt ASID} (\warl)} &
\multicolumn{1}{|c|}{{\tt PPN}  (\warl)} \\
\hline
1 & 9 & 22 \\
\end{tabular}
\end{center}
}
\vspace{-0.1in}
\caption{RV32 Supervisor address translation and protection register {\tt satp}.}
\label{rv32satp}
\end{figure}

\begin{commentary}
Storing a PPN in {\tt satp}, rather than a physical address, supports
a physical address space larger than \wunits{4}{GiB} for RV32.
\end{commentary}

\begin{figure}[h!]
{\footnotesize
\begin{center}
\begin{tabular}{@{}S@{}T@{}U}
\instbitrange{63}{60} &
\instbitrange{59}{44} &
\instbitrange{43}{0} \\
\hline
\multicolumn{1}{|c|}{{\tt MODE} (\warl)} &
\multicolumn{1}{|c|}{{\tt ASID} (\warl)} &
\multicolumn{1}{|c|}{{\tt PPN}  (\warl)} \\
\hline
4 & 16 & 44 \\
\end{tabular}
\end{center}
}
\vspace{-0.1in}
\caption{RV64 Supervisor address translation and protection register {\tt satp}, for MODE
values Sv39 and Sv48.}
\label{rv64satp}
\end{figure}

\begin{samepage-commentary}
We store the ASID and the page table base address in the same CSR to allow the
pair to be changed atomically on a context switch.  Swapping them
non-atomically could pollute the old virtual address space with new
translations, or vice-versa.  This approach also slightly reduces the cost of
a context switch.
\end{samepage-commentary}

Table~\ref{tab:satp-mode} shows the encodings of the MODE field for RV32 and
RV64.  When MODE=Bare, supervisor virtual addresses are equal to
supervisor physical addresses, and there is no additional memory protection
beyond the physical memory protection scheme described in
Section~\ref{sec:pmp}.  In this case, the remaining fields in {\tt satp}
have no effect.

For RV32, the only other valid setting for MODE is Sv32, a paged
virtual-memory scheme described in Section~\ref{sec:sv32}.

For RV64, two paged virtual-memory schemes are defined: Sv39 and Sv48,
described in Sections~\ref{sec:sv39} and \ref{sec:sv48}, respectively.
Two additional schemes, Sv57 and Sv64, will be defined in a later version
of this specification.  The remaining MODE settings are reserved
for future use and may define different interpretations of the other fields in
{\tt satp}.

Implementations are not required to support all MODE settings,
and if {\tt satp} is written with an unsupported MODE, the entire write has
no effect; no fields in {\tt satp} are modified.

\begin{table}[h]
\begin{center}
\begin{tabular}{|c|c|l|}
\hline
\multicolumn{3}{|c|}{RV32} \\
\hline
Value  & Name & Description \\
\hline
0       & Bare  & No translation or protection. \\
1       & Sv32  & Page-based 32-bit virtual addressing. \\
\hline \hline
\multicolumn{3}{|c|}{RV64} \\
\hline
Value  & Name & Description \\
\hline  
0       & Bare  & No translation or protection. \\
1--7    & ---   & {\em Reserved} \\
8       & Sv39  & Page-based 39-bit virtual addressing. \\
9       & Sv48  & Page-based 48-bit virtual addressing. \\
10      & {\em Sv57} & {\em Reserved for page-based 57-bit virtual addressing.} \\
11      & {\em Sv64} & {\em Reserved for page-based 64-bit virtual addressing.} \\
12--15  & ---   & {\em Reserved} \\
\hline
\end{tabular}
\end{center}
\caption{Encoding of {\tt satp} MODE field.}
\label{tab:satp-mode}
\end{table}

The number of supervisor physical address bits is implementation-defined; any
unimplemented address bits are hardwired to zero in the {\tt satp} register.
The number of ASID bits is also implementation-defined and may be zero.  The
number of implemented ASID bits, termed {\mbox {\em ASIDLEN}}, may be
determined by writing one to every bit position in the ASID field, then
reading back the value in {\tt satp} to see which bit positions in the ASID
field hold a one.  The least-significant bits of ASID are implemented first:
that is, if ASIDLEN~$>$~0, ASID[ASIDLEN-1:0] is writable.  The maximal value
of ASIDLEN, termed ASIDMAX, is 9 for Sv32 or 16 for Sv39 and Sv48.


\begin{commentary}
For many applications, the choice of page size has a substantial
performance impact.  A large page size increases TLB reach and loosens
the associativity constraints on virtually-indexed, physically-tagged
caches.  At the same time, large pages exacerbate internal
fragmentation, wasting physical memory and possibly cache capacity.

After much deliberation, we have settled on a conventional page size
of 4 KiB for both RV32 and RV64.  We expect this decision to ease the
porting of low-level runtime software and device drivers.  The TLB
reach problem is ameliorated by transparent superpage support in
modern operating systems~\cite{transparent-superpages}.  Additionally,
multi-level TLB hierarchies are quite inexpensive relative to the
multi-level cache hierarchies whose address space they map.
\end{commentary}

Note that writing {\tt satp} does not imply any ordering constraints
between page-table updates and subsequent address translations.
If the new address space's page tables have been modified, it may be
necessary to execute an SFENCE.VMA instruction
(see Section~\ref{sec:sfence.vma}) prior to writing {\tt satp}.

\begin{commentary}
Not imposing upon implementations to flush address-translation caches
upon {\tt satp} writes reduces the cost of context switches, provided
a sufficiently large ASID space.
\end{commentary}

\section{Supervisor Instructions}

In addition to the SRET instruction defined in
Section~\ref{otherpriv}, one new supervisor-level instruction is
provided.

\subsection{Supervisor Memory-Management Fence Instruction}
\label{sec:sfence.vma}

\vspace{-0.2in}
\begin{center}
\begin{tabular}{O@{}R@{}R@{}F@{}R@{}S}
\\
\instbitrange{31}{25} &
\instbitrange{24}{20} &
\instbitrange{19}{15} &
\instbitrange{14}{12} &
\instbitrange{11}{7} &
\instbitrange{6}{0} \\
\hline
\multicolumn{1}{|c|}{funct7} &
\multicolumn{1}{c|}{rs2} &
\multicolumn{1}{c|}{rs1} &
\multicolumn{1}{c|}{funct3} &
\multicolumn{1}{c|}{rd} &
\multicolumn{1}{c|}{opcode} \\
\hline
7 & 5 & 5 & 3 & 5 & 7 \\
SFENCE.VMA & asid & vaddr & PRIV & 0 & SYSTEM \\
\end{tabular}
\end{center}

The supervisor memory-management fence instruction SFENCE.VMA is used to
synchronize updates to in-memory memory-management data structures with
current execution.  Instruction execution causes implicit reads and writes to
these data structures; however, these implicit references are ordinarily not
ordered with respect to loads and stores in the instruction stream.  Executing
an SFENCE.VMA instruction guarantees that any stores in the instruction stream
prior to the SFENCE.VMA are ordered before all implicit references subsequent
to the SFENCE.VMA.

\begin{commentary}
The SFENCE.VMA is used to flush any local hardware caches related to
address translation.  It is specified as a fence rather than a TLB
flush to provide cleaner semantics with respect to which instructions
are affected by the flush operation and to support a wider variety of
dynamic caching structures and memory-management schemes.  SFENCE.VMA
is also used by higher privilege levels to synchronize page table
writes and the address translation hardware.
\end{commentary}

\begin{commentary}
Note the instruction has no effect on the translations of other harts,
which must be notified separately.  One approach is to use 1)
a local data fence to ensure local writes are visible globally, then
2) an interprocessor interrupt to the other thread, then 3) a local
SFENCE.VMA in the interrupt handler of the remote thread, and finally
4) signal back to originating thread that operation is complete.  This
is, of course, the RISC-V analog to a TLB shootdown.  Alternatively,
implementations might provide direct hardware support for remote TLB
invalidation.  TLB shootdowns are handled by an SBI call to hide
implementation details.
\end{commentary}

For the common case that the translation data structures have only been
modified for a single address mapping (i.e., one page or superpage), {\em rs1}
can specify a virtual address within that mapping to effect a translation
fence for that mapping only.  Furthermore, for the common case that the
translation data structures have only been modified for a single address-space
identifier, {\em rs2} can specify the address space.  The behavior of
SFENCE.VMA depends on {\em rs1} and {\em rs2} as follows:

\begin{itemize}
\item If {\em rs1}={\tt x0} and {\em rs2}={\tt x0}, the fence orders all
      reads and writes made to any level of the page tables, for all
      address spaces.
\item If {\em rs1}={\tt x0} and {\em rs2}$\neq${\tt x0}, the fence orders
      all reads and writes made to any level of the page tables, but only
      for the address space identified by integer register {\em rs2}.
      Accesses to {\em global} mappings (see Section~\ref{sec:translation})
      are not ordered.
\item If {\em rs1}$\neq${\tt x0} and {\em rs2}={\tt x0}, the fence orders
      only reads and writes made to the leaf page table entry corresponding
      to the virtual address in {\em rs1}, for all address spaces.
\item If {\em rs1}$\neq${\tt x0} and {\em rs2}$\neq${\tt x0}, the fence
      orders only reads and writes made to the leaf page table entry
      corresponding to the virtual address in {\em rs1}, for the address
      space identified by integer register {\em rs2}.
      Accesses to global mappings are not ordered.
\end{itemize}

When {\em rs2}$\neq${\tt x0}, bits XLEN-1:ASIDMAX of the value held in {\em
rs2} are reserved for future use and should be zeroed by software and ignored
by current implementations.  Furthermore, if ASIDLEN~$<$~ASIDMAX, the
implementation shall ignore bits ASIDMAX-1:ASIDLEN of the value held in {\em
rs2}.

\begin{commentary}
Simpler implementations can ignore the virtual address in {\em rs1} and
the ASID value in {\em rs2} and always perform a global fence.
\end{commentary}

\section{Sv32: Page-Based 32-bit Virtual-Memory Systems}
\label{sec:sv32}

When Sv32 is written to the MODE field in the {\tt satp} register
(see Section~\ref{sec:satp}),
the supervisor operates in a 32-bit paged virtual-memory system.  Sv32
is supported on RV32 systems and is designed to include mechanisms
sufficient for supporting modern Unix-based operating systems.

\begin{commentary}
The initial RISC-V paged virtual-memory architectures have been
designed as straightforward implementations to support existing
operating systems.  We have architected page table layouts to support
a hardware page-table walker.  Software TLB refills are a performance
bottleneck on high-performance systems, and are especially troublesome
with decoupled specialized coprocessors.  An implementation can choose
to implement software TLB refills using a machine-mode trap handler as
an extension to M-mode.
\end{commentary}

\subsection{Addressing and Memory Protection}
\label{sec:translation}

Sv32 implementations support a 32-bit virtual address space, divided
into \wunits{4}{KiB} pages.  An Sv32 virtual address is partitioned
into a virtual page number (VPN) and page offset, as shown in
Figure~\ref{sv32va}.  When Sv32 virtual memory mode is selected in the
MODE field of the {\tt satp} register, supervisor virtual addresses
are translated into supervisor physical addresses via a two-level page
table.  The 20-bit VPN is translated into a 22-bit physical page
number (PPN), while the 12-bit page offset is untranslated.  The
resulting supervisor-level physical addresses are then checked using
any physical memory protection structures (Sections~\ref{sec:pmp}),
before being directly converted to machine-level physical addresses.

\begin{figure*}[h!]
{\footnotesize
\begin{center}
\begin{tabular}{@{}O@{}O@{}E}
\instbitrange{31}{22} &
\instbitrange{21}{12} &
\instbitrange{11}{0} \\
\hline
\multicolumn{1}{|c|}{VPN[1]} &
\multicolumn{1}{c|}{VPN[0]} &
\multicolumn{1}{c|}{page offset} \\
\hline
10 & 10 & 12 \\
\end{tabular}
\end{center}
}
\vspace{-0.1in}
\caption{Sv32 virtual address.}
\label{sv32va}
\end{figure*}

\begin{figure*}[h!]
{\footnotesize
\begin{center}
\begin{tabular}{@{}E@{}O@{}E}
\instbitrange{33}{22} &
\instbitrange{21}{12} &
\instbitrange{11}{0} \\
\hline
\multicolumn{1}{|c|}{PPN[1]} &
\multicolumn{1}{c|}{PPN[0]} &
\multicolumn{1}{c|}{page offset} \\
\hline
12 & 10 & 12 \\
\end{tabular}
\end{center}
}
\vspace{-0.1in}
\caption{Sv32 physical address.}
\label{rv32va}
\end{figure*}

\begin{figure*}[h!]
{\footnotesize
\begin{center}
\begin{tabular}{@{}E@{}O@{}Fcccccccc}
\instbitrange{31}{20} &
\instbitrange{19}{10} &
\instbitrange{9}{8} &
\instbit{7} &
\instbit{6} &
\instbit{5} &
\instbit{4} &
\instbit{3} &
\instbit{2} &
\instbit{1} &
\instbit{0} \\
\hline
\multicolumn{1}{|c|}{PPN[1]} &
\multicolumn{1}{c|}{PPN[0]} &
\multicolumn{1}{c|}{RSW} &
\multicolumn{1}{c|}{D} &
\multicolumn{1}{c|}{A} &
\multicolumn{1}{c|}{G} &
\multicolumn{1}{c|}{U} &
\multicolumn{1}{c|}{X} &
\multicolumn{1}{c|}{W} &
\multicolumn{1}{c|}{R} &
\multicolumn{1}{c|}{V} \\
\hline
12 & 10 & 2 & 1 & 1 & 1 & 1 & 1 & 1 & 1 & 1\\
\end{tabular}
\end{center}
}
\vspace{-0.1in}
\caption{Sv32 page table entry.}
\label{sv32pte}
\end{figure*}

Sv32 page tables consist of $2^{10}$ page-table entries (PTEs), each
of four bytes.  A page table is exactly the size of a page and must
always be aligned to a page boundary.  The physical page number of the
root page table is stored in the {\tt satp} register.

The PTE format for Sv32 is shown in Figures~\ref{sv32pte}.  The V bit
indicates whether the PTE is valid; if it is 0, bits 31--1 of the PTE are
don't-cares and may be used freely by software.  The permission bits, R, W,
and X, indicate whether the page is readable, writable, and executable,
respectively.  When all three are zero, the PTE is a pointer to the next level
of the page table; otherwise, it is a leaf PTE.  Writable pages must also be
marked readable; the contrary combinations are reserved for future use.
Table~\ref{pteperm} summarizes the encoding of the permission bits.

\begin{table*}[h!]
\begin{center}
\begin{tabular}{|c|c|c||l|}
\hline
X & W & R & Meaning \\
\hline
0 & 0 & 0 & Pointer to next level of page table. \\
0 & 0 & 1 & Read-only page. \\
0 & 1 & 0 & {\em Reserved for future use.} \\
0 & 1 & 1 & Read-write page. \\
1 & 0 & 0 & Execute-only page. \\
1 & 0 & 1 & Read-execute page. \\
1 & 1 & 0 & {\em Reserved for future use.} \\
1 & 1 & 1 & Read-write-execute page. \\
\hline
\end{tabular}
\end{center}
\caption{Encoding of PTE R/W/X fields.}
\label{pteperm}
\end{table*}

The U bit indicates whether the page is accessible to user mode.
U-mode software may only access the page when U=1.  If the SUM bit
in the {\tt sstatus} register is
set, supervisor mode software may also access pages with U=1.
However, supervisor code normally operates with the SUM bit clear, in
which case, supervisor code will fault on accesses to user-mode pages.

\begin{commentary}
An alternative PTE format would support different permissions for supervisor
and user.  We omitted this feature because it would be largely redundant with
the SUM mechanism (see Section~\ref{sec:sum}) and would require more encoding
space in the PTE.
\end{commentary}

The G bit designates a {\em global} mapping.  Global mappings are those that
exist in all address spaces.  For non-leaf PTEs, the global setting implies
that all mappings in the subsequent levels of the page table are global.  Note
that failing to mark a global mapping as global merely reduces performance,
whereas marking a non-global mapping as global is an error.

\begin{commentary}
Global mappings need not be stored redundantly in address-translation caches
for multiple ASIDs.  Additionally, they need not be flushed from local
address-translation caches when an SFENCE.VMA instruction is executed with
{\em rs2}$\neq${\tt x0}.
\end{commentary}

The RSW field is reserved for use by supervisor software; the implementation
shall ignore this field.

Each leaf PTE contains an accessed (A) and dirty (D) bit.  The A bit indicates
the virtual page has been read, written, or fetched from since the last time
the A bit was cleared.  The D bit indicates the virtual page has been written
since the last time the D bit was cleared.

Two schemes to manage the A and D bits are permitted:
\begin{itemize}
\item When a virtual page is accessed and the A bit is clear, or is
      written and the D bit is clear, the implementation sets the
      corresponding bit in the PTE.  The PTE update must be atomic with
      respect to other accesses to the PTE, and must atomically check
      that the PTE is valid and grants sufficient permissions.  The
      PTE update must be exact (i.e., not speculative), and observed
      in program order by the local hart. The ordering on loads and stores
      provided by FENCE instructions and the acquire/release bits on atomic
      instructions also orders the PTE updates associated with those loads
      and stores as observed by remote harts.
\item When a virtual page is accessed and the A bit is clear, or is
      written and the D bit is clear, a page-fault exception is raised.
\end{itemize}
Standard supervisor software should be written to assume either or both
PTE update schemes may be in effect.

\begin{commentary}
Mandating that the PTE updates to be exact, atomic, and in program order
simplifies the specification, and makes the feature more useful for system
software.  Simple implementations may instead generate page-fault exceptions.
  
The A and D bits are never cleared by the implementation.  If the
supervisor software does not rely on accessed and/or dirty bits,
e.g. if it does not swap memory pages to secondary storage or if the
pages are being used to map I/O space, it should always set them to 1
in the PTE to improve performance.
\end{commentary}

Any level of PTE may be a leaf PTE, so in addition to 4 KiB pages, Sv32
supports 4 MiB {\em megapages}.  A megapage must be virtually and
physically aligned to a 4 MiB boundary; a page-fault exception is raised
if the physical address is insufficiently aligned.

For non-leaf PTEs, the D, A, and U bits are reserved for future use and
must be cleared by software for forward compatibility.

\subsection{Virtual Address Translation Process}
\label{sv32algorithm}

A virtual address $va$ is translated into a physical address $pa$ as
follows:

\begin{enumerate}

\item Let $a$ be ${\tt satp}.ppn \times \textrm{PAGESIZE}$, and let $i=\textrm{LEVELS} - 1$. (For Sv32, PAGESIZE=$2^{12}$ and LEVELS=2.)

\item Let $pte$ be the value of the PTE at address
  $a+va.vpn[i]\times \textrm{PTESIZE}$. (For Sv32, PTESIZE=4.)
  If accessing $pte$ violates a PMA or PMP check, raise an
  access exception corresponding to the original access type.

\item If $pte.v=0$, or if $pte.r=0$ and $pte.w=1$, stop and raise a page-fault exception.

\item Otherwise, the PTE is valid.
  If $pte.r=1$ or $pte.x=1$, go to step 5.
  Otherwise, this PTE is a pointer to the next level of the page table.  Let
  $i=i-1$.  If $i<0$, stop and raise a page-fault exception.  Otherwise, let
  $a=pte.ppn \times \textrm{PAGESIZE}$ and go to step 2.

\item A leaf PTE has been found.  Determine if the requested memory access is
  allowed by the $pte.r$, $pte.w$, $pte.x$, and $pte.u$ bits, given the
  current privilege mode and the value of the SUM and MXR fields of
  the {\tt mstatus} register.  If not, stop and raise a page-fault exception.

\item If $i>0$ and $pa.ppn[i-1:0]\neq 0$, this is a misaligned superpage;
      stop and raise a page-fault exception.

\item If $pte.a=0$, or if the memory access is a store and $pte.d=0$, either
  raise a page-fault exception or:
  \begin{itemize}
  \item Set $pte.a$ to 1 and, if the memory access is a store, also set
    $pte.d$ to 1.
  \item If this access violates a PMA or PMP check, raise an access exception
    corresponding to the original access type.
  \item This update and the loading of $pte$ in step 2 must be atomic; in
    particular, no intervening store to the PTE may be perceived to have
    occurred in-between.
  \end{itemize}

\item The translation is successful. The translated physical address is
  given as follows:
\begin{itemize}
\item $\textit{pa.pgoff} = \textit{va.pgoff}$.
\item If $i>0$, then this is a superpage translation and $pa.ppn[i-1:0]=va.vpn[i-1:0]$.
\item $pa.ppn[\textrm{LEVELS} - 1:i] = pte.ppn[\textrm{LEVELS} - 1:i]$.
\end{itemize}

\end{enumerate}

\section{Sv39: Page-Based 39-bit Virtual-Memory System}
\label{sec:sv39}

This section describes a simple paged virtual-memory system designed
for RV64 systems, which supports 39-bit virtual address spaces.  The
design of Sv39 follows the overall scheme of Sv32, and this section
details only the differences between the schemes.

\begin{commentary}
We specified multiple virtual memory systems for RV64 to relieve the tension
between providing a large address space and minimizing address-translation
cost.  For many systems, \wunits{512}{GiB} of virtual-address space is ample,
and so Sv39 suffices.  Sv48 increases the virtual address space to
\wunits{256}{TiB}, but increases the physical memory
capacity dedicated to page tables, the latency of page-table traversals, and
the size of hardware structures that store virtual addresses.
\end{commentary}

\subsection{Addressing and Memory Protection}

Sv39 implementations support a 39-bit virtual address space, divided
into \wunits{4}{KiB} pages.  An Sv39 address is partitioned as
shown in Figure~\ref{sv39va}.  Load and store effective addresses,
which are 64 bits, must have bits 63--39 all equal to bit 38, or else
a page-fault exception will occur.  The 27-bit VPN is translated into a
44-bit PPN via a three-level page table, while the 12-bit page offset
is untranslated.

\begin{figure*}[h!]
{\footnotesize
\begin{center}
\begin{tabular}{@{}O@{}O@{}O@{}O}
\instbitrange{38}{30} &
\instbitrange{29}{21} &
\instbitrange{20}{12} &
\instbitrange{11}{0} \\
\hline
\multicolumn{1}{|c|}{VPN[2]} &
\multicolumn{1}{c|}{VPN[1]} &
\multicolumn{1}{c|}{VPN[0]} &
\multicolumn{1}{c|}{page offset} \\
\hline
9 & 9 & 9 & 12 \\
\end{tabular}
\end{center}
}
\vspace{-0.1in}
\caption{Sv39 virtual address.}
\label{sv39va}
\end{figure*}

\begin{figure*}[h!]
{\footnotesize
\begin{center}
\begin{tabular}{@{}T@{}O@{}O@{}O}
\instbitrange{55}{30} &
\instbitrange{29}{21} &
\instbitrange{20}{12} &
\instbitrange{11}{0} \\
\hline
\multicolumn{1}{|c|}{PPN[2]} &
\multicolumn{1}{c|}{PPN[1]} &
\multicolumn{1}{c|}{PPN[0]} &
\multicolumn{1}{c|}{page offset} \\
\hline
26 & 9 & 9 & 12 \\
\end{tabular}
\end{center}
}
\vspace{-0.1in}
\caption{Sv39 physical address.}
\label{sv39pa}
\end{figure*}

\begin{figure*}[h!]
{\footnotesize
\begin{center}
\begin{tabular}{@{}Y@{}Y@{}Y@{}Y@{}Fcccccccc}
\instbitrange{63}{54} &
\instbitrange{53}{28} &
\instbitrange{27}{19} &
\instbitrange{18}{10} &
\instbitrange{9}{8} &
\instbit{7} &
\instbit{6} &
\instbit{5} &
\instbit{4} &
\instbit{3} &
\instbit{2} &
\instbit{1} &
\instbit{0} \\
\hline
\multicolumn{1}{|c|}{\it Reserved} &
\multicolumn{1}{c|}{PPN[2]} &
\multicolumn{1}{c|}{PPN[1]} &
\multicolumn{1}{c|}{PPN[0]} &
\multicolumn{1}{c|}{RSW} &
\multicolumn{1}{c|}{D} &
\multicolumn{1}{c|}{A} &
\multicolumn{1}{c|}{G} &
\multicolumn{1}{c|}{U} &
\multicolumn{1}{c|}{X} &
\multicolumn{1}{c|}{W} &
\multicolumn{1}{c|}{R} &
\multicolumn{1}{c|}{V} \\
\hline
10 & 26 & 9 & 9 & 2 & 1 & 1 & 1 & 1 & 1 & 1 & 1 & 1\\
\end{tabular}
\end{center}
}
\vspace{-0.1in}
\caption{Sv39 page table entry.}
\label{sv39pte}
\end{figure*}

Sv39 page tables contain $2^9$ page table entries (PTEs), eight
bytes each.  A page table is exactly the size of a page and must
always be aligned to a page boundary.  The physical page number of the
root page table is stored in the {\tt satp} register's PPN field.

The PTE format for Sv39 is shown in Figure~\ref{sv39pte}.  Bits 9--0
have the same meaning as for Sv32.  Bits 63--54 are reserved
for future use and must be zeroed by software for forward compatibility.

\begin{commentary}
We reserved several PTE bits for a possible extension that improves
support for sparse address spaces by allowing page-table levels to be
skipped, reducing memory usage and TLB refill latency.  These reserved
bits may also be used to facilitate research experimentation.  The
cost is reducing the physical address space, but \wunits{64}{PiB} is
presently ample.  When it no longer suffices, the reserved
bits that remain unallocated could be used to expand the physical
address space.
\end{commentary}

Any level of PTE may be a leaf PTE, so in addition to \wunits{4}{KiB}
pages, Sv39 supports \wunits{2}{MiB} {\em megapages} and
\wunits{1}{GiB} {\em gigapages}, each of which must be virtually and
physically aligned to a boundary equal to its size.
A page-fault exception is raised if the physical address is insufficiently
aligned.

The algorithm for virtual-to-physical address translation is the same as in
Section~\ref{sv32algorithm}, except LEVELS equals 3 and PTESIZE equals 8.

\section{Sv48: Page-Based 48-bit Virtual-Memory System}
\label{sec:sv48}

This section describes a simple paged virtual-memory system designed
for RV64 systems, which supports 48-bit virtual address spaces.  Sv48
is intended for systems for which a 39-bit virtual address space is
insufficient.  It closely follows the design of Sv39, simply adding an
additional level of page table, and so this chapter only details the
differences between the two schemes.

Implementations that support Sv48 must also support Sv39.

\begin{commentary}
Systems that support Sv48 can also support Sv39 at essentially no cost, and so
should do so to maintain compatibility with supervisor software that assumes
Sv39.
\end{commentary}

\subsection{Addressing and Memory Protection}

Sv48 implementations support a 48-bit virtual address space, divided
into \wunits{4}{KiB} pages.  An Sv48 address is partitioned as
shown in Figure~\ref{sv48va}.  Load and store effective addresses,
which are 64 bits, must have bits 63--48 all equal to bit 47, or else
a page-fault exception will occur.  The 36-bit VPN is translated into a
44-bit PPN via a four-level page table, while the 12-bit page offset
is untranslated.

\begin{figure*}[h!]
{\footnotesize
\begin{center}
\begin{tabular}{@{}O@{}O@{}O@{}O@{}O}
\instbitrange{47}{39} &
\instbitrange{38}{30} &
\instbitrange{29}{21} &
\instbitrange{20}{12} &
\instbitrange{11}{0} \\
\hline
\multicolumn{1}{|c|}{VPN[3]} &
\multicolumn{1}{c|}{VPN[2]} &
\multicolumn{1}{c|}{VPN[1]} &
\multicolumn{1}{c|}{VPN[0]} &
\multicolumn{1}{c|}{page offset} \\
\hline
9 & 9 & 9 & 9 & 12 \\
\end{tabular}
\end{center}
}
\vspace{-0.1in}
\caption{Sv48 virtual address.}
\label{sv48va}
\end{figure*}

\begin{figure*}[h!]
{\footnotesize
\begin{center}
\begin{tabular}{@{}E@{}O@{}O@{}O@{}O}
\instbitrange{55}{39} &
\instbitrange{38}{30} &
\instbitrange{29}{21} &
\instbitrange{20}{12} &
\instbitrange{11}{0} \\
\hline
\multicolumn{1}{|c|}{PPN[3]} &
\multicolumn{1}{c|}{PPN[2]} &
\multicolumn{1}{c|}{PPN[1]} &
\multicolumn{1}{c|}{PPN[0]} &
\multicolumn{1}{c|}{page offset} \\
\hline
17 & 9 & 9 & 9 & 12 \\
\end{tabular}
\end{center}
}
\vspace{-0.1in}
\caption{Sv48 physical address.}
\label{sv48pa}
\end{figure*}

\begin{figure*}[h!]
{\footnotesize
\begin{center}
\begin{tabular}{@{}Y@{}Y@{}Y@{}Y@{}Y@{}Fcccccccc}
\instbitrange{63}{54} &
\instbitrange{53}{37} &
\instbitrange{36}{28} &
\instbitrange{27}{19} &
\instbitrange{18}{10} &
\instbitrange{9}{8} &
\instbit{7} &
\instbit{6} &
\instbit{5} &
\instbit{4} &
\instbit{3} &
\instbit{2} &
\instbit{1} &
\instbit{0} \\
\hline
\multicolumn{1}{|c|}{\it Reserved} &
\multicolumn{1}{c|}{PPN[3]} &
\multicolumn{1}{c|}{PPN[2]} &
\multicolumn{1}{c|}{PPN[1]} &
\multicolumn{1}{c|}{PPN[0]} &
\multicolumn{1}{c|}{RSW} &
\multicolumn{1}{c|}{D} &
\multicolumn{1}{c|}{A} &
\multicolumn{1}{c|}{G} &
\multicolumn{1}{c|}{U} &
\multicolumn{1}{c|}{X} &
\multicolumn{1}{c|}{W} &
\multicolumn{1}{c|}{R} &
\multicolumn{1}{c|}{V} \\
\hline
10 & 17 & 9 & 9 & 9 & 2 & 1 & 1 & 1 & 1 & 1 & 1 & 1 & 1\\
\end{tabular}
\end{center}
}
\vspace{-0.1in}
\caption{Sv48 page table entry.}
\label{sv48pte}
\end{figure*}

The PTE format for Sv48 is shown in Figure~\ref{sv48pte}.  Bits 9--0
have the same meaning as for Sv32.  Any level of PTE may be a leaf
PTE, so in addition to \wunits{4}{KiB} pages, Sv48 supports
\wunits{2}{MiB} {\em megapages}, \wunits{1}{GiB} {\em gigapages}, and
\wunits{512}{GiB} {\em terapages}, each of which must be virtually and
physically aligned to a boundary equal to its size.
A page-fault exception is raised if the physical address is insufficiently
aligned.

The algorithm for virtual-to-physical address translation is the same
as in Section~\ref{sv32algorithm}, except LEVELS equals 4 and PTESIZE
equals 8.
