\chapter{``D'' Standard Extension for Double-Precision Floating-Point,
Version 2.0}

This chapter describes the standard double-precision floating-point
instruction-set extension, which is named ``D'' and adds
double-precision floating-point computational instructions compliant
with the IEEE 754-2008 arithmetic standard.  The D extension depends on
the base single-precision instruction subset F.

\section{D Register State}

The D extension widens the 32 floating-point registers, {\tt f0}--{\tt
  f31}, to 64 bits (FLEN=64 in Figure~\ref{fprs}).  The {\tt f}
registers can now hold either 32-bit or 64-bit floating-point values
as described below in Section~\ref{nanboxing}.

\begin{commentary}
FLEN can be 32, 64, or 128 depending on which of the F, D, and Q
extensions are supported.  There can be up to four different
floating-point precisions supported, including H, F, D, and Q.
\end{commentary}

\section{NaN Boxing of Narrower Values}
\label{nanboxing}

When multiple floating-point precisions are supported, then valid
values of narrower $n$-bit types, \mbox{$n<$ FLEN}, are represented in
the lower $n$ bits of an FLEN-bit NaN value, in a process termed
NaN-boxing.  The upper bits of a valid NaN-boxed value must be all 1s.
Valid NaN-boxed $n$-bit values therefore appear as negative quiet NaNs
(qNaNs) when viewed as any wider $m$-bit value, \mbox{$n < m \leq$
  FLEN}.  Any operation that writes a narrower result to an {\tt f}
register must write all 1s to the uppermost FLEN$-n$ bits to yield a
legal NaN-boxed value.

\begin{samepage-commentary}
Software might not know the current type of data stored in a
floating-point register but has to be able to save and restore the
register values, hence the result of using wider operations to
transfer narrower values has to be defined.  A common case is for
callee-saved registers, but a standard convention is also desirable for
features including varargs, user-level threading libraries, virtual
machine migration, and debugging.
\end{samepage-commentary}

Floating-point $n$-bit transfer operations move external values held
in IEEE standard formats into and out of the {\tt f} registers, and
comprise floating-point loads and stores (FL$n$/FS$n$) and
floating-point move instructions (FMV.$n$.X/FMV.X.$n$).  A narrower
$n$-bit transfer, \mbox{$n<$ FLEN}, into the {\tt f} registers will create a
valid NaN-boxed value.  A narrower $n$-bit transfer out of
the floating-point registers will transfer the lower $n$ bits of the
register ignoring the upper FLEN$-n$ bits.

Apart from transfer operations described in the previous paragraph,
all other floating-point operations on narrower $n$-bit operations,
\mbox{$n<$ FLEN}, check if the input operands are correctly NaN-boxed,
i.e., all upper FLEN$-n$ bits are 1.  If so, the $n$ least-significant
bits of the input are used as the input value, otherwise the input
value is treated as an $n$-bit canonical NaN.

\begin{commentary}
Earlier versions of this document did not define the behavior of
feeding the results of narrower or wider operands into an operation,
except to require that wider saves and restores would preserve the
value of a narrower operand.  The new definition removes this
implementation-specific behavior, while still accommodating both
non-recoded and recoded implementations of the floating-point unit.
The new definition also helps catch software errors by propagating
NaNs if values are used incorrectly.

Non-recoded implementations unpack and pack the operands to IEEE
standard format on the input and output of every floating-point
operation.  The NaN-boxing cost to a non-recoded implementation is
primarily in checking if the upper bits of a narrower operation
represent a legal NaN-boxed value, and in writing all 1s to the upper
bits of a result.

Recoded implementations use a more convenient internal format to
represent floating-point values, with an added exponent bit to allow
all values to be held normalized.  The cost to the recoded
implementation is primarily the extra tagging needed to track the
internal types and sign bits, but this can be done without adding new
state bits by recoding NaNs internally in the exponent field.  Small
modifications are needed to the pipelines used to transfer values in
and out of the recoded format, but the datapath and latency costs are
minimal.  The recoding process has to handle shifting of input
subnormal values for wide operands in any case, and extracting the
NaN-boxed value is a similar process to normalization except for
skipping over leading-1 bits instead of skipping over leading-0 bits,
allowing the datapath muxing to be shared.
\end{commentary}

\section{Double-Precision Load and Store Instructions}
\label{fld_fsd}

The FLD instruction loads a double-precision floating-point value from
memory into floating-point register {\em rd}.  FSD stores a double-precision
value from the floating-point registers to memory.
\begin{commentary}
The double-precision value may be a NaN-boxed single-precision value.
\end{commentary}

\vspace{-0.2in}
\begin{center}
\begin{tabular}{M@{}R@{}F@{}R@{}O}
\\
\instbitrange{31}{20} &
\instbitrange{19}{15} &
\instbitrange{14}{12} &
\instbitrange{11}{7} &
\instbitrange{6}{0} \\
\hline
\multicolumn{1}{|c|}{imm[11:0]} &
\multicolumn{1}{c|}{rs1} &
\multicolumn{1}{c|}{width} &
\multicolumn{1}{c|}{rd} &
\multicolumn{1}{c|}{opcode} \\
\hline
12 & 5 & 3 & 5 & 7 \\
offset[11:0] & base & D & dest & LOAD-FP \\
\end{tabular}
\end{center}

\vspace{-0.2in}
\begin{center}
\begin{tabular}{O@{}R@{}R@{}F@{}R@{}O}
\\
\instbitrange{31}{25} &
\instbitrange{24}{20} &
\instbitrange{19}{15} &
\instbitrange{14}{12} &
\instbitrange{11}{7} &
\instbitrange{6}{0} \\
\hline
\multicolumn{1}{|c|}{imm[11:5]} &
\multicolumn{1}{c|}{rs2} &
\multicolumn{1}{c|}{rs1} &
\multicolumn{1}{c|}{width} &
\multicolumn{1}{c|}{imm[4:0]} &
\multicolumn{1}{c|}{opcode} \\
\hline
7 & 5 & 5 & 3 & 5 & 7 \\
offset[11:5] & src & base & D & offset[4:0] & STORE-FP \\
\end{tabular}
\end{center}

FLD and FSD are only guaranteed to execute atomically if the effective address
is naturally aligned and XLEN$\geq$64.

\section{Double-Precision Floating-Point Computational Instructions}

The double-precision floating-point computational instructions are
defined analogously to their single-precision counterparts, but operate on
double-precision operands and produce double-precision results.
\vspace{-0.2in}
\begin{center}
\begin{tabular}{R@{}F@{}R@{}R@{}F@{}R@{}O}
\\
\instbitrange{31}{27} &
\instbitrange{26}{25} &
\instbitrange{24}{20} &
\instbitrange{19}{15} &
\instbitrange{14}{12} &
\instbitrange{11}{7} &
\instbitrange{6}{0} \\
\hline
\multicolumn{1}{|c|}{funct5} &
\multicolumn{1}{c|}{fmt} &
\multicolumn{1}{c|}{rs2} &
\multicolumn{1}{c|}{rs1} &
\multicolumn{1}{c|}{rm} &
\multicolumn{1}{c|}{rd} &
\multicolumn{1}{c|}{opcode} \\
\hline
5 & 2 & 5 & 5 & 3 & 5 & 7 \\
FADD/FSUB & D & src2 & src1 & RM  & dest & OP-FP  \\
FMUL/FDIV & D & src2 & src1 & RM  & dest & OP-FP  \\
FMIN-MAX  & D & src2 & src1 & MIN/MAX & dest & OP-FP  \\
FSQRT     & D & 0    & src  & RM  & dest & OP-FP  \\
\end{tabular}
\end{center}

\vspace{-0.2in}
\begin{center}
\begin{tabular}{R@{}F@{}R@{}R@{}F@{}R@{}O}
\\
\instbitrange{31}{27} &
\instbitrange{26}{25} &
\instbitrange{24}{20} &
\instbitrange{19}{15} &
\instbitrange{14}{12} &
\instbitrange{11}{7} &
\instbitrange{6}{0} \\
\hline
\multicolumn{1}{|c|}{rs3} &
\multicolumn{1}{c|}{fmt} &
\multicolumn{1}{c|}{rs2} &
\multicolumn{1}{c|}{rs1} &
\multicolumn{1}{c|}{rm} &
\multicolumn{1}{c|}{rd} &
\multicolumn{1}{c|}{opcode} \\
\hline
5 & 2 & 5 & 5 & 3 & 5 & 7 \\
src3 & D & src2 & src1 & RM  & dest & F[N]MADD/F[N]MSUB  \\
\end{tabular}
\end{center}

\section{Double-Precision Floating-Point Conversion and Move Instructions}

Floating-point-to-integer and integer-to-floating-point conversion
instructions are encoded in the OP-FP major opcode space.
FCVT.W.D or FCVT.L.D converts a double-precision floating-point number
in floating-point register {\em rs1} to a signed 32-bit or 64-bit
integer, respectively, in integer register {\em rd}.  FCVT.D.W
or FCVT.D.L converts a 32-bit or 64-bit signed integer,
respectively, in integer register {\em rs1} into a
double-precision floating-point
number in floating-point register {\em rd}. FCVT.WU.D,
FCVT.LU.D, FCVT.D.WU, and FCVT.D.LU variants
convert to or from unsigned integer values.
For RV64, FCVT.W[U].D sign-extends the 32-bit result.
FCVT.L[U].D and FCVT.D.L[U] are RV64-only instructions.
The range of valid inputs for FCVT.{\em int}.D and
the behavior for invalid inputs are the same as for FCVT.{\em int}.S.

All floating-point to integer and integer to floating-point conversion
instructions round according to the {\em rm} field.  Note FCVT.D.W[U] always
produces an exact result and is unaffected by rounding mode.

\vspace{-0.2in}
\begin{center}
\begin{tabular}{R@{}F@{}R@{}R@{}F@{}R@{}O}
\\
\instbitrange{31}{27} &
\instbitrange{26}{25} &
\instbitrange{24}{20} &
\instbitrange{19}{15} &
\instbitrange{14}{12} &
\instbitrange{11}{7} &
\instbitrange{6}{0} \\
\hline
\multicolumn{1}{|c|}{funct5} &
\multicolumn{1}{c|}{fmt} &
\multicolumn{1}{c|}{rs2} &
\multicolumn{1}{c|}{rs1} &
\multicolumn{1}{c|}{rm} &
\multicolumn{1}{c|}{rd} &
\multicolumn{1}{c|}{opcode} \\
\hline
5 & 2 & 5 & 5 & 3 & 5 & 7 \\
FCVT.{\em int}.D & D & W[U]/L[U] & src & RM  & dest & OP-FP  \\
FCVT.D.{\em int} & D & W[U]/L[U] & src & RM  & dest & OP-FP  \\
\end{tabular}
\end{center}

The double-precision to single-precision and single-precision to
double-precision conversion instructions, FCVT.S.D and FCVT.D.S, are
encoded in the OP-FP major opcode space and both the source and
destination are floating-point registers.  The {\em rs2} field
encodes the datatype of the source, and the {\em fmt} field encodes
the datatype of the destination.  FCVT.S.D rounds according to the
RM field; FCVT.D.S will never round.

\vspace{-0.2in}
\begin{center}
\begin{tabular}{R@{}F@{}R@{}R@{}F@{}R@{}O}
\\
\instbitrange{31}{27} &
\instbitrange{26}{25} &
\instbitrange{24}{20} &
\instbitrange{19}{15} &
\instbitrange{14}{12} &
\instbitrange{11}{7} &
\instbitrange{6}{0} \\
\hline
\multicolumn{1}{|c|}{funct5} &
\multicolumn{1}{c|}{fmt} &
\multicolumn{1}{c|}{rs2} &
\multicolumn{1}{c|}{rs1} &
\multicolumn{1}{c|}{rm} &
\multicolumn{1}{c|}{rd} &
\multicolumn{1}{c|}{opcode} \\
\hline
5 & 2 & 5 & 5 & 3 & 5 & 7 \\
FCVT.S.D & S & D & src & RM  & dest & OP-FP  \\
FCVT.D.S & D & S & src & RM  & dest & OP-FP  \\
\end{tabular}
\end{center}

Floating-point to floating-point sign-injection instructions, FSGNJ.D,
FSGNJN.D, and FSGNJX.D are defined analogously to the single-precision
sign-injection instruction.

\vspace{-0.2in}
\begin{center}
\begin{tabular}{R@{}F@{}R@{}R@{}F@{}R@{}O}
\\
\instbitrange{31}{27} &
\instbitrange{26}{25} &
\instbitrange{24}{20} &
\instbitrange{19}{15} &
\instbitrange{14}{12} &
\instbitrange{11}{7} &
\instbitrange{6}{0} \\
\hline
\multicolumn{1}{|c|}{funct5} &
\multicolumn{1}{c|}{fmt} &
\multicolumn{1}{c|}{rs2} &
\multicolumn{1}{c|}{rs1} &
\multicolumn{1}{c|}{rm} &
\multicolumn{1}{c|}{rd} &
\multicolumn{1}{c|}{opcode} \\
\hline
5 & 2 & 5 & 5 & 3 & 5 & 7 \\
FSGNJ & D & src2 & src1 & J[N]/JX & dest & OP-FP  \\
\end{tabular}
\end{center}

For XLEN$\geq$64 only, instructions are provided to move bit patterns
between the floating-point and integer registers.  FMV.X.D moves the
double-precision value in floating-point register {\em rs1} to a
representation in IEEE 754-2008 standard encoding in integer register
{\em rd}.  FMV.D.X moves the double-precision value encoded in IEEE
754-2008 standard encoding from the integer register {\em rs1} to the
floating-point register {\em rd}.

\vspace{-0.2in}
\begin{center}
\begin{tabular}{R@{}F@{}R@{}R@{}F@{}R@{}O}
\\
\instbitrange{31}{27} &
\instbitrange{26}{25} &
\instbitrange{24}{20} &
\instbitrange{19}{15} &
\instbitrange{14}{12} &
\instbitrange{11}{7} &
\instbitrange{6}{0} \\
\hline
\multicolumn{1}{|c|}{funct5} &
\multicolumn{1}{c|}{fmt} &
\multicolumn{1}{c|}{rs2} &
\multicolumn{1}{c|}{rs1} &
\multicolumn{1}{c|}{rm} &
\multicolumn{1}{c|}{rd} &
\multicolumn{1}{c|}{opcode} \\
\hline
5 & 2 & 5 & 5 & 3 & 5 & 7 \\
FMV.X.D & D & 0    & src  & 000  & dest & OP-FP  \\
FMV.D.X & D & 0    & src  & 000  & dest & OP-FP  \\
\end{tabular}
\end{center}

\begin{commentary}
  Early versions of the RISC-V ISA had additional instructions to
  allow RV32 systems to transfer between the upper and lower portions
  of a 64-bit floating-point register and an integer register.
  However, these would be the only instructions with partial register
  writes and would add complexity in implementations with recoded
  floating-point or register renaming, requiring a pipeline read-modify-write
  sequence.  Scaling up to handling quad-precision for RV32 and RV64
  would also require additional instructions if they were to follow
  this pattern.  The ISA was defined to reduce the number of explicit
  int-float register moves, by having conversions and comparisons
  write results to the appropriate register file, so we expect the
  benefit of these instructions to be lower than for other ISAs.

  We note that for systems that implement a 64-bit floating-point unit
  including fused multiply-add support and 64-bit floating-point loads
  and stores, the marginal hardware cost of moving from a 32-bit to
  a 64-bit integer datapath is low, and a software ABI supporting 32-bit
  wide address-space and pointers can be used to avoid growth of
  static data and dynamic memory traffic.
\end{commentary}

\section{Double-Precision Floating-Point Compare Instructions}

The double-precision floating-point compare instructions are
defined analogously to their single-precision counterparts, but operate on
double-precision operands.

\vspace{-0.2in}
\begin{center}
\begin{tabular}{S@{}F@{}R@{}R@{}F@{}R@{}O}
\\
\instbitrange{31}{27} &
\instbitrange{26}{25} &
\instbitrange{24}{20} &
\instbitrange{19}{15} &
\instbitrange{14}{12} &
\instbitrange{11}{7} &
\instbitrange{6}{0} \\
\hline
\multicolumn{1}{|c|}{funct5} &
\multicolumn{1}{c|}{fmt} &
\multicolumn{1}{c|}{rs2} &
\multicolumn{1}{c|}{rs1} &
\multicolumn{1}{c|}{rm} &
\multicolumn{1}{c|}{rd} &
\multicolumn{1}{c|}{opcode} \\
\hline
5 & 2 & 5 & 5 & 3 & 5 & 7 \\
FCMP & D & src2 & src1 & EQ/LT/LE & dest & OP-FP  \\
\end{tabular}
\end{center}

\section{Double-Precision Floating-Point Classify Instruction}

The double-precision floating-point classify instruction, FCLASS.D, is
defined analogously to its single-precision counterpart, but operates on
double-precision operands.

\vspace{-0.2in}
\begin{center}
\begin{tabular}{S@{}F@{}R@{}R@{}F@{}R@{}O}
\\
\instbitrange{31}{27} &
\instbitrange{26}{25} &
\instbitrange{24}{20} &
\instbitrange{19}{15} &
\instbitrange{14}{12} &
\instbitrange{11}{7} &
\instbitrange{6}{0} \\
\hline
\multicolumn{1}{|c|}{funct5} &
\multicolumn{1}{c|}{fmt} &
\multicolumn{1}{c|}{rs2} &
\multicolumn{1}{c|}{rs1} &
\multicolumn{1}{c|}{rm} &
\multicolumn{1}{c|}{rd} &
\multicolumn{1}{c|}{opcode} \\
\hline
5 & 2 & 5 & 5 & 3 & 5 & 7 \\
FCLASS & D & 0 & src & 001 & dest & OP-FP  \\
\end{tabular}
\end{center}
