\chapter{RV32I Base Integer Instruction Set, Version 2.0}
\label{rv32}

This chapter describes version 2.0 of the RV32I base integer
instruction set.  Much of the commentary also applies to the RV64I
variant.

\begin{commentary}
RV32I was designed to be sufficient to form a compiler target and to
support modern operating system environments.  The ISA was also
designed to reduce the hardware required in a minimal implementation.
RV32I contains 47 unique instructions, though a simple implementation
might cover the eight ECALL/EBREAK/CSRR* instructions with a single
SYSTEM hardware instruction that always traps and might be able to
implement the FENCE and FENCE.I instructions as NOPs, reducing
hardware instruction count to 38 total.  RV32I can emulate almost any
other ISA extension (except the A extension, which requires additional
hardware support for atomicity).
\end{commentary}

\section{Programmers' Model for Base Integer Subset}

Figure~\ref{gprs} shows the user-visible state for the base integer
subset.  There are 31 general-purpose registers {\tt x1}--{\tt x31},
which hold integer values.  Register {\tt x0} is hardwired to the
constant 0.  There is no hardwired subroutine return address link
register, but the standard software calling convention uses register
{\tt x1} to hold the return address on a call.  For RV32, the {\tt x}
registers are 32 bits wide, and for RV64, they are 64 bits wide.  This
document uses the term XLEN to refer to the current width of an {\tt
  x} register in bits (either 32 or 64).

There is one additional user-visible register: the program counter {\tt pc}
holds the address of the current instruction.

\begin{commentary}
The number of available architectural registers can have large impacts
on code size, performance, and energy consumption.  Although 16
registers would arguably be sufficient for an integer ISA running
compiled code, it is impossible to encode a complete ISA with 16
registers in 16-bit instructions using a 3-address format.  Although a
2-address format would be possible, it would increase instruction
count and lower efficiency.  We wanted to avoid intermediate
instruction sizes (such as Xtensa's 24-bit instructions) to simplify
base hardware implementations, and once a 32-bit instruction size was
adopted, it was straightforward to support 32 integer registers.  A
larger number of integer registers also helps performance on
high-performance code, where there can be extensive use of loop
unrolling, software pipelining, and cache tiling.

For these reasons, we chose a conventional size of 32 integer
registers for the base ISA.  Dynamic register usage tends to be
dominated by a few frequently accessed registers, and regfile
implementations can be optimized to reduce access energy for the
frequently accessed registers~\cite{jtseng:sbbci}.  The optional
compressed 16-bit instruction format mostly only accesses 8 registers
and hence can provide a dense instruction encoding, while additional
instruction-set extensions could support a much larger register space
(either flat or hierarchical) if desired.

For resource-constrained embedded applications, we have defined the
RV32E subset, which only has 16 registers (Chapter~\ref{rv32e}).
\end{commentary}

\begin{figure}[H]
{\footnotesize
\begin{center}
\begin{tabular}{p{2in}}
\instbitrange{XLEN-1}{0}                                  \\ \cline{1-1}
\multicolumn{1}{|c|}{\reglabel{\ \ \ \ \ \ x0 / zero}}      \\ \cline{1-1}
\multicolumn{1}{|c|}{\reglabel{\ \ \ \ x1\ \ \ \ \ }}            \\ \cline{1-1}
\multicolumn{1}{|c|}{\reglabel{\ \ \ \ x2\ \ \ \ \ }}       \\ \cline{1-1}
\multicolumn{1}{|c|}{\reglabel{\ \ \ \ x3\ \ \ \ \ }}       \\ \cline{1-1}
\multicolumn{1}{|c|}{\reglabel{\ \ \ \ x4\ \ \ \ \ }}       \\ \cline{1-1}
\multicolumn{1}{|c|}{\reglabel{\ \ \ \ x5\ \ \ \ \ }}       \\ \cline{1-1}
\multicolumn{1}{|c|}{\reglabel{\ \ \ \ x6\ \ \ \ \ }}       \\ \cline{1-1}
\multicolumn{1}{|c|}{\reglabel{\ \ \ \ x7\ \ \ \ \ }}       \\ \cline{1-1}
\multicolumn{1}{|c|}{\reglabel{\ \ \ \ x8\ \ \ \ \ }}       \\ \cline{1-1}
\multicolumn{1}{|c|}{\reglabel{\ \ \ \ x9\ \ \ \ \ }}       \\ \cline{1-1}
\multicolumn{1}{|c|}{\reglabel{\ \ \ x10\ \ \ \ \ }}        \\ \cline{1-1}
\multicolumn{1}{|c|}{\reglabel{\ \ \ x11\ \ \ \ \ }}        \\ \cline{1-1}
\multicolumn{1}{|c|}{\reglabel{\ \ \ x12\ \ \ \ \ }}        \\ \cline{1-1}
\multicolumn{1}{|c|}{\reglabel{\ \ \ x13\ \ \ \ \ }}        \\ \cline{1-1}
\multicolumn{1}{|c|}{\reglabel{\ \ \ x14\ \ \ \ \ }}        \\ \cline{1-1}
\multicolumn{1}{|c|}{\reglabel{\ \ \ x15\ \ \ \ \ }}        \\ \cline{1-1}
\multicolumn{1}{|c|}{\reglabel{\ \ \ x16\ \ \ \ \ }}        \\ \cline{1-1}
\multicolumn{1}{|c|}{\reglabel{\ \ \ x17\ \ \ \ \ }}        \\ \cline{1-1}
\multicolumn{1}{|c|}{\reglabel{\ \ \ x18\ \ \ \ \ }}        \\ \cline{1-1}
\multicolumn{1}{|c|}{\reglabel{\ \ \ x19\ \ \ \ \ }}        \\ \cline{1-1}
\multicolumn{1}{|c|}{\reglabel{\ \ \ x20\ \ \ \ \ }}        \\ \cline{1-1}
\multicolumn{1}{|c|}{\reglabel{\ \ \ x21\ \ \ \ \ }}        \\ \cline{1-1}
\multicolumn{1}{|c|}{\reglabel{\ \ \ x22\ \ \ \ \ }}        \\ \cline{1-1}
\multicolumn{1}{|c|}{\reglabel{\ \ \ x23\ \ \ \ \ }}        \\ \cline{1-1}
\multicolumn{1}{|c|}{\reglabel{\ \ \ x24\ \ \ \ \ }}        \\ \cline{1-1}
\multicolumn{1}{|c|}{\reglabel{\ \ \ x25\ \ \ \ \ }}        \\ \cline{1-1}
\multicolumn{1}{|c|}{\reglabel{\ \ \ x26\ \ \ \ \ }}        \\ \cline{1-1}
\multicolumn{1}{|c|}{\reglabel{\ \ \ x27\ \ \ \ \ }}        \\ \cline{1-1}
\multicolumn{1}{|c|}{\reglabel{\ \ \ x28\ \ \ \ \ }}        \\ \cline{1-1}
\multicolumn{1}{|c|}{\reglabel{\ \ \ x29\ \ \ \ \ }}        \\ \cline{1-1}
\multicolumn{1}{|c|}{\reglabel{\ \ \ x30\ \ \ \ \ }}        \\ \cline{1-1}
\multicolumn{1}{|c|}{\reglabel{\ \ \ x31\ \ \ \ \ }}        \\ \cline{1-1}
\multicolumn{1}{c}{XLEN}                                  \\

\instbitrange{XLEN-1}{0}                                  \\ \cline{1-1}
\multicolumn{1}{|c|}{\reglabel{pc}}                         \\ \cline{1-1}
\multicolumn{1}{c}{XLEN}                                  \\
\end{tabular}
\end{center}
}
\caption{RISC-V user-level base integer register state.}
\label{gprs}
\end{figure}

\newpage

\section{Base Instruction Formats}

In the base ISA, there are four core instruction formats (R/I/S/U), as
shown in Figure~\ref{fig:baseinstformats}.  All are a fixed 32 bits in
length and must be aligned on a four-byte boundary in memory.  An instruction address misaligned exception is generated on a
taken branch or unconditional jump if the target address is not
four-byte aligned.  No instruction fetch misaligned exception is
generated for a conditional branch that is not taken.

\begin{commentary}
The alignment constraint for base ISA instructions is relaxed to a
two-byte boundary when instruction extensions with 16-bit lengths or
other odd multiples of 16-bit lengths are added.
\end{commentary}

\vspace{-0.2in}
\begin{figure}[h]
\begin{center}
\setlength{\tabcolsep}{4pt}
\begin{tabular}{p{1.2in}@{}p{0.8in}@{}p{0.8in}@{}p{0.6in}@{}p{0.8in}@{}p{1in}l}
\\
\instbitrange{31}{25} &
\instbitrange{24}{20} &
\instbitrange{19}{15} &
\instbitrange{14}{12} &
\instbitrange{11}{7} &
\instbitrange{6}{0} \\
\cline{1-6}
\multicolumn{1}{|c|}{funct7} &
\multicolumn{1}{c|}{rs2} &
\multicolumn{1}{c|}{rs1} &
\multicolumn{1}{c|}{funct3} &
\multicolumn{1}{c|}{rd} &
\multicolumn{1}{c|}{opcode} &
R-type \\
\cline{1-6}
\\
\cline{1-6}
\multicolumn{2}{|c|}{imm[11:0]} &
\multicolumn{1}{c|}{rs1} &
\multicolumn{1}{c|}{funct3} &
\multicolumn{1}{c|}{rd} &
\multicolumn{1}{c|}{opcode} &
I-type \\
\cline{1-6}
\\
\cline{1-6}
\multicolumn{1}{|c|}{imm[11:5]} &
\multicolumn{1}{c|}{rs2} &
\multicolumn{1}{c|}{rs1} &
\multicolumn{1}{c|}{funct3} &
\multicolumn{1}{c|}{imm[4:0]} &
\multicolumn{1}{c|}{opcode} &
S-type \\
\cline{1-6}
\\
\cline{1-6}
\multicolumn{4}{|c|}{imm[31:12]} &
\multicolumn{1}{c|}{rd} &
\multicolumn{1}{c|}{opcode} &
U-type \\
\cline{1-6}
\end{tabular}
\end{center}
\caption{RISC-V base instruction formats.  Each immediate subfield is
  labeled with the bit position (imm[{\em x}\,]) in the immediate
  value being produced, rather than the bit position within the
  instruction's immediate field as is usually done.  }
\label{fig:baseinstformats}
\end{figure}

The RISC-V ISA keeps the source ({\em rs1} and {\em rs2}) and
destination ({\em rd}) registers at the same position in all formats
to simplify decoding.  Except for the 5-bit immediates used in CSR
instructions (Section~\ref{sec:csrinsts}), immediates are always
sign-extended, and are generally packed towards the leftmost available
bits in the instruction and have been allocated to reduce hardware
complexity.  In particular, the sign bit for all immediates is always
in bit 31 of the instruction to speed sign-extension circuitry.

\begin{commentary}
Decoding register specifiers is usually on the critical paths in
implementations, and so the instruction format was chosen to keep all
register specifiers at the same position in all formats at the expense
of having to move immediate bits across formats (a property shared
with RISC-IV aka. SPUR~\cite{spur-jsscc1989}).

In practice, most immediates are either small or require all XLEN
bits.  We chose an asymmetric immediate split (12 bits in regular
instructions plus a special load upper immediate instruction with 20
bits) to increase the opcode space available for regular instructions.

Immediates are sign-extended because we did not observe a benefit to
using zero-extension for some immediates as in the MIPS ISA and wanted
to keep the ISA as simple as possible.
\end{commentary}

\section{Immediate Encoding Variants}

There are a further two variants of the instruction formats (B/J)
based on the handling of immediates, as shown in
Figure~\ref{fig:baseinstformatsimm}.

The only difference between the S and B formats is that the 12-bit
immediate field is used to encode branch offsets in multiples of 2 in
the B format.  Instead of shifting all bits in the
instruction-encoded immediate left by one in hardware as is
conventionally done, the middle bits (imm[10:1]) and sign bit stay in
fixed positions, while the lowest bit in S format (inst[7]) encodes a
high-order bit in B format.

Similarly, the only difference between the U and J formats is
that the 20-bit immediate is shifted left by 12 bits to form U
immediates and by 1 bit to form J immediates.  The location of
instruction bits in the U and J format immediates is chosen to
maximize overlap with the other formats and with each other.

\begin{figure}[h]
\begin{small}
\begin{center}
\setlength{\tabcolsep}{4pt}
\begin{tabular}{p{0.3in}@{}p{0.8in}@{}p{0.6in}@{}p{0.18in}@{}p{0.7in}@{}p{0.6in}@{}p{0.6in}@{}p{0.3in}@{}p{0.5in}l}
\\
\multicolumn{1}{c}{\instbit{31}} &
\instbitrange{30}{25} &
\instbitrange{24}{21} &
\multicolumn{1}{c}{\instbit{20}} &
\instbitrange{19}{15} &
\instbitrange{14}{12} &
\instbitrange{11}{8} &
\multicolumn{1}{c}{\instbit{7}} &
\instbitrange{6}{0} \\
\cline{1-9}
\multicolumn{2}{|c|}{funct7} &
\multicolumn{2}{c|}{rs2} &
\multicolumn{1}{c|}{rs1} &
\multicolumn{1}{c|}{funct3} &
\multicolumn{2}{c|}{rd} &
\multicolumn{1}{c|}{opcode} &
R-type \\
\cline{1-9}
\\
\cline{1-9}
\multicolumn{4}{|c|}{imm[11:0]} &
\multicolumn{1}{c|}{rs1} &
\multicolumn{1}{c|}{funct3} &
\multicolumn{2}{c|}{rd} &
\multicolumn{1}{c|}{opcode} &
I-type \\
\cline{1-9}
\\
\cline{1-9}
\multicolumn{2}{|c|}{imm[11:5]} &
\multicolumn{2}{c|}{rs2} &
\multicolumn{1}{c|}{rs1} &
\multicolumn{1}{c|}{funct3} &
\multicolumn{2}{c|}{imm[4:0]} &
\multicolumn{1}{c|}{opcode} &
S-type \\
\cline{1-9}
\\
\cline{1-9}
\multicolumn{1}{|c|}{imm[12]} &
\multicolumn{1}{c|}{imm[10:5]} &
\multicolumn{2}{c|}{rs2} &
\multicolumn{1}{c|}{rs1} &
\multicolumn{1}{c|}{funct3} &
\multicolumn{1}{c|}{imm[4:1]} &
\multicolumn{1}{c|}{imm[11]} &
\multicolumn{1}{c|}{opcode} &
B-type \\
\cline{1-9}
\\
\cline{1-9}
\multicolumn{6}{|c|}{imm[31:12]} &
\multicolumn{2}{c|}{rd} &
\multicolumn{1}{c|}{opcode} &
U-type \\
\cline{1-9}
\\
\cline{1-9}
\multicolumn{1}{|c|}{imm[20]} &
\multicolumn{2}{c|}{imm[10:1]} &
\multicolumn{1}{c|}{imm[11]} &
\multicolumn{2}{c|}{imm[19:12]} &
\multicolumn{2}{c|}{rd} &
\multicolumn{1}{c|}{opcode} &
J-type \\
\cline{1-9}
\end{tabular}
\end{center}
\end{small}
\caption{RISC-V base instruction formats showing immediate variants.}
\label{fig:baseinstformatsimm}
\end{figure}

Figure~\ref{fig:immtypes} shows the immediates produced by each of the
base instruction formats, and is labeled to show which instruction
bit (inst[{\em y}\,]) produces each bit of the immediate value.

\begin{figure}[h]
\begin{center}
\setlength{\tabcolsep}{4pt}
\begin{tabular}{p{0.2in}@{}p{1.2in}@{}p{1.0in}@{}p{0.2in}@{}p{0.7in}@{}p{0.7in}@{}p{0.2in}l}
\\
\multicolumn{1}{c}{\instbit{31}} &
\instbitrange{30}{20} &
\instbitrange{19}{12} &
\multicolumn{1}{c}{\instbit{11}} &
\instbitrange{10}{5} &
\instbitrange{4}{1} &
\multicolumn{1}{c}{\instbit{0}} &
\\
\cline{1-7}
\multicolumn{4}{|c|}{--- inst[31] ---} &
\multicolumn{1}{c|}{inst[30:25]} &
\multicolumn{1}{c|}{inst[24:21]} &
\multicolumn{1}{c|}{inst[20]} &
I-immediate \\
\cline{1-7}
\\
\cline{1-7}
\multicolumn{4}{|c|}{--- inst[31] ---} &
\multicolumn{1}{c|}{inst[30:25]} &
\multicolumn{1}{c|}{inst[11:8]} &
\multicolumn{1}{c|}{inst[7]} &
S-immediate \\
\cline{1-7}
\\
\cline{1-7}
\multicolumn{3}{|c|}{--- inst[31] ---} &
\multicolumn{1}{c|}{inst[7]} &
\multicolumn{1}{c|}{inst[30:25]} &
\multicolumn{1}{c|}{inst[11:8]} &
\multicolumn{1}{c|}{0} &
B-immediate \\
\cline{1-7}
\\
\cline{1-7}
\multicolumn{1}{|c|}{inst[31]} &
\multicolumn{1}{c|}{inst[30:20]} &
\multicolumn{1}{c|}{inst[19:12]} &
\multicolumn{4}{c|}{--- 0 ---} &
U-immediate \\
\cline{1-7}
\\
\cline{1-7}
\multicolumn{2}{|c|}{--- inst[31] ---} &
\multicolumn{1}{c|}{inst[19:12]} &
\multicolumn{1}{c|}{inst[20]} &
\multicolumn{1}{c|}{inst[30:25]} &
\multicolumn{1}{c|}{inst[24:21]} &
\multicolumn{1}{c|}{0} &
J-immediate \\
\cline{1-7}
\end{tabular}
\end{center}
\caption{Types of immediate produced by RISC-V instructions.  The fields are labeled with the
  instruction bits used to construct their value.  Sign extension
  always uses inst[31].}
\label{fig:immtypes}
\end{figure}

\begin{commentary}
Sign-extension is one of the most critical operations on immediates
(particularly in RV64I), and in RISC-V the sign bit for all immediates
is always held in bit 31 of the instruction to allow sign-extension to
proceed in parallel with instruction decoding.

Although more complex implementations might have separate adders for
branch and jump calculations and so would not benefit from keeping the
location of immediate bits constant across types of instruction, we
wanted to reduce the hardware cost of the simplest implementations.
By rotating bits in the instruction encoding of B and J immediates
instead of using dynamic hardware muxes to multiply the immediate by
2, we reduce instruction signal fanout and immediate mux costs by
around a factor of 2.  The scrambled immediate encoding will add
negligible time to static or ahead-of-time compilation.  For dynamic
generation of instructions, there is some small additional
overhead, but the most common short forward branches have
straightforward immediate encodings.
\end{commentary}

\section{Integer Computational Instructions}

Most integer computational instructions operate on XLEN bits of values
held in the integer register file.  Integer computational instructions
are either encoded as register-immediate operations using the I-type
format or as register-register operations using the R-type format.
The destination is register {\em rd} for both register-immediate and
register-register instructions.  No integer computational instructions
cause arithmetic exceptions.

\begin{commentary}
We did not include special instruction set support for overflow checks
on integer arithmetic operations in the base instruction set, as many
overflow checks can be cheaply implemented using RISC-V branches.
Overflow checking for unsigned addition requires only a single
additional branch instruction after the addition:
\verb! add t0, t1, t2; bltu t0, t1, overflow!.

For signed addition, if one operand's sign is known, overflow checking
requires only a single branch after the addition:
\verb! addi t0, t1, +imm; blt t0, t1, overflow!.  This covers the
common case of addition with an immediate operand.

For general signed addition, three additional instructions after the
addition are required, leveraging the observation that the sum should
be less than one of the operands if and only if the other operand is
negative.
\begin{verbatim}
         add t0, t1, t2
         slti t3, t2, 0
         slt t4, t0, t1
         bne t3, t4, overflow
\end{verbatim}
In RV64, checks of 32-bit signed additions can be optimized further by
comparing the results of ADD and ADDW on the operands.
\end{commentary}

\subsubsection*{Integer Register-Immediate Instructions}
\vspace{-0.4in}
\begin{center}
\begin{tabular}{M@{}R@{}S@{}R@{}O}
\\
\instbitrange{31}{20} &
\instbitrange{19}{15} &
\instbitrange{14}{12} &
\instbitrange{11}{7} &
\instbitrange{6}{0} \\
\hline
\multicolumn{1}{|c|}{imm[11:0]} &
\multicolumn{1}{c|}{rs1} &
\multicolumn{1}{c|}{funct3} &
\multicolumn{1}{c|}{rd} &
\multicolumn{1}{c|}{opcode} \\
\hline
12 & 5 & 3 & 5 & 7 \\
I-immediate[11:0] & src & ADDI/SLTI[U]  & dest & OP-IMM \\
I-immediate[11:0] & src & ANDI/ORI/XORI & dest & OP-IMM \\
\end{tabular}
\end{center}
ADDI adds the sign-extended 12-bit immediate to register {\em rs1}.
Arithmetic overflow is ignored and the result is simply the low
XLEN bits of the result.  ADDI {\em rd, rs1, 0} is used to implement the
MV {\em rd, rs1} assembler pseudo-instruction.

SLTI (set less than immediate) places the value 1 in register {\em rd}
if register {\em rs1} is less than the sign-extended immediate when
both are treated as signed numbers, else 0 is written to {\em rd}.
SLTIU is similar but compares the values as unsigned numbers (i.e.,
the immediate is first sign-extended to XLEN bits then treated as an
unsigned number).  Note, SLTIU {\em rd}, {\em rs1}, 1 sets {\em rd}
to 1 if {\em rs1} equals zero, otherwise sets {\em rd} to 0 (assembler
pseudo-op SEQZ {\em rd, rs}).

ANDI, ORI, XORI are logical operations that perform bitwise AND, OR,
and XOR on register {\em rs1} and the sign-extended 12-bit immediate
and place the result in {\em rd}.  Note, XORI {\em rd, rs1, -1}
performs a bitwise logical inversion of register {\em rs1} (assembler
pseudo-instruction NOT {\em rd, rs}).

\vspace{-0.2in}
\begin{center}
\begin{tabular}{S@{}R@{}R@{}S@{}R@{}O}
\\
\instbitrange{31}{25} &
\instbitrange{24}{20} &
\instbitrange{19}{15} &
\instbitrange{14}{12} &
\instbitrange{11}{7} &
\instbitrange{6}{0} \\
\hline
\multicolumn{1}{|c|}{imm[11:5]} &
\multicolumn{1}{c|}{imm[4:0]} &
\multicolumn{1}{c|}{rs1} &
\multicolumn{1}{c|}{funct3} &
\multicolumn{1}{c|}{rd} &
\multicolumn{1}{c|}{opcode} \\
\hline
7 & 5 & 5 & 3 & 5 & 7 \\
0000000 & shamt[4:0]  & src & SLLI & dest & OP-IMM \\
0000000 & shamt[4:0]  & src & SRLI & dest & OP-IMM \\
0100000 & shamt[4:0]  & src & SRAI & dest & OP-IMM \\
\end{tabular}
\end{center}

Shifts by a constant are encoded as a specialization of the
I-type format.  The operand to be shifted is in {\em rs1}, and the
shift amount is encoded in the lower 5 bits of the I-immediate field.
The right shift type is encoded in bit 30.
SLLI is a logical left shift (zeros are shifted into the lower bits);
SRLI is a logical right shift (zeros are shifted into the upper bits);
and SRAI is an arithmetic right shift (the original sign bit is copied
into the vacated upper bits).

\vspace{-0.2in}
\begin{center}
\begin{tabular}{U@{}R@{}O}
\\
\instbitrange{31}{12} &
\instbitrange{11}{7} &
\instbitrange{6}{0} \\
\hline
\multicolumn{1}{|c|}{imm[31:12]} &
\multicolumn{1}{c|}{rd} &
\multicolumn{1}{c|}{opcode} \\
\hline
20 & 5 & 7 \\
U-immediate[31:12] & dest & LUI \\
U-immediate[31:12] & dest & AUIPC
\end{tabular}
\end{center}

LUI (load upper immediate) is used to build 32-bit constants and uses
the U-type format.  LUI places the U-immediate value in the top 20
bits of the destination register {\em rd}, filling in the lowest 12
bits with zeros.

AUIPC (add upper immediate to {\tt pc}) is used to build {\tt pc}-relative
addresses and uses the U-type format.  AUIPC forms a 32-bit offset from the
20-bit U-immediate, filling in the lowest 12 bits with zeros, adds this offset
to the {\tt pc}, then places the result in register {\em rd}.

\begin{commentary}
The AUIPC instruction supports two-instruction sequences to access
arbitrary offsets from the PC for both control-flow transfers and data
accesses.  The combination of an AUIPC and the 12-bit immediate in a
JALR can transfer control to any 32-bit PC-relative address, while an
AUIPC plus the 12-bit immediate offset in regular load or store
instructions can access any 32-bit PC-relative data address.

The current PC can be obtained by setting the U-immediate to 0.  Although
a JAL +4 instruction could also be used to obtain the PC, it might cause
pipeline breaks in simpler microarchitectures or pollute the BTB structures in
more complex microarchitectures.
\end{commentary}

\subsubsection*{Integer Register-Register Operations}

RV32I defines several arithmetic R-type operations.  All operations
read the {\em rs1} and {\em rs2} registers as source operands and
write the result into register {\em rd}.  The {\em funct7} and {\em
  funct3} fields select the type of operation.

\vspace{-0.2in}
\begin{center}
\begin{tabular}{S@{}R@{}R@{}S@{}R@{}O}
\\
\instbitrange{31}{25} &
\instbitrange{24}{20} &
\instbitrange{19}{15} &
\instbitrange{14}{12} &
\instbitrange{11}{7} &
\instbitrange{6}{0} \\
\hline
\multicolumn{1}{|c|}{funct7} &
\multicolumn{1}{c|}{rs2} &
\multicolumn{1}{c|}{rs1} &
\multicolumn{1}{c|}{funct3} &
\multicolumn{1}{c|}{rd} &
\multicolumn{1}{c|}{opcode} \\
\hline
7 & 5 & 5 & 3 & 5 & 7 \\
0000000 & src2 & src1 & ADD/SLT/SLTU & dest & OP    \\
0000000 & src2 & src1 & AND/OR/XOR  & dest & OP    \\
0000000 & src2 & src1 & SLL/SRL     & dest & OP    \\
0100000 & src2 & src1 & SUB/SRA     & dest & OP    \\
\end{tabular}
\end{center}

ADD and SUB perform addition and subtraction respectively.  Overflows
are ignored and the low XLEN bits of results are written to the
destination.  SLT and SLTU perform signed and unsigned compares
respectively, writing 1 to {\em rd} if $\mbox{\em rs1} < \mbox{\em
  rs2}$, 0 otherwise.  Note, SLTU {\em rd}, {\em x0}, {\em rs2} sets
{\em rd} to 1 if {\em rs2} is not equal to zero, otherwise sets {\em
  rd} to zero (assembler pseudo-op SNEZ {\em rd, rs}).  AND, OR, and
XOR perform bitwise logical operations.

SLL, SRL, and SRA perform logical left, logical right, and arithmetic
right shifts on the value in register {\em rs1} by the shift amount
held in the lower 5 bits of register {\em rs2}.

\subsubsection*{NOP Instruction}
\vspace{-0.4in}
\begin{center}
\begin{tabular}{M@{}R@{}S@{}R@{}O}
\\
\instbitrange{31}{20} &
\instbitrange{19}{15} &
\instbitrange{14}{12} &
\instbitrange{11}{7} &
\instbitrange{6}{0} \\
\hline
\multicolumn{1}{|c|}{imm[11:0]} &
\multicolumn{1}{c|}{rs1} &
\multicolumn{1}{c|}{funct3} &
\multicolumn{1}{c|}{rd} &
\multicolumn{1}{c|}{opcode} \\
\hline
12 & 5 & 3 & 5 & 7 \\
0 & 0 & ADDI & 0 & OP-IMM \\
\end{tabular}
\end{center}

The NOP instruction does not change any user-visible state, except
for advancing the {\tt pc}.  NOP is encoded as ADDI {\em x0, x0, 0}.

\begin{commentary}
NOPs can be used to align code segments to microarchitecturally
significant address boundaries, or to leave space for inline code
modifications.  Although there are many possible ways to encode a NOP,
we define a canonical NOP encoding to allow microarchitectural
optimizations as well as for more readable disassembly output.
\end{commentary}

\section{Control Transfer Instructions}

RV32I provides two types of control transfer instructions:
unconditional jumps and conditional branches.  Control transfer
instructions in RV32I do {\em not} have architecturally visible delay
slots.

\subsubsection*{Unconditional Jumps}

\vspace{-0.1in} The jump and link (JAL) instruction uses the J-type
format, where the J-immediate encodes a signed offset in multiples of
2 bytes.  The offset is sign-extended and added to the {\tt pc}
to form the jump target address.  Jumps can therefore target a
$\pm$\wunits{1}{MiB} range. JAL stores the address of the instruction
following the jump ({\tt pc}+4) into register {\em rd}.  The standard
software calling convention uses {\tt x1} as the return address
register and {\tt x5} as an alternate link register.

\begin{commentary}
The alternate link register supports calling millicode routines (e.g.,
those to save and restore registers in compressed code) while
preserving the regular return address register.  The register {\tt x5}
was chosen as the alternate link register as it maps to a temporary in
the standard calling convention, and has an encoding that is only one
bit different than the regular link register.
\end{commentary}

Plain unconditional jumps (assembler pseudo-op J) are encoded as a JAL
with {\em rd}={\tt x0}.

\vspace{-0.2in}
\begin{center}
\begin{tabular}{W@{}E@{}W@{}R@{}R@{}O}
\\
\multicolumn{1}{c}{\instbit{31}} &
\instbitrange{30}{21} &
\multicolumn{1}{c}{\instbit{20}} &
\instbitrange{19}{12} &
\instbitrange{11}{7} &
\instbitrange{6}{0} \\
\hline
\multicolumn{1}{|c|}{imm[20]} &
\multicolumn{1}{c|}{imm[10:1]} &
\multicolumn{1}{c|}{imm[11]} &
\multicolumn{1}{c|}{imm[19:12]} &
\multicolumn{1}{c|}{rd} &
\multicolumn{1}{c|}{opcode} \\
\hline
1 & 10 & \multicolumn{1}{c}{1} & 8 & 5 & 7 \\
\multicolumn{4}{c}{offset[20:1]} & dest & JAL \\
\end{tabular}
\end{center}

The indirect jump instruction JALR (jump and link register) uses the
I-type encoding.  The target address is obtained by adding the 12-bit
signed I-immediate to the register {\em rs1}, then setting the
least-significant bit of the result to zero.  The address of
the instruction following the jump ({\tt pc}+4) is written to register
{\em rd}.  Register {\tt x0} can be used as the destination if the
result is not required.
\vspace{-0.4in}
\begin{center}
\begin{tabular}{M@{}R@{}F@{}R@{}O}
\\
\instbitrange{31}{20} &
\instbitrange{19}{15} &
\instbitrange{14}{12} &
\instbitrange{11}{7} &
\instbitrange{6}{0} \\
\hline
\multicolumn{1}{|c|}{imm[11:0]} &
\multicolumn{1}{c|}{rs1} &
\multicolumn{1}{c|}{funct3} &
\multicolumn{1}{c|}{rd} &
\multicolumn{1}{c|}{opcode} \\
\hline
12 & 5 & 3 & 5 & 7 \\
offset[11:0] & base & 0 & dest & JALR \\
\end{tabular}
\end{center}

\begin{commentary}
The unconditional jump instructions all use PC-relative addressing to
help support position-independent code.  The JALR instruction was
defined to enable a two-instruction sequence to jump anywhere in a
32-bit absolute address range.  A LUI instruction can first load {\em
  rs1} with the upper 20 bits of a target address, then JALR can add
in the lower bits. Similarly, AUIPC then JALR can jump
anywhere in a 32-bit {\tt pc}-relative address range.

Note that the JALR instruction does not treat the 12-bit immediate as
multiples of 2 bytes, unlike the conditional branch instructions.
This avoids one more immediate format in hardware.  In
practice, most uses of JALR will have either a zero immediate or be
paired with a LUI or AUIPC, so the slight reduction in range is not
significant.

Clearing the least-significant bit when calculating the JALR target
address both simplifies the hardware slightly and allows the
low bit of function pointers to be used to store auxiliary
information.  Although there is potentially a slight loss of error
checking in this case, in practice jumps to an incorrect instruction
address will usually quickly raise an exception.

When used with a base {\em rs1}$=${\tt x0}, JALR can be used to implement
a single instruction subroutine call to the lowest \wunits{2}{KiB} or highest
\wunits{2}{KiB} address region from anywhere in the address space, which could
be used to implement fast calls to a small runtime library.
\end{commentary}

The JAL and JALR instructions will generate a misaligned instruction
fetch exception if the target address is not aligned to a four-byte
boundary.

\begin{commentary}
Instruction fetch misaligned exceptions are not possible on machines
that support extensions with 16-bit aligned instructions, such as the
compressed instruction set extension, C.
\end{commentary}

Return-address prediction stacks are a common feature of
high-performance instruction-fetch units, but require accurate
detection of instructions used for procedure calls and returns to be
effective.  For RISC-V, hints as to the instructions' usage are encoded
implicitly via the register numbers used.  A JAL instruction should
push the return address onto a return-address stack (RAS) only when
{\em rd}$=${\tt x1}/{\tt x5}.  JALR instructions should push/pop a
RAS as shown in the Table~\ref{rashints}.
\begin{table}[hbt]
\centering
\begin{tabular}{|c|c|c|l|}
  \hline
  \em rd & \em rs1 & {\em rs1}$=${\em rd} & RAS action \\
  \hline
  !{\em link} & !{\em link} & - & none \\
  !{\em link} &  {\em link} & - & pop \\
   {\em link} & !{\em link} & - & push  \\
   {\em link} &  {\em link} & 0 & pop, then push \\
   {\em link} &  {\em link} & 1 & push \\
   \hline
\end{tabular}
\caption{Return-address stack prediction hints encoded in register
  specifiers used in the instruction.  In the above, {\em link} is
  true when the register is either {\tt x1} or {\tt x5}.}
\label{rashints}
\end{table}

\begin{commentary}
Some other ISAs added explicit hint bits to their indirect-jump instructions
to guide return-address stack manipulation.  We use implicit hinting tied to
register numbers and the calling convention to reduce the encoding space used
for these hints.

When two different link registers ({\tt x1} and {\tt x5}) are given as
{\em rs1} and {\em rd}, then the RAS is both popped and pushed to
support coroutines.  If {\em rs1} and {\em rd} are the same link
register (either {\tt x1} or {\tt x5}), the RAS is only pushed to
enable macro-op fusion of the sequences:\linebreak
{\tt lui ra, imm20; jalr ra, ra, imm12} \ and \ 
{\tt auipc ra, imm20; jalr ra, ra, imm12}
\end{commentary}

\subsubsection*{Conditional Branches}

All branch instructions use the B-type instruction format.  The
12-bit B-immediate encodes signed offsets in multiples of 2, and is
added to the current {\tt pc} to give the target address.  The
conditional branch range is $\pm$\wunits{4}{KiB}.

\vspace{-0.2in}
\begin{center}
\begin{tabular}{W@{}R@{}F@{}F@{}R@{}R@{}F@{}S}
\\
\multicolumn{1}{c}{\instbit{31}} &
\instbitrange{30}{25} &
\instbitrange{24}{20} &
\instbitrange{19}{15} &
\instbitrange{14}{12} &
\instbitrange{11}{8} &
\multicolumn{1}{c}{\instbit{7}} &
\instbitrange{6}{0} \\
\hline
\multicolumn{1}{|c|}{imm[12]} &
\multicolumn{1}{c|}{imm[10:5]} &
\multicolumn{1}{c|}{rs2} &
\multicolumn{1}{c|}{rs1} &
\multicolumn{1}{c|}{funct3} &
\multicolumn{1}{c|}{imm[4:1]} &
\multicolumn{1}{c|}{imm[11]} &
\multicolumn{1}{c|}{opcode} \\
\hline
1 & 6 & 5 & 5 & 3 & 4 & 1 & 7 \\
\multicolumn{2}{c}{offset[12,10:5]} & src2 & src1 & BEQ/BNE & \multicolumn{2}{c}{offset[11,4:1]} & BRANCH \\
\multicolumn{2}{c}{offset[12,10:5]} & src2 & src1 & BLT[U] & \multicolumn{2}{c}{offset[11,4:1]} & BRANCH \\
\multicolumn{2}{c}{offset[12,10:5]} & src2 & src1 & BGE[U]  & \multicolumn{2}{c}{offset[11,4:1]} & BRANCH \\
\end{tabular}
\end{center}

Branch instructions compare two registers.  BEQ and BNE take the
branch if registers {\em rs1} and {\em rs2} are equal or unequal
respectively.  BLT and BLTU take the branch if {\em rs1} is less than
{\em rs2}, using signed and unsigned comparison respectively.  BGE and
BGEU take the branch if {\em rs1} is greater than or equal to {\em rs2},
using signed and unsigned comparison respectively. Note, BGT, BGTU,
BLE, and BLEU can be synthesized by reversing the operands to BLT,
BLTU, BGE, and BGEU, respectively.

\begin{commentary}
Signed array bounds may be checked with a single BLTU instruction, since
any negative index will compare greater than any nonnegative bound.
\end{commentary}

Software should be optimized such that the sequential code path is the
most common path, with less-frequently taken code paths placed out of
line.  Software should also assume that backward branches will be
predicted taken and forward branches as not taken, at least the
first time they are encountered.  Dynamic predictors should quickly
learn any predictable branch behavior.

Unlike some other architectures, the RISC-V jump (JAL with {\em
  rd}={\tt x0}) instruction should always be used for unconditional
branches instead of a conditional branch instruction with an always-true
condition.  RISC-V jumps are also PC-relative and support a much
wider offset range than branches, and will not pressure conditional
branch prediction tables.

\begin{commentary}
The conditional branches were designed to include arithmetic
comparison operations between two registers (as also done in PA-RISC
and Xtensa ISA), rather than use condition codes (x86, ARM, SPARC,
PowerPC), or to only compare one register against zero (Alpha, MIPS),
or two registers only for equality (MIPS).  This design was motivated
by the observation that a combined compare-and-branch instruction fits
into a regular pipeline, avoids additional condition code state or use
of a temporary register, and reduces static code size and dynamic
instruction fetch traffic.  Another point is that comparisons against
zero require non-trivial circuit delay (especially after the move to
static logic in advanced processes) and so are almost as expensive as
arithmetic magnitude compares.  Another advantage of a fused
compare-and-branch instruction is that branches are observed earlier
in the front-end instruction stream, and so can be predicted earlier.
There is perhaps an advantage to a design with condition codes in the
case where multiple branches can be taken based on the same condition
codes, but we believe this case to be relatively rare.

We considered but did not include static branch hints in the
instruction encoding.  These can reduce the pressure on dynamic
predictors, but require more instruction encoding space and
software profiling for best results, and can result in poor
performance if production runs do not match profiling runs.

We considered but did not include conditional moves or predicated
instructions, which can effectively replace unpredictable short
forward branches.  Conditional moves are the simpler of the two, but
are difficult to use with conditional code that might cause exceptions
(memory accesses and floating-point operations).  Predication adds
additional flag state to a system, additional instructions to set and
clear flags, and additional encoding overhead on every instruction.
Both conditional move and predicated instructions add complexity to
out-of-order microarchitectures, adding an implicit third source
operand due to the need to copy the original value of the destination
architectural register into the renamed destination physical register
if the predicate is false.  Also, static compile-time decisions to use
predication instead of branches can result in lower performance on
inputs not included in the compiler training set, especially given
that unpredictable branches are rare, and becoming rarer as branch
prediction techniques improve.

We note that various microarchitectural techniques exist to
dynamically convert unpredictable short forward branches into
internally predicated code to avoid the cost of flushing pipelines on
a branch mispredict~\cite{heil-tr1996,Klauser-1998,Kim-micro2005} and
have been implemented in commercial processors~\cite{ibmpower7}.
The simplest techniques just reduce the penalty of recovering from a
mispredicted short forward branch by only flushing instructions in the
branch shadow instead of the entire fetch pipeline, or by fetching
instructions from both sides using wide instruction fetch or idle
instruction fetch slots.  More complex techniques for out-of-order
cores add internal predicates on instructions in the branch shadow,
with the internal predicate value written by the branch instruction,
allowing the branch and following instructions to be executed
speculatively and out-of-order with respect to other code~\cite{ibmpower7}.
\end{commentary}

\section{Load and Store Instructions}

RV32I is a load-store architecture, where only load and store
instructions access memory and arithmetic instructions only operate on
CPU registers.  RV32I provides a 32-bit user address space that is
byte-addressed and little-endian.  The execution environment will
define what portions of the address space are legal to access.  Loads
with a destination of {\tt x0} must still raise any exceptions and
action any other side effects even though the load value is discarded.

\vspace{-0.4in}
\begin{center}
\begin{tabular}{M@{}R@{}F@{}R@{}O}
\\
\instbitrange{31}{20} &
\instbitrange{19}{15} &
\instbitrange{14}{12} &
\instbitrange{11}{7} &
\instbitrange{6}{0} \\
\hline
\multicolumn{1}{|c|}{imm[11:0]} &
\multicolumn{1}{c|}{rs1} &
\multicolumn{1}{c|}{funct3} &
\multicolumn{1}{c|}{rd} &
\multicolumn{1}{c|}{opcode} \\
\hline
12 & 5 & 3 & 5 & 7 \\
offset[11:0] & base & width & dest & LOAD \\
\end{tabular}
\end{center}

\vspace{-0.2in}
\begin{center}
\begin{tabular}{O@{}R@{}R@{}F@{}R@{}O}
\\
\instbitrange{31}{25} &
\instbitrange{24}{20} &
\instbitrange{19}{15} &
\instbitrange{14}{12} &
\instbitrange{11}{7} &
\instbitrange{6}{0} \\
\hline
\multicolumn{1}{|c|}{imm[11:5]} &
\multicolumn{1}{c|}{rs2} &
\multicolumn{1}{c|}{rs1} &
\multicolumn{1}{c|}{funct3} &
\multicolumn{1}{c|}{imm[4:0]} &
\multicolumn{1}{c|}{opcode} \\
\hline
7 & 5 & 5 & 3 & 5 & 7 \\
offset[11:5] & src & base & width & offset[4:0] & STORE \\
\end{tabular}
\end{center}

Load and store instructions transfer a value between the registers and
memory.  Loads are encoded in the I-type format and stores are
S-type.  The effective byte address is obtained by adding register
{\em rs1} to the sign-extended 12-bit offset.  Loads copy a value
from memory to register {\em rd}.  Stores copy the value in register
{\em rs2} to memory.

The LW instruction loads a 32-bit value from memory into {\em rd}.  LH
loads a 16-bit value from memory, then sign-extends to 32-bits before
storing in {\tt rd}. LHU loads a 16-bit value from memory but then
zero extends to 32-bits before storing in {\em rd}.  LB and LBU are
defined analogously for 8-bit values.  The SW, SH, and SB instructions
store 32-bit, 16-bit, and 8-bit values from the low bits of register
{\em rs2} to memory.

For best performance, the effective address for all loads and stores
should be naturally aligned for each data type (i.e., on a four-byte
boundary for 32-bit accesses, and a two-byte boundary for 16-bit
accesses).  The base ISA supports misaligned accesses, but these might
run extremely slowly depending on the implementation.  Furthermore,
naturally aligned loads and stores are guaranteed to execute
atomically, whereas misaligned loads and stores might not, and hence
require additional synchronization to ensure atomicity.

\begin{commentary}
Misaligned accesses are occasionally required when porting legacy
code, and are essential for good performance on many applications when
using any form of packed-SIMD extension.  Our rationale for supporting
misaligned accesses via the regular load and store instructions is to
simplify the addition of misaligned hardware support.  One option
would have been to disallow misaligned accesses in the base ISA and
then provide some separate ISA support for misaligned accesses, either
special instructions to help software handle misaligned accesses or a
new hardware addressing mode for misaligned accesses.  Special
instructions are difficult to use, complicate the ISA, and often add
new processor state (e.g., SPARC VIS align address offset register) or
complicate access to existing processor state (e.g., MIPS LWL/LWR
partial register writes).  In addition, for loop-oriented packed-SIMD
code, the extra overhead when operands are misaligned motivates
software to provide multiple forms of loop depending on operand
alignment, which complicates code generation and adds to loop startup
overhead.  New misaligned hardware addressing modes take considerable
space in the instruction encoding or require very simplified
addressing modes (e.g., register indirect only).

We do not mandate atomicity for misaligned accesses so simple
implementations can just use a machine trap and software handler to
handle some or all misaligned accesses.  If hardware misaligned support is
provided, software can exploit this by simply using regular load and
store instructions.  Hardware can then automatically optimize accesses
depending on whether runtime addresses are aligned.
\end{commentary}

\section{Memory Model}

\begin{commentary}
This section is out of date as the RISC-V memory model is currently
under revision to ensure it can efficiently support current
programming language memory models.  The revised base memory model
will contain further ordering constraints, including at least that
loads to the same address from the same hart cannot be reordered, and
that syntactic data dependencies between instructions are respected.
\end{commentary}

The base RISC-V ISA supports multiple concurrent threads of execution
within a single user address space.  Each RISC-V hardware thread, or
{\em hart}, has its own user register state and program counter, and
executes an independent sequential instruction stream.  The execution
environment will define how RISC-V harts are created and managed.
RISC-V harts can communicate and synchronize with other harts either
via calls to the execution environment, which are documented
separately in the specification for each execution environment, or
directly via the shared memory system.  RISC-V harts can also interact
with I/O devices, and indirectly with each other, via loads and stores
to portions of the address space assigned to I/O.

\begin{commentary}
  We use the term {\em hart} to unambiguously and concisely describe a
  hardware thread as opposed to software-managed thread contexts.
\end{commentary}

In the base RISC-V ISA, each RISC-V hart observes its own memory
operations as if they executed sequentially in program order.  RISC-V
has a relaxed memory model between harts, requiring an explicit
FENCE instruction to guarantee ordering between memory
operations from different RISC-V harts.  Chapter~\ref{atomics}
describes the optional atomic memory instruction extensions ``A'',
which provide additional synchronization operations.

\vspace{-0.2in}
\begin{center}
\begin{tabular}{F@{}IIIIIIIIF@{}F@{}F@{}S}
\\
\instbitrange{31}{28} &
\multicolumn{1}{c}{\instbit{27}} &
\multicolumn{1}{c}{\instbit{26}} &
\multicolumn{1}{c}{\instbit{25}} &
\multicolumn{1}{c}{\instbit{24}} &
\multicolumn{1}{c}{\instbit{23}} &
\multicolumn{1}{c}{\instbit{22}} &
\multicolumn{1}{c}{\instbit{21}} &
\multicolumn{1}{c}{\instbit{20}} &
\instbitrange{19}{15} &
\instbitrange{14}{12} &
\instbitrange{11}{7} &
\instbitrange{6}{0} \\
\hline
\multicolumn{1}{|c|}{0} &
\multicolumn{1}{c|}{PI} &
\multicolumn{1}{c|}{PO} &
\multicolumn{1}{c|}{PR} &
\multicolumn{1}{c|}{PW} &
\multicolumn{1}{|c|}{SI} &
\multicolumn{1}{c|}{SO} &
\multicolumn{1}{c|}{SR} &
\multicolumn{1}{c|}{SW} &
\multicolumn{1}{c|}{rs1} &
\multicolumn{1}{c|}{funct3} &
\multicolumn{1}{c|}{rd} &
\multicolumn{1}{c|}{opcode} \\
\hline
4 & 1 & 1 & 1 & 1 & 1 & 1 & 1 & 1 & 5 & 3 & 5 & 7 \\
0 & \multicolumn{4}{c}{predecessor} & \multicolumn{4}{c}{successor} & 0 & FENCE & 0 & MISC-MEM \\
\end{tabular}
\end{center}

The FENCE instruction is used to order device I/O and
memory accesses as viewed by other RISC-V harts and external devices
or coprocessors.  Any combination of device input (I), device output
(O), memory reads (R), and memory writes (W) may be ordered with
respect to any combination of the same.  Informally, no other RISC-V
hart or external device can observe any operation in the {\em
  successor} set following a FENCE before any operation in the {\em
  predecessor} set preceding the FENCE.  The execution environment
will define what I/O operations are possible, and in particular, which
load and store instructions might be treated and ordered as device
input and device output operations respectively rather than memory
reads and writes.  For example, memory-mapped I/O devices will
typically be accessed with uncached loads and stores that are ordered
using the I and O bits rather than the R and W bits.  Instruction-set
extensions might also describe new coprocessor I/O instructions that
will also be ordered using the I and O bits in a FENCE.

The unused fields in the FENCE instruction, {\em imm[11:8]}, {\em rs1}, and
{\em rd}, are reserved for finer-grain fences in future extensions.  For
forward compatibility, base implementations shall ignore these fields, and
standard software shall zero these fields.

\begin{commentary}
We chose a relaxed memory model to allow high performance from simple
machine implementations, however a completely relaxed memory model is
too weak to support programming language memory models and so the
memory model is being tightened.

A relaxed memory model is also most compatible with likely future
coprocessor or accelerator extensions.  We separate out I/O ordering
from memory R/W ordering to avoid unnecessary serialization within a
device-driver hart and also to support alternative non-memory paths
to control added coprocessors or I/O devices.  Simple implementations
may additionally ignore the {\em predecessor} and {\em successor}
fields and always execute a conservative fence on all operations.
\end{commentary}

\vspace{-0.4in}
\begin{center}
\begin{tabular}{M@{}R@{}S@{}R@{}O}
\\
\instbitrange{31}{20} &
\instbitrange{19}{15} &
\instbitrange{14}{12} &
\instbitrange{11}{7} &
\instbitrange{6}{0} \\
\hline
\multicolumn{1}{|c|}{imm[11:0]} &
\multicolumn{1}{c|}{rs1} &
\multicolumn{1}{c|}{funct3} &
\multicolumn{1}{c|}{rd} &
\multicolumn{1}{c|}{opcode} \\
\hline
12 & 5 & 3 & 5 & 7 \\
0 & 0 & FENCE.I & 0 & MISC-MEM \\
\end{tabular}
\end{center}

The FENCE.I instruction is used to synchronize the instruction and
data streams.  RISC-V does not guarantee that stores to instruction
memory will be made visible to instruction fetches on the same RISC-V
hart until a FENCE.I instruction is executed.  A FENCE.I instruction
only ensures that a subsequent instruction fetch on a RISC-V hart
will see any previous data stores already visible to the same RISC-V
hart.  FENCE.I does {\em not} ensure that other RISC-V harts'
instruction fetches will observe the local hart's stores in a
multiprocessor system. To make a store to instruction memory visible
to all RISC-V harts, the writing hart has to execute a data FENCE
before requesting that all remote RISC-V harts execute a FENCE.I.

The unused fields in the FENCE.I instruction, {\em imm[11:0]}, {\em rs1}, and
{\em rd}, are reserved for finer-grain fences in future extensions.  For
forward compatibility, base implementations shall ignore these fields, and
standard software shall zero these fields.

\begin{commentary}
Because FENCE.I only orders stores with a hart's own instruction fetches,
application code should only rely upon FENCE.I if the application thread will
not be migrated to a different hart.  The ABI will provide mechanisms for
multiprocessor instruction-stream synchronization.
\end{commentary}

\begin{commentary}
The FENCE.I instruction was designed to support a wide variety of
implementations.  A simple implementation can flush the local
instruction cache and the instruction pipeline when the FENCE.I is
executed.  A more complex implementation might snoop the instruction
(data) cache on every data (instruction) cache miss, or use an
inclusive unified private L2 cache to invalidate lines from the
primary instruction cache when they are being written by a local store
instruction.  If instruction and data caches are kept coherent in this
way, then only the pipeline needs to be flushed at a FENCE.I.

We considered but did not include a ``store instruction word''
instruction (as in MAJC~\cite{majc}).  JIT compilers may generate a
large trace of instructions before a single FENCE.I, and amortize any
instruction cache snooping/invalidation overhead by writing translated
instructions to memory regions that are known not to reside in the
I-cache.
\end{commentary}

\section{Control and Status Register Instructions}
\label{sec:csrinsts}

SYSTEM instructions are used to access system functionality that might
require privileged access and are encoded using the I-type instruction
format.  These can be divided into two main classes: those that
atomically read-modify-write control and status registers (CSRs), and
all other potentially privileged instructions. CSR instructions are
described in this section, with the two other user-level SYSTEM
instructions described in the following section.

\begin{commentary}
The SYSTEM instructions are defined to allow simpler implementations
to always trap to a single software trap handler.  More sophisticated
implementations might execute more of each system instruction in
hardware.
\end{commentary}

\subsubsection*{CSR Instructions}

We define the full set of CSR instructions here, although in the standard
user-level base ISA, only a handful of read-only counter CSRs are accessible.

\vspace{-0.2in}
\begin{center}
\begin{tabular}{M@{}R@{}F@{}R@{}S}
\\
\instbitrange{31}{20} &
\instbitrange{19}{15} &
\instbitrange{14}{12} &
\instbitrange{11}{7} &
\instbitrange{6}{0} \\
\hline
\multicolumn{1}{|c|}{csr} &
\multicolumn{1}{c|}{rs1} &
\multicolumn{1}{c|}{funct3} &
\multicolumn{1}{c|}{rd} &
\multicolumn{1}{c|}{opcode} \\
\hline
12 & 5 & 3 & 5 & 7 \\
source/dest  & source & CSRRW  & dest & SYSTEM \\
source/dest  & source & CSRRS  & dest & SYSTEM \\
source/dest  & source & CSRRC  & dest & SYSTEM \\
source/dest  & uimm[4:0]   & CSRRWI & dest & SYSTEM \\
source/dest  & uimm[4:0]   & CSRRSI & dest & SYSTEM \\
source/dest  & uimm[4:0]   & CSRRCI & dest & SYSTEM \\
\end{tabular}
\end{center}

The CSRRW (Atomic Read/Write CSR) instruction atomically swaps values
in the CSRs and integer registers. CSRRW reads the old value of the
CSR, zero-extends the value to XLEN bits, then writes it to integer
register {\em rd}.  The initial value in {\em rs1} is written to the
CSR.  If {\em rd}={\tt x0}, then the instruction shall not read the CSR
and shall not cause any of the side-effects that might occur on a CSR
read.

The CSRRS (Atomic Read and Set Bits in CSR) instruction reads the
value of the CSR, zero-extends the value to XLEN bits, and writes it
to integer register {\em rd}.  The initial value in integer register
{\em rs1} is treated as a bit mask that specifies bit positions to be
set in the CSR.  Any bit that is high in {\em rs1} will cause the
corresponding bit to be set in the CSR, if that CSR bit is writable.
Other bits in the CSR are unaffected (though CSRs might have side
effects when written).

The CSRRC (Atomic Read and Clear Bits in CSR) instruction reads the
value of the CSR, zero-extends the value to XLEN bits, and writes it
to integer register {\em rd}.  The initial value in integer register
{\em rs1} is treated as a bit mask that specifies bit positions to be
cleared in the CSR.  Any bit that is high in {\em rs1} will cause the
corresponding bit to be cleared in the CSR, if that CSR bit is
writable.  Other bits in the CSR are unaffected.

For both CSRRS and CSRRC, if {\em rs1}={\tt x0}, then the instruction
will not write to the CSR at all, and so shall not cause any of the
side effects that might otherwise occur on a CSR write, such as
raising illegal instruction exceptions on accesses to read-only CSRs.
Note that if {\em rs1} specifies a register holding a zero value other
than {\tt x0}, the instruction will still attempt to write the
unmodified value back to the CSR and will cause any attendant side effects.

The CSRRWI, CSRRSI, and CSRRCI variants are similar to CSRRW, CSRRS,
and CSRRC respectively, except they update the CSR using an XLEN-bit
value obtained by zero-extending a 5-bit unsigned immediate (uimm[4:0]) field
encoded in the {\em rs1} field instead of a value from an integer
register.  For CSRRSI and CSRRCI, if the uimm[4:0] field is zero, then
these instructions will not write to the CSR, and shall not cause any
of the side effects that might otherwise occur on a CSR write.  For
CSRRWI, if {\em rd}={\tt x0}, then the instruction shall not read the
CSR and shall not cause any of the side-effects that might occur on a
CSR read.

Some CSRs, such as the instructions retired counter, {\tt instret}, may be
modified as side effects of instruction execution.  In these cases, if a CSR
access instruction reads a CSR, it reads the value prior to the execution of
the instruction.  If a CSR access instruction writes a CSR, the update occurs
after the execution of the instruction.  In particular, a value written to
{\tt instret} by one instruction will be the value read by the following
instruction (i.e., the increment of {\tt instret} caused by the first
instruction retiring happens before the write of the new value).

The assembler pseudo-instruction to read a CSR, CSRR {\em rd, csr}, is
encoded as CSRRS {\em rd, csr, x0}.  The assembler pseudo-instruction
to write a CSR, CSRW {\em csr, rs1}, is encoded as CSRRW {\em x0, csr,
  rs1}, while CSRWI {\em csr, uimm}, is encoded as CSRRWI {\em x0,
  csr, uimm}.

Further assembler pseudo-instructions are defined to set and clear
bits in the CSR when the old value is not required: CSRS/CSRC {\em
  csr, rs1}; CSRSI/CSRCI {\em csr, uimm}.

\subsubsection*{Timers and Counters}

\vspace{-0.2in}
\begin{center}
\begin{tabular}{M@{}R@{}F@{}R@{}S}
\\
\instbitrange{31}{20} &
\instbitrange{19}{15} &
\instbitrange{14}{12} &
\instbitrange{11}{7} &
\instbitrange{6}{0} \\
\hline
\multicolumn{1}{|c|}{csr} &
\multicolumn{1}{c|}{rs1} &
\multicolumn{1}{c|}{funct3} &
\multicolumn{1}{c|}{rd} &
\multicolumn{1}{c|}{opcode} \\
\hline
12 & 5 & 3 & 5 & 7 \\
RDCYCLE[H]   & 0 & CSRRS  & dest & SYSTEM \\
RDTIME[H]    & 0 & CSRRS  & dest & SYSTEM \\
RDINSTRET[H] & 0 & CSRRS  & dest & SYSTEM \\
\end{tabular}
\end{center}

RV32I provides a number of 64-bit read-only user-level counters, which
are mapped into the 12-bit CSR address space and accessed in 32-bit
pieces using CSRRS instructions.

The RDCYCLE pseudo-instruction reads the low XLEN bits of the {\tt
  cycle} CSR which holds a count of the number of clock cycles
executed by the processor core on which the hart is running from
an arbitrary start time in the past.  RDCYCLEH is
an RV32I-only instruction that reads bits 63--32 of the same cycle
counter.  The underlying 64-bit counter should never overflow in
practice.  The rate at which the cycle counter advances will depend on
the implementation and operating environment.  The execution
environment should provide a means to determine the current rate
(cycles/second) at which the cycle counter is incrementing.

The RDTIME pseudo-instruction reads the low XLEN bits of the {\tt
  time} CSR, which counts wall-clock real time that has passed from an
arbitrary start time in the past.  RDTIMEH is an RV32I-only instruction
that reads bits 63--32 of the same real-time counter.  The underlying 64-bit
counter should never overflow in practice.  The execution environment
should provide a means of determining the period of the real-time
counter (seconds/tick).  The period must be constant.  The
real-time clocks of all harts in a single user application
should be synchronized to within one tick of the real-time clock.  The
environment should provide a means to determine the accuracy of the
clock.

The RDINSTRET pseudo-instruction reads the low XLEN bits of the {\tt
  instret} CSR, which counts the number of instructions retired by
this hart from some arbitrary start point in the past.  RDINSTRETH is
an RV32I-only instruction that reads bits 63--32 of the same
instruction counter. The underlying 64-bit counter that should never
overflow in practice.

The following code sequence will read a valid 64-bit cycle counter value into
{\tt x3}:{\tt x2}, even if the counter overflows between reading its upper
and lower halves.

\begin{figure}[h!]
\begin{center}
\begin{verbatim}
    again:
        rdcycleh     x3
        rdcycle      x2
        rdcycleh     x4
        bne          x3, x4, again
\end{verbatim}
\end{center}
\caption{Sample code for reading the 64-bit cycle counter in RV32.}
\label{critical}
\end{figure}

\begin{commentary}
We mandate these basic counters be provided in all implementations as
they are essential for basic performance analysis, adaptive and
dynamic optimization, and to allow an application to work with
real-time streams.  Additional counters should be provided to help
diagnose performance problems and these should be made accessible from
user-level application code with low overhead.

We required the counters be 64 bits wide, even on RV32, as otherwise
it is very difficult for software to determine if values have
overflowed.  For a low-end implementation, the upper 32 bits of each
counter can be implemented using software counters incremented by a
trap handler triggered by overflow of the lower 32 bits.  The sample
code described above shows how the full 64-bit width value can be
safely read using the individual 32-bit instructions.

In some applications, it is important to be able to read multiple
counters at the same instant in time.  When run under a multitasking
environment, a user thread can suffer a context switch while
attempting to read the counters.  One solution is for the user thread
to read the real-time counter before and after reading the other
counters to determine if a context switch occurred in the middle of the
sequence, in which case the reads can be retried.  We considered
adding output latches to allow a user thread to snapshot the counter
values atomically, but this would increase the size of the user
context, especially for implementations with a richer set of counters.
\end{commentary}


\section{Environment Call and Breakpoints}

\vspace{-0.2in}
\begin{center}
\begin{tabular}{M@{}R@{}F@{}R@{}S}
\\
\instbitrange{31}{20} &
\instbitrange{19}{15} &
\instbitrange{14}{12} &
\instbitrange{11}{7} &
\instbitrange{6}{0} \\
\hline
\multicolumn{1}{|c|}{funct12} &
\multicolumn{1}{c|}{rs1} &
\multicolumn{1}{c|}{funct3} &
\multicolumn{1}{c|}{rd} &
\multicolumn{1}{c|}{opcode} \\
\hline
12 & 5 & 3 & 5 & 7 \\
ECALL   & 0 & PRIV & 0 & SYSTEM \\
EBREAK  & 0 & PRIV & 0 & SYSTEM \\
\end{tabular}
\end{center}

The ECALL instruction is used to make a request to the supporting
execution environment, which is usually an operating system.  The ABI
for the system will define how parameters for the environment request
are passed, but usually these will be in defined locations in the
integer register file.

The EBREAK instruction is used by debuggers to cause control to be
transferred back to a debugging environment.

\begin{commentary}
ECALL and EBREAK were previously named SCALL and SBREAK.  The
instructions have the same functionality and encoding, but were
renamed to reflect that they can be used more generally than to call a
supervisor-level operating system or debugger.
\end{commentary}

