\chapter{Preface}

This is {\bf a draft of} version 1.10 of the RISC-V privileged
architecture proposal.  Changes from version 1.9.1 include:

\begin{itemize}
  \parskip 0pt
  \itemsep 1pt
\item The previous version of this document was released under a
  Creative Commons Attribution 4.0 International Licence by the
  original authors, and this and future versions of this document will
  be released under the same licence.
\item The explicit convention on shadow CSR addresses has been removed
  to reclaim CSR space.  Shadow CSRs can still be added as needed.
\item The {\tt mvendorid} register now contains the JEDEC code of the
  core provider as opposed to a code supplied by the Foundation.  This
  avoids redundancy and offloads work from the Foundation.
\item The interrupt-enable stack discipline has been simplified.
\item An optional mechanism to change the base ISA used by supervisor
  and user modes has been added to the {\tt mstatus} CSR, and the
  field previously called Base in {\tt misa} has been renamed to {\tt
    MXL} for consistency.
\item The machine-mode base-and-bounds translation and protection
  schemes have been removed from the specification as part of moving
  the virtual memory configuration to {\tt sptbr}.  Some of the
  motivation for the base and bound schemes are now covered by the PMP
  registers, but space remains available in {\tt mstatus} to add these
  back at a later date if deemed useful.
\item In systems with only M-mode, or with both M-mode and U-mode but
  without U-mode trap support, the {\tt medeleg} and {\tt mideleg}
    registers now do not exist, whereas previously they returned zero.
\item The supervisor virtual memory configuration has been moved from the
  {\tt mstatus} register to the {\tt sptbr} register.
\item The SFENCE.VM instruction has been removed in favor of the improved
  SFENCE.VMA instruction.
\item The {\tt mstatus} bit MXR has been exposed to S-mode via {\tt sstatus}.
\item The polarity of the PUM bit in {\tt sstatus} has been inverted to
  shorten code sequences involving MXR.  The bit has been renamed to SUM.
\item Hardware management of page-table entry Accessed and Dirty bits has
  been made optional; simpler implementations may trap to software to
  set them.
\item The counter-enable scheme has changed, so that S-mode can
  control availability of counters to U-mode.
\item H-mode has been removed, as we are focusing on recursive
  virtualization support in S-mode.  The encoding space has been
  reserved and may be repurposed at a later date.
\item A mechanism to improve virtualization performance by
  trapping S-mode virtual-memory management operations has been added.
\end{itemize}

\newpage

\section*{Preface to Version 1.9.1}

This is version 1.9.1 of the RISC-V privileged architecture
proposal.  Changes from version 1.9 include:

\begin{itemize}
  \parskip 0pt
  \itemsep 1pt
\item Numerous additions and improvements to the commentary sections.
\item Change configuration string proposal to be use a search process
  that supports various formats including Device Tree String and
  flattened Device Tree.
\item Made {\tt misa} optionally writable to support modifying base
  and supported ISA extensions.  CSR address of {\tt misa} changed.
\item Added description of debug mode and debug CSRs.
\item Added a hardware performance monitoring scheme.  Simplified the
  handling of existing hardware counters, removing privileged versions
  of the counters and the corresponding delta registers.
\item Fixed description of SPIE in presence of user-level interrupts.
\end{itemize}
