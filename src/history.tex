\chapter{History and Acknowledgments}
\label{history}

\section{History from Revision 1.0 of ISA manual}

The RISC-V ISA and instruction set manual builds up several earlier
projects.  Several aspects of the supervisor-level machine and the
overall format of the manual date back to the T0 (Torrent-0) vector
microprocessor project at UC Berkeley and ICSI, begun in 1992.  T0 was
a vector processor based on the MIPS-II ISA, with Krste Asanovi\'{c}
as main architect and RTL designer, and Brian Kingsbury and Bertrand
Irrisou as principal VLSI implementors.  David Johnson at ICSI was a
major contributor to the T0 ISA design, particularly supervisor mode,
and to the manual text.  John Hauser also provided considerable
feedback on the T0 ISA design.

The Scale (Software-Controlled Architecture for Low Energy) project at
MIT, begun in 2000, built upon the T0 project infrastructure, refined
the supervisor-level interface, and moved away from the MIPS scalar
ISA by dropping the branch delay slot.  Ronny Krashinsky and
Christopher Batten were the principal architects of the Scale
Vector-Thread processor at MIT, while Mark Hampton ported the
GCC-based compiler infrastructure and tools for Scale.

A lightly edited version of the T0 MIPS scalar processor specification
(MIPS-6371) was used in teaching a new version of the MIT 6.371
Introduction to VLSI Systems class in the Fall 2002 semester, with
Chris Terman and Krste Asanovi\'{c} as lecturers.  Chris Terman
contributed most of the lab material for the class (there was no
TA!). The 6.371 class evolved into the trial 6.884 Complex Digital
Design class at MIT, taught by Arvind and Krste Asanovi\'{c} in Spring
2005, which became a regular Spring class 6.375.  A reduced version of
the Scale MIPS-based scalar ISA, named SMIPS, was used in 6.884/6.375.
Christopher Batten was the TA for the early offerings of these classes
and developed a considerable amount of documentation and lab material
based around the SMIPS ISA.  This same SMIPS lab material was adapted
and enhanced by TA Yunsup Lee for the UC Berkeley Fall 2009 CS250 VLSI
Systems Design class taught by John Wawrzynek, Krste Asanovi\'{c}, and
John Lazzaro.

The Maven (Malleable Array of Vector-thread ENgines) project was a
second-generation vector-thread architecture.  Its design was led by
Christopher Batten when he was an Exchange Scholar at UC Berkeley starting
in summer 2007.  Hidetaka Aoki, a visiting industrial fellow from
Hitachi, gave considerable feedback on the early Maven ISA and
microarchitecture design.  The Maven infrastructure was based on the
Scale infrastructure but the Maven ISA moved further away from the
MIPS ISA variant defined in Scale, with a unified floating-point and
integer register file.  Maven was designed to support experimentation
with alternative data-parallel accelerators.  Yunsup Lee was the main
implementor of the various Maven vector units, while Rimas Avi\v{z}ienis
was the main implementor of the various Maven scalar units.
Yunsup Lee and Christopher Batten ported GCC to work with the new
Maven ISA.  Christopher Celio provided the initial definition of a
traditional vector instruction set (``Flood'') variant of Maven.

Based on experience with all these previous projects, the RISC-V ISA
definition was begun in Summer 2010.  An initial version of the RISC-V
32-bit instruction subset was used in the UC Berkeley Fall 2010 CS250
VLSI Systems Design class, with Yunsup Lee as TA.  RISC-V is a clean
break from the earlier MIPS-inspired designs.  John Hauser contributed
to the floating-point ISA definition, including the sign-injection
instructions and a register encoding scheme that permits
internal recoding of floating-point values.

\section{History from Revision 2.0 of ISA manual}

Multiple implementations of RISC-V processors have been completed,
including several silicon fabrications, as shown in
Figure~\ref{silicon}.

\begin{table*}[!h]
\begin{center}
\begin{tabular}{|l|r|l|l|}
\hline
\multicolumn{1}{|c|}{Name} & \multicolumn{1}{|c|}{Tapeout Date} & \multicolumn{1}{|c|}{Process} & \multicolumn{1}{|c|}{ISA} \\ \hline
\hline
Raven-1 & May 29, 2011 & ST 28nm FDSOI & RV64G1\_Xhwacha1 \\ \hline
EOS14 & April 1, 2012 & IBM 45nm SOI & RV64G1p1\_Xhwacha2 \\ \hline
EOS16 & August 17, 2012 & IBM 45nm SOI & RV64G1p1\_Xhwacha2 \\ \hline
Raven-2 & August 22, 2012 & ST 28nm FDSOI & RV64G1p1\_Xhwacha2 \\ \hline
EOS18 & February 6, 2013 & IBM 45nm SOI & RV64G1p1\_Xhwacha2 \\ \hline
EOS20 & July 3, 2013 & IBM 45nm SOI & RV64G1p99\_Xhwacha2 \\ \hline
Raven-3 & September 26, 2013 & ST 28nm SOI & RV64G1p99\_Xhwacha2 \\ \hline
EOS22 & March 7, 2014 & IBM 45nm SOI & RV64G1p9999\_Xhwacha3 \\ \hline
\end{tabular}
\end{center}
\vspace{-0.15in}
\caption{Fabricated RISC-V testchips.}
\label{silicon}
\end{table*}

The first RISC-V processors to be fabricated were written in Verilog and
manufactured in a pre-production \wunits{28}{nm} FDSOI technology from
ST as the Raven-1 testchip in 2011.  Two cores were developed by Yunsup
Lee and Andrew Waterman, advised by Krste Asanovi\'{c}, and fabricated
together: 1) an RV64 scalar core with error-detecting flip-flops, and 2)
an RV64 core with an attached 64-bit floating-point vector unit.  The
first microarchitecture was informally known as ``TrainWreck'', due to
the short time available to complete the design with immature design
libraries.

Subsequently, a clean microarchitecture for an in-order decoupled RV64
core was developed by Andrew Waterman, Rimas Avi\v{z}ienis, and Yunsup
Lee, advised by Krste Asanovi\'{c}, and, continuing the railway theme,
was codenamed ``Rocket'' after George Stephenson's successful steam
locomotive design.  Rocket was written in Chisel, a new hardware
design language developed at UC Berkeley.  The IEEE floating-point
units used in Rocket were developed by John Hauser, Andrew
Waterman, and Brian Richards.
Rocket has since been refined and developed further, and has been
fabricated two more times in \wunits{28}{nm} FDSOI (Raven-2, Raven-3),
and five times in IBM \wunits{45}{nm} SOI technology (EOS14, EOS16,
EOS18, EOS20, EOS22) for a photonics project.  Work is ongoing to make
the Rocket design available as a parameterized RISC-V processor
generator.

EOS14--EOS22 chips include early versions of Hwacha, a 64-bit IEEE
floating-point vector unit, developed by Yunsup Lee, Andrew Waterman,
Huy Vo, Albert Ou, Quan Nguyen, and Stephen Twigg, advised by Krste
Asanovi\'{c}.  EOS16--EOS22 chips include dual cores with a
cache-coherence protocol developed by Henry Cook and Andrew Waterman,
advised by Krste Asanovi\'{c}.  EOS14 silicon has successfully run at
\wunits{1.25}{GHz}. EOS16 silicon suffered from a bug in the IBM pad
libraries.  EOS18 and EOS20 have successfully run at \wunits{1.35}{GHz}.

Contributors to the Raven testchips include Yunsup Lee, Andrew Waterman,
Rimas Avi\v{z}ienis, Brian Zimmer, Jaehwa Kwak, Ruzica Jevti\'{c},
Milovan Blagojevi\'{c}, Alberto Puggelli, Steven Bailey, Ben Keller,
Pi-Feng Chiu, Brian Richards, Borivoje Nikoli\'{c}, and Krste
Asanovi\'{c}.

Contributors to the EOS testchips include Yunsup Lee, Rimas
Avi\v{z}ienis, Andrew Waterman, Henry Cook, Huy Vo, Daiwei Li, Chen Sun,
Albert Ou, Quan Nguyen, Stephen Twigg, Vladimir Stojanovi\'{c}, and
Krste Asanovi\'{c}.

Andrew Waterman and Yunsup Lee developed the C++ ISA simulator
``Spike'', used as a golden model in development and named after the
golden spike used to celebrate completion of the US transcontinental
railway.  Spike has been made available as a BSD open-source project.

Andrew Waterman completed a Master's thesis with a preliminary design
of the RISC-V compressed instruction set~\cite{waterman-ms}. 

Various FPGA implementations of the RISC-V have been completed,
primarily as part of integrated demos for the Par Lab project research
retreats.  The largest FPGA design has 3 cache-coherent RV64IMA
processors running a research operating system.  Contributors to the
FPGA implementations include Andrew Waterman, Yunsup Lee, Rimas
Avi\v{z}ienis, and Krste Asanovi\'{c}.

RISC-V processors have been used in several classes at UC Berkeley.
Rocket was used in the Fall 2011 offering of CS250 as a basis for class
projects, with Brian Zimmer as TA.  For the undergraduate CS152 class in
Spring 2012, Christopher Celio used Chisel to write a suite of educational
RV32 processors, named ``Sodor'' after the island on which ``Thomas the
Tank Engine'' and friends live.  The suite includes a microcoded core,
an unpipelined core, and 2, 3, and 5-stage pipelined cores, and is
publicly available under a BSD license.  The suite was subsequently
updated and used again in CS152 in Spring 2013, with Yunsup Lee as TA,
and in Spring 2014, with Eric Love as TA.
Christopher Celio also developed an out-of-order RV64 design known as BOOM
(Berkeley Out-of-Order Machine), with accompanying pipeline
visualizations, that was used in the CS152 classes.  The CS152 classes
also used cache-coherent versions of the Rocket core developed by Andrew
Waterman and Henry Cook.

Over the summer of 2013, the RoCC (Rocket Custom Coprocessor)
interface was defined to simplify adding custom accelerators to the
Rocket core.  Rocket and the RoCC interface were used extensively in
the Fall 2013 CS250 VLSI class taught by Jonathan Bachrach, with
several student accelerator projects built to the RoCC interface.  The
Hwacha vector unit has been rewritten as a RoCC coprocessor.

Two Berkeley undergraduates, Quan Nguyen and Albert Ou, have
successfully ported Linux to run on RISC-V in Spring 2013.

Colin Schmidt successfully completed an LLVM backend for RISC-V 2.0 in
January 2014.

Darius Rad at Bluespec contributed soft-float ABI support to the GCC port in
March 2014.

John Hauser contributed the definition of the floating-point classification
instructions.

We are aware of several other RISC-V core implementations, including
one in Verilog by Tommy Thorn, and one in Bluespec by Rishiyur Nikhil.

\section*{Acknowledgments}

Thanks to Christopher F. Batten, Preston Briggs, Christopher Celio, David
Chisnall, Stefan Freudenberger, John Hauser, Ben Keller, Rishiyur
Nikhil, Michael Taylor, Tommy Thorn, and Robert Watson for comments on
the draft ISA version 2.0 specification.

\section{History for Revision 2.1}

Uptake of the RISC-V ISA has been very rapid since the introduction of
the frozen version 2.0 in May 2014, with too much activity to record
in a short history section such as this.  Perhaps the most important
single event was the formation of the non-profit RISC-V Foundation in
August 2015. The Foundation will now take over stewardship of the
official RISC-V ISA standard, and the official website {\tt riscv.org}
is the best place to obtain news and updates on the RISC-V standard.

\section*{Acknowledgments}

Thanks to Scott Beamer, Allen J. Baum, Christopher Celio, David Chisnall,
Paul Clayton, Palmer Dabbelt, Jan Gray, Michael Hamburg, and John
Hauser for comments on the version 2.0 specification.

\section{History for Revision 2.2}


\section*{Acknowledgments}

Thanks to Alex Bradbury, David Horner, and Joseph Myers for comments on the
version 2.1 specification.

\section{Funding}

Development of the RISC-V architecture and implementations has been
partially funded by the following sponsors.
\begin{itemize}

\item {\bf Par Lab:} Research supported by Microsoft (Award \#024263) and Intel (Award
    \#024894) funding and by matching funding by U.C. Discovery
    (Award \#DIG07-10227). Additional support came from Par Lab
    affiliates Nokia, NVIDIA, Oracle, and Samsung.

\item {\bf Project Isis:} DoE Award DE-SC0003624.

\item {\bf ASPIRE Lab}: DARPA PERFECT program, Award
    HR0011-12-2-0016.  DARPA POEM program Award HR0011-11-C-0100.  The
    Center for Future Architectures Research (C-FAR), a STARnet center
    funded by the Semiconductor Research Corporation.  Additional
    support from ASPIRE industrial sponsor, Intel, and ASPIRE
    affiliates, Google, Hewlett Packard Enterprise, Huawei, Nokia,
    NVIDIA, Oracle, and Samsung.

\end{itemize}

The content of this paper does not necessarily reflect the position or the
policy of the US government and no official endorsement should be
inferred. 
