\chapter{``Zicsr'', Control and Status Register (CSR) Instructions, Version 2.0}
\label{csrinsts}

RISC-V defines a separate address space of 4096 Control and Status
registers associated with each hart.  This chapter defines the full
set of CSR instructions that operate on these CSRs.

\begin{commentary}
  While CSRs are primarily used by the privileged architecture, there
  are several uses in unprivileged code including for counters and
  timers, and for floating-point status.

  The counters and timers are no longer considered mandatory parts of
  the standard base ISAs, and so the CSR instructions required to
  access them have been moved out of Chapter~\ref{rv32} into this
  separate chapter.
\end{commentary}

\section{CSR Instructions}

All CSR instructions atomically read-modify-write a single CSR, whose
CSR specifier is encoded in the 12-bit {\em csr} field of the
instruction held in bits 31--20.  The immediate forms use a 5-bit
zero-extended immediate encoded in the {\em rs1} field.

\vspace{-0.2in}
\begin{center}
\begin{tabular}{M@{}R@{}F@{}R@{}S}
\\
\instbitrange{31}{20} &
\instbitrange{19}{15} &
\instbitrange{14}{12} &
\instbitrange{11}{7} &
\instbitrange{6}{0} \\
\hline
\multicolumn{1}{|c|}{csr} &
\multicolumn{1}{c|}{rs1} &
\multicolumn{1}{c|}{funct3} &
\multicolumn{1}{c|}{rd} &
\multicolumn{1}{c|}{opcode} \\
\hline
12 & 5 & 3 & 5 & 7 \\
source/dest  & source & CSRRW  & dest & SYSTEM \\
source/dest  & source & CSRRS  & dest & SYSTEM \\
source/dest  & source & CSRRC  & dest & SYSTEM \\
source/dest  & uimm[4:0]   & CSRRWI & dest & SYSTEM \\
source/dest  & uimm[4:0]   & CSRRSI & dest & SYSTEM \\
source/dest  & uimm[4:0]   & CSRRCI & dest & SYSTEM \\
\end{tabular}
\end{center}

The CSRRW (Atomic Read/Write CSR) instruction atomically swaps values
in the CSRs and integer registers. CSRRW reads the old value of the
CSR, zero-extends the value to XLEN bits, then writes it to integer
register {\em rd}.  The initial value in {\em rs1} is written to the
CSR.  If {\em rd}={\tt x0}, then the instruction shall not read the CSR
and shall not cause any of the side effects that might occur on a CSR
read.

The CSRRS (Atomic Read and Set Bits in CSR) instruction reads the
value of the CSR, zero-extends the value to XLEN bits, and writes it
to integer register {\em rd}.  The initial value in integer register
{\em rs1} is treated as a bit mask that specifies bit positions to be
set in the CSR.  Any bit that is high in {\em rs1} will cause the
corresponding bit to be set in the CSR, if that CSR bit is writable.
Other bits in the CSR are not explicitly written.

The CSRRC (Atomic Read and Clear Bits in CSR) instruction reads the
value of the CSR, zero-extends the value to XLEN bits, and writes it
to integer register {\em rd}.  The initial value in integer register
{\em rs1} is treated as a bit mask that specifies bit positions to be
cleared in the CSR.  Any bit that is high in {\em rs1} will cause the
corresponding bit to be cleared in the CSR, if that CSR bit is writable.
Other bits in the CSR are not explicitly written.

For both CSRRS and CSRRC, if {\em rs1}={\tt x0}, then the instruction
will not write to the CSR at all, and so shall not cause any of the
side effects that might otherwise occur on a CSR write, nor
raise illegal instruction exceptions on accesses to read-only CSRs.
Both CSRRS and CSRRC always read the addressed CSR and cause any read
side effects regardless of {\em rs1} and {\em rd} fields.  Note that
if {\em rs1} specifies a register holding a zero value other than {\tt
  x0}, the instruction will still attempt to write the unmodified
value back to the CSR and will cause any attendant side effects.  A
CSRRW with {\em rs1}={\tt x0} will attempt to write zero to the
destination CSR.

The CSRRWI, CSRRSI, and CSRRCI variants are similar to CSRRW, CSRRS,
and CSRRC respectively, except they update the CSR using an XLEN-bit
value obtained by zero-extending a 5-bit unsigned immediate (uimm[4:0]) field
encoded in the {\em rs1} field instead of a value from an integer
register.  For CSRRSI and CSRRCI, if the uimm[4:0] field is zero, then
these instructions will not write to the CSR, and shall not cause any
of the side effects that might otherwise occur on a CSR write, nor raise
illegal instruction exceptions on accesses to read-only CSRs.
For CSRRWI, if {\em rd}={\tt x0}, then the instruction shall not read the
CSR and shall not cause any of the side effects that might occur on a
CSR read.  Both CSRRSI and CSRRCI will always read the CSR and cause
any read side effects regardless of {\em rd} and {\em rs1} fields.

\begin{table}
  \centering
  \begin{tabular}{|l|c|c|c|c|}
    \hline
    \multicolumn{5}{|c|}{Register operand} \\
    \hline
    Instruction & \textit{rd} is \texttt{x0}
                      & \textit{rs1} is \texttt{x0}
                            & Reads CSR & Writes CSR \\
    \hline
    CSRRW       & Yes & --  & No  & Yes \\
    CSRRW       & No  & --  & Yes & Yes \\
    CSRRS/CSRRC & --  & Yes & Yes & No \\
    CSRRS/CSRRC & --  & No  & Yes & Yes \\
    \hline
    \multicolumn{5}{|c|}{Immediate operand} \\
    \hline
    Instruction & \textit{rd} is \texttt{x0}
                        & \textit{uimm}$=$0
                              & Reads CSR & Writes CSR \\
    \hline
    CSRRWI        & Yes & --  & No  & Yes \\
    CSRRWI        & No  & --  & Yes & Yes \\
    CSRRSI/CSRRCI & --  & Yes & Yes & No \\
    CSRRSI/CSRRCI & --  & No  & Yes & Yes \\
    \hline
  \end{tabular}
  \caption{Conditions determining whether a CSR instruction reads or writes
    the specified CSR.}
  \label{tab:csrsideeffects}
\end{table}

Table~\ref{tab:csrsideeffects} summarizes the behavior of the CSR
instructions with respect to whether they read and/or write the CSR.

For any event or consequence that occurs due to a CSR having a particular
value, if a write to the CSR gives it that value, the resulting event or
consequence is said to be an \emph{indirect effect} of the write.
Indirect effects of a CSR write are not considered by the RISC-V ISA to
be side effects of that write.

\begin{commentary}
  An example of side effects for CSR accesses would be if reading from a
  specific CSR causes a light bulb to turn on, while writing an odd value
  to the same CSR causes the light to turn off.
  Assume writing an even value has no effect.
  In this case, both the read and write have side effects controlling
  whether the bulb is lit, as this condition is not determined solely
  from the CSR value.
  (Note that after writing an odd value to the CSR to turn off the light,
  then reading to turn the light on, writing again the same odd value
  causes the light to turn off again.
  Hence, on the last write, it is not a change in the CSR value that
  turns off the light.)

  On the other hand, if a bulb is rigged to light whenever the value
  of a particular CSR is odd, then turning the light on and off is not
  considered a side effect of writing to the CSR but merely an indirect
  effect of such writes.

  More concretely, the RISC-V privileged architecture defined in
  Volume~II specifies that certain combinations of CSR values cause a
  trap to occur.
  When an explicit write to a CSR creates the conditions that trigger the
  trap, the trap is not considered a side effect of the write but merely
  an indirect effect.

  Standard CSRs do not have any side effects on reads.
  Standard CSRs may have side effects on writes.
  Custom extensions might add CSRs for which accesses have side effects
  on either reads or writes.
\end{commentary}

Some CSRs, such as the instructions-retired counter, {\tt instret},
may be modified as side effects of instruction execution.  In these
cases, if a CSR access instruction reads a CSR, it reads the value
prior to the execution of the instruction.  If a CSR access
instruction writes such a CSR, the write is done instead of the
increment.  In particular, a value written to {\tt instret} by one
instruction will be the value read by the following instruction.

The assembler pseudoinstruction to read a CSR, CSRR {\em rd, csr}, is
encoded as CSRRS {\em rd, csr, x0}.  The assembler pseudoinstruction
to write a CSR, CSRW {\em csr, rs1}, is encoded as CSRRW {\em x0, csr,
  rs1}, while CSRWI {\em csr, uimm}, is encoded as CSRRWI {\em x0,
  csr, uimm}.

Further assembler pseudoinstructions are defined to set and clear
bits in the CSR when the old value is not required: CSRS/CSRC {\em
  csr, rs1}; CSRSI/CSRCI {\em csr, uimm}.


\subsection*{CSR Access Ordering}

Each RISC-V hart normally observes its own CSR accesses, including its
implicit CSR accesses, as performed in program order.
In particular, unless specified otherwise, a CSR access is performed
after the execution of any prior instructions in program order whose behavior
modifies or is modified by the CSR state and before the execution of any
subsequent instructions in program order whose behavior modifies or is modified
by the CSR state.
Furthermore, an explicit CSR read returns the
CSR state before the execution of the instruction, while an
explicit CSR write suppresses and overrides any implicit writes or
modifications to the same CSR by the same instruction.

Likewise, any side effects from an explicit CSR access are normally
observed to occur synchronously in program order.
Unless specified otherwise, the full consequences of any such side
effects are observable by the very next instruction, and no consequences
may be observed out-of-order by preceding instructions.
(Note the distinction made earlier between side effects and indirect
effects of CSR writes.)

For the RVWMO memory consistency model (Chapter~\ref{ch:memorymodel}),
CSR accesses are weakly ordered by default,
so other harts or devices may observe CSR accesses in an order
different from program order. In addition, CSR accesses are not ordered with
respect to explicit memory accesses, unless a CSR access modifies the execution
behavior of the instruction that performs the explicit memory access or unless
a CSR access and an explicit memory access are ordered by either the syntactic
dependencies defined by the memory model or the ordering requirements defined
by the Memory-Ordering PMAs section in Volume II of this manual. To enforce
ordering in all other cases, software should execute a FENCE instruction
between the relevant accesses. For the purposes of the FENCE instruction, CSR
read accesses are classified as device input (I), and CSR write accesses are
classified as device output (O).

\begin{commentary}
Informally, the CSR space acts as a weakly ordered memory-mapped I/O region, as
defined by the Memory-Ordering PMAs section in Volume II of this manual. As a
result, the order of CSR accesses with respect to all other accesses is
constrained by the same mechanisms that constrain the order of memory-mapped
I/O accesses to such a region.

These CSR-ordering constraints are imposed to support ordering main
memory and memory-mapped I/O accesses with respect to CSR accesses that
are visible to, or affected by, devices or other harts.
Examples include the {\tt time}, {\tt cycle}, and {\tt mcycle}
CSRs, in addition to CSRs that reflect pending interrupts, like {\tt mip} and
{\tt sip}.
Note that implicit reads of such CSRs (e.g., taking an interrupt because of
a change in {\tt mip}) are also ordered as device input.

Most CSRs (including, e.g., the {\tt fcsr}) are not visible to other harts;
their accesses can be freely reordered in the global memory order with respect
to FENCE instructions without violating this specification.
\end{commentary}

The hardware platform may define that accesses to certain CSRs are
strongly ordered, as defined by the Memory-Ordering PMAs section in Volume II
of this manual. Accesses to strongly ordered CSRs have stronger ordering
constraints with respect to accesses to both weakly ordered CSRs and accesses
to memory-mapped I/O regions.

\begin{commentary}
The rules for the reordering of CSR accesses in the global memory order
should probably be moved to Chapter~\ref{ch:memorymodel} concerning the
RVWMO memory consistency model.
\end{commentary}
