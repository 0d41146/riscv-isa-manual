\chapter{``F'' Standard Extension for Single-Precision Floating-Point,
Version 2.2}
\label{sec:single-float}

This chapter describes the standard instruction-set extension for
single-precision floating-point, which is named ``F'' and adds
single-precision floating-point computational instructions compliant
with the IEEE 754-2008 arithmetic standard~\cite{ieee754-2008}.
The F extension depends on the ``Zicsr'' extension for control
and status register access.

\section{F Register State}

The F extension adds 32 floating-point registers, {\tt f0}--{\tt f31},
each 32 bits wide, and a floating-point control and status register
{\tt fcsr}, which contains the operating mode and exception status of the
floating-point unit.  This additional state is shown in
Figure~\ref{fprs}.  We use the term FLEN to describe the width of the
floating-point registers in the RISC-V ISA, and FLEN=32 for the F
single-precision floating-point extension.  Most floating-point
instructions operate on values in the floating-point register file.
Floating-point load and store instructions transfer floating-point
values between registers and memory.  Instructions to transfer values
to and from the integer register file are also provided.

\begin{figure}[htbp]
{\footnotesize
\begin{center}
\begin{tabular}{p{2in}}
\instbitrange{FLEN-1}{0}                                    \\ \cline{1-1}
\multicolumn{1}{|c|}{\reglabel{\ \ \ \ f0\ \ \ \ \ }}       \\ \cline{1-1}
\multicolumn{1}{|c|}{\reglabel{\ \ \ \ f1\ \ \ \ \ }}       \\ \cline{1-1}
\multicolumn{1}{|c|}{\reglabel{\ \ \ \ f2\ \ \ \ \ }}       \\ \cline{1-1}
\multicolumn{1}{|c|}{\reglabel{\ \ \ \ f3\ \ \ \ \ }}       \\ \cline{1-1}
\multicolumn{1}{|c|}{\reglabel{\ \ \ \ f4\ \ \ \ \ }}       \\ \cline{1-1}
\multicolumn{1}{|c|}{\reglabel{\ \ \ \ f5\ \ \ \ \ }}       \\ \cline{1-1}
\multicolumn{1}{|c|}{\reglabel{\ \ \ \ f6\ \ \ \ \ }}       \\ \cline{1-1}
\multicolumn{1}{|c|}{\reglabel{\ \ \ \ f7\ \ \ \ \ }}       \\ \cline{1-1}
\multicolumn{1}{|c|}{\reglabel{\ \ \ \ f8\ \ \ \ \ }}       \\ \cline{1-1}
\multicolumn{1}{|c|}{\reglabel{\ \ \ \ f9\ \ \ \ \ }}       \\ \cline{1-1}
\multicolumn{1}{|c|}{\reglabel{\ \ \ f10\ \ \ \ \ }}        \\ \cline{1-1}
\multicolumn{1}{|c|}{\reglabel{\ \ \ f11\ \ \ \ \ }}        \\ \cline{1-1}
\multicolumn{1}{|c|}{\reglabel{\ \ \ f12\ \ \ \ \ }}        \\ \cline{1-1}
\multicolumn{1}{|c|}{\reglabel{\ \ \ f13\ \ \ \ \ }}        \\ \cline{1-1}
\multicolumn{1}{|c|}{\reglabel{\ \ \ f14\ \ \ \ \ }}        \\ \cline{1-1}
\multicolumn{1}{|c|}{\reglabel{\ \ \ f15\ \ \ \ \ }}        \\ \cline{1-1}
\multicolumn{1}{|c|}{\reglabel{\ \ \ f16\ \ \ \ \ }}        \\ \cline{1-1}
\multicolumn{1}{|c|}{\reglabel{\ \ \ f17\ \ \ \ \ }}        \\ \cline{1-1}
\multicolumn{1}{|c|}{\reglabel{\ \ \ f18\ \ \ \ \ }}        \\ \cline{1-1}
\multicolumn{1}{|c|}{\reglabel{\ \ \ f19\ \ \ \ \ }}        \\ \cline{1-1}
\multicolumn{1}{|c|}{\reglabel{\ \ \ f20\ \ \ \ \ }}        \\ \cline{1-1}
\multicolumn{1}{|c|}{\reglabel{\ \ \ f21\ \ \ \ \ }}        \\ \cline{1-1}
\multicolumn{1}{|c|}{\reglabel{\ \ \ f22\ \ \ \ \ }}        \\ \cline{1-1}
\multicolumn{1}{|c|}{\reglabel{\ \ \ f23\ \ \ \ \ }}        \\ \cline{1-1}
\multicolumn{1}{|c|}{\reglabel{\ \ \ f24\ \ \ \ \ }}        \\ \cline{1-1}
\multicolumn{1}{|c|}{\reglabel{\ \ \ f25\ \ \ \ \ }}        \\ \cline{1-1}
\multicolumn{1}{|c|}{\reglabel{\ \ \ f26\ \ \ \ \ }}        \\ \cline{1-1}
\multicolumn{1}{|c|}{\reglabel{\ \ \ f27\ \ \ \ \ }}        \\ \cline{1-1}
\multicolumn{1}{|c|}{\reglabel{\ \ \ f28\ \ \ \ \ }}        \\ \cline{1-1}
\multicolumn{1}{|c|}{\reglabel{\ \ \ f29\ \ \ \ \ }}        \\ \cline{1-1}
\multicolumn{1}{|c|}{\reglabel{\ \ \ f30\ \ \ \ \ }}        \\ \cline{1-1}
\multicolumn{1}{|c|}{\reglabel{\ \ \ f31\ \ \ \ \ }}        \\ \cline{1-1}
\multicolumn{1}{c}{FLEN}                                    \\

\instbitrange{31}{0}                                        \\ \cline{1-1}
\multicolumn{1}{|c|}{\reglabel{fcsr}}                       \\ \cline{1-1}
\multicolumn{1}{c}{32}                                      \\
\end{tabular}
\end{center}
}
\caption{RISC-V standard F extension single-precision floating-point state.}
\label{fprs}
\end{figure}

\begin{commentary}
We considered a unified register file for both integer and
floating-point values as this simplifies software register allocation
and calling conventions, and reduces total user state.  However, a
split organization increases the total number of registers accessible
with a given instruction width, simplifies provision of enough regfile
ports for wide superscalar issue, supports decoupled
floating-point-unit architectures, and simplifies use of internal
floating-point encoding techniques.  Compiler support and calling
conventions for split register file architectures are well understood,
and using dirty bits on floating-point register file state can reduce
context-switch overhead.
\end{commentary}

\clearpage

\section{Floating-Point Control and Status Register}

The floating-point control and status register, {\tt fcsr}, is a RISC-V
control and status register (CSR).  It is a 32-bit read/write register that
selects the dynamic rounding mode for floating-point arithmetic operations and
holds the accrued exception flags, as shown in Figure~\ref{fcsr}.

\begin{figure*}[h]
{\footnotesize
\begin{center}
\begin{tabular}{K@{}E@{}ccccc}
\instbitrange{31}{8} &
\instbitrange{7}{5} &
\instbit{4} &
\instbit{3} &
\instbit{2} &
\instbit{1} &
\instbit{0} \\
\hline
\multicolumn{1}{|c|}{{\em Reserved}} &
\multicolumn{1}{c|}{Rounding Mode ({\tt frm})} &
\multicolumn{5}{c|}{Accrued Exceptions ({\tt fflags})} \\
\hline
\multicolumn{1}{c}{} &
\multicolumn{1}{c|}{} &
\multicolumn{1}{c|}{NV} &
\multicolumn{1}{c|}{DZ} &
\multicolumn{1}{c|}{OF} &
\multicolumn{1}{c|}{UF} &
\multicolumn{1}{c|}{NX} \\
\cline{3-7}
24 & 3 & 1 & 1 & 1 & 1 & 1 \\
\end{tabular}
\end{center}
}
\vspace{-0.1in}
\caption{Floating-point control and status register.}
\label{fcsr}
\end{figure*}

The {\tt fcsr} register can be read and written with the FRCSR and
FSCSR instructions, which are assembler pseudoinstructions built on the
underlying CSR access instructions.  FRCSR reads {\tt fcsr} by copying
it into integer register {\em rd}.  FSCSR swaps the value in {\tt
  fcsr} by copying the original value into integer register {\em rd},
and then writing a new value obtained from integer register {\em rs1}
into {\tt fcsr}.

The fields within the {\tt fcsr} can also be accessed individually
through different CSR addresses, and separate assembler pseudoinstructions are
defined for these accesses.  The FRRM instruction reads the Rounding
Mode field {\tt frm} and copies it into the least-significant three
bits of integer register {\em rd}, with zero in all other bits.  FSRM
swaps the value in {\tt frm} by copying the original value into
integer register {\em rd}, and then writing a new value obtained from
the three least-significant bits of integer register {\em rs1} into
{\tt frm}.  FRFLAGS and FSFLAGS are defined analogously for the
Accrued Exception Flags field {\tt fflags}.

Bits 31--8 of the {\tt fcsr} are reserved for other standard extensions. If
these extensions are not present, implementations shall ignore writes to
these bits and supply a zero value when read.  Standard software should
preserve the contents of these bits.

Floating-point operations use either a static rounding mode encoded in
the instruction, or a dynamic rounding mode held in {\tt frm}.
Rounding modes are encoded as shown in Table~\ref{rm}.  A value of 111
in the instruction's {\em rm} field selects the dynamic rounding mode
held in {\tt frm}.  The behavior of floating-point instructions that
depend on rounding mode when executed with a reserved rounding mode is
{\em reserved}, including both static reserved rounding modes (101--110) and
dynamic reserved rounding modes (101--111).  Some instructions, including
widening conversions, have the {\em rm} field but are nevertheless
mathematically unaffected by the rounding mode; software should set their
{\em rm} field to RNE (000) but implementations must treat the {\em rm}
field as usual (in particular, with regard to decoding legal vs. reserved
encodings).

\begin{table}[htp]
\begin{small}
\begin{center}
\begin{tabular}{ccl}
\hline
\multicolumn{1}{|c|}{Rounding Mode} &
\multicolumn{1}{c|}{Mnemonic} &
\multicolumn{1}{c|}{Meaning} \\
\hline
\multicolumn{1}{|c|}{000} &
\multicolumn{1}{l|}{RNE} &
\multicolumn{1}{l|}{Round to Nearest, ties to Even}\\
\hline
\multicolumn{1}{|c|}{001} &
\multicolumn{1}{l|}{RTZ} &
\multicolumn{1}{l|}{Round towards Zero}\\
\hline
\multicolumn{1}{|c|}{010} &
\multicolumn{1}{l|}{RDN} &
\multicolumn{1}{l|}{Round Down (towards $-\infty$)}\\
\hline
\multicolumn{1}{|c|}{011} &
\multicolumn{1}{l|}{RUP} &
\multicolumn{1}{l|}{Round Up (towards $+\infty$)}\\
\hline
\multicolumn{1}{|c|}{100} &
\multicolumn{1}{l|}{RMM} &
\multicolumn{1}{l|}{Round to Nearest, ties to Max Magnitude}\\
\hline
\multicolumn{1}{|c|}{101} &
\multicolumn{1}{l|}{} &
\multicolumn{1}{l|}{\em Reserved for future use.}\\
\hline
\multicolumn{1}{|c|}{110} &
\multicolumn{1}{l|}{} &
\multicolumn{1}{l|}{\em Reserved for future use.}\\
\hline
\multicolumn{1}{|c|}{111} &
\multicolumn{1}{l|}{DYN} &
\multicolumn{1}{l|}{In instruction's {\em rm} field, selects dynamic rounding mode;}\\
\multicolumn{1}{|c|}{} &
\multicolumn{1}{l|}{} &
\multicolumn{1}{l|}{In Rounding Mode register, {\em reserved}.}\\
\hline
\end{tabular}
\end{center}
\end{small}
\caption{Rounding mode encoding.}
\label{rm}
\end{table}

\begin{commentary}
The C99 language standard effectively mandates the provision of a
dynamic rounding mode register.  In typical implementations, writes to
the dynamic rounding mode CSR state will serialize the pipeline.
Static rounding modes are used to implement specialized arithmetic
operations that often have to switch frequently between different
rounding modes.

The ratified version of the F spec mandated that an illegal
instruction exception was raised when an instruction was executed with
a reserved dynamic rounding mode.  This has been weakened to reserved,
which matches the behavior of static rounding-mode instructions.
Raising an illegal instruction exception is still valid behavior when
encountering a reserved encoding, so implementations compatible with
the ratified spec are compatible with the weakened spec.
\end{commentary}
 
The accrued exception flags indicate the exception conditions that
have arisen on any floating-point arithmetic instruction since the
field was last reset by software, as shown in Table~\ref{bitdef}.
The base RISC-V ISA
does not support generating a trap on the setting of a floating-point
exception flag.

\begin{table}[htp]
\begin{small}
\begin{center}
\begin{tabular}{cl}
\hline
\multicolumn{1}{|c|}{Flag Mnemonic} &
\multicolumn{1}{c|}{Flag Meaning} \\
\hline
\multicolumn{1}{|c|}{NV} &
\multicolumn{1}{c|}{Invalid Operation}\\
\hline
\multicolumn{1}{|c|}{DZ} &
\multicolumn{1}{c|}{Divide by Zero}\\
\hline
\multicolumn{1}{|c|}{OF} &
\multicolumn{1}{c|}{Overflow}\\
\hline
\multicolumn{1}{|c|}{UF} &
\multicolumn{1}{c|}{Underflow}\\
\hline
\multicolumn{1}{|c|}{NX} &
\multicolumn{1}{c|}{Inexact}\\
\hline
\end{tabular}
\end{center}
\end{small}
\caption{Accrued exception flag encoding.}
\label{bitdef}
\end{table}

\begin{commentary}
As allowed by the standard, we do not support traps on floating-point
exceptions in the F extension, but instead require explicit checks of the flags
in software.  We considered adding branches controlled directly by the
contents of the floating-point accrued exception flags, but ultimately chose
to omit these instructions to keep the ISA simple.
\end{commentary}

\section{NaN Generation and Propagation}

Except when otherwise stated, if the result of a floating-point operation is
NaN, it is the canonical NaN.  The canonical NaN has a positive sign and all
significand bits clear except the MSB, a.k.a. the quiet bit.  For
single-precision floating-point, this corresponds to the pattern {\tt
0x7fc00000}.

\begin{commentary}
We considered propagating NaN payloads, as is recommended by the standard,
but this decision would have increased hardware cost.  Moreover, since this
feature is optional in the standard, it cannot be used in portable code.

Implementors are free to provide a NaN payload propagation scheme as
a nonstandard extension enabled by a nonstandard operating mode.  However, the
canonical NaN scheme described above must always be supported and should be
the default mode.
\end{commentary}

\begin{commentary}
We require implementations to return the standard-mandated default
values in the case of exceptional conditions, without any further
intervention on the part of user-level software (unlike the Alpha ISA
floating-point trap barriers).  We believe full hardware handling of
exceptional cases will become more common, and so wish to avoid
complicating the user-level ISA to optimize other approaches.
Implementations can always trap to machine-mode software handlers to
provide exceptional default values.
\end{commentary}

\section{Subnormal Arithmetic}

Operations on subnormal numbers are handled in accordance with the IEEE
754-2008 standard.

In the parlance of the IEEE standard, tininess is detected after rounding.

\begin{commentary}
Detecting tininess after rounding results in fewer spurious underflow signals.
\end{commentary}

\section{Single-Precision Load and Store Instructions}

Floating-point loads and stores use the same base+offset addressing
mode as the integer base ISAs, with a base address in register {\em
  rs1} and a 12-bit signed byte offset.  The FLW instruction loads a
single-precision floating-point value from memory into floating-point
register {\em rd}.  FSW stores a single-precision value from
floating-point register {\em rs2} to memory.

\vspace{-0.2in}
\begin{center}
\begin{tabular}{M@{}R@{}F@{}R@{}O}
\\
\instbitrange{31}{20} &
\instbitrange{19}{15} &
\instbitrange{14}{12} &
\instbitrange{11}{7} &
\instbitrange{6}{0} \\
\hline
\multicolumn{1}{|c|}{imm[11:0]} &
\multicolumn{1}{c|}{rs1} &
\multicolumn{1}{c|}{width} &
\multicolumn{1}{c|}{rd} &
\multicolumn{1}{c|}{opcode} \\
\hline
12 & 5 & 3 & 5 & 7 \\
offset[11:0] & base & W & dest & LOAD-FP \\
\end{tabular}
\end{center}

\vspace{-0.2in}
\begin{center}
\begin{tabular}{O@{}R@{}R@{}F@{}R@{}O}
\\
\instbitrange{31}{25} &
\instbitrange{24}{20} &
\instbitrange{19}{15} &
\instbitrange{14}{12} &
\instbitrange{11}{7} &
\instbitrange{6}{0} \\
\hline
\multicolumn{1}{|c|}{imm[11:5]} &
\multicolumn{1}{c|}{rs2} &
\multicolumn{1}{c|}{rs1} &
\multicolumn{1}{c|}{width} &
\multicolumn{1}{c|}{imm[4:0]} &
\multicolumn{1}{c|}{opcode} \\
\hline
7 & 5 & 5 & 3 & 5 & 7 \\
offset[11:5] & src & base & W & offset[4:0] & STORE-FP \\
\end{tabular}
\end{center}

FLW and FSW are only guaranteed to execute atomically if the effective address
is naturally aligned.

FLW and FSW do not modify the bits being transferred; in particular, the
payloads of non-canonical NaNs are preserved.

As described in Section~\ref{sec:rv32:ldst}, the EEI defines whether
misaligned floating-point loads and stores are handled invisibly or raise
a contained or fatal trap.

\section{Single-Precision Floating-Point Computational Instructions}
\label{sec:single-float-compute}

Floating-point arithmetic instructions with one or two source operands use the
R-type format with the OP-FP major opcode.  FADD.S and FMUL.S perform
single-precision floating-point addition and multiplication respectively,
between {\em rs1} and {\em rs2}. FSUB.S performs the single-precision
floating-point subtraction of {\em rs2} from {\em rs1}.  FDIV.S performs the
single-precision floating-point division of {\em rs1} by {\em rs2}. FSQRT.S
computes the square root of {\em rs1}.  In each case, the result is written to
{\em rd}.

The 2-bit floating-point format field {\em fmt} is encoded as shown in
Table~\ref{tab:fmt}.  It is set to {\em S} (00) for all instructions in
the F extension.

\begin{table}[htp]
\begin{small}
\begin{center}
\begin{tabular}{|c|c|l|}
\hline
{\em fmt} field &
Mnemonic &
Meaning \\
\hline
00 & S & 32-bit single-precision \\
01 & D & 64-bit double-precision \\
10 & H & 16-bit half-precision \\
11 & Q & 128-bit quad-precision \\
\hline
\end{tabular}
\end{center}
\end{small}
\caption{Format field encoding.}
\label{tab:fmt}
\end{table}

All floating-point operations that perform rounding can select the
rounding mode using the {\em rm} field with the encoding shown in
Table~\ref{rm}.

Floating-point minimum-number and maximum-number instructions FMIN.S and
FMAX.S write, respectively, the smaller or larger of {\em rs1} and {\em rs2}
to {\em rd}.  For the purposes of these instructions only, the value $-0.0$ is
considered to be less than the value $+0.0$.  If both inputs are NaNs, the
result is the canonical NaN.  If only one operand is a NaN, the result is the
non-NaN operand.  Signaling NaN inputs set the invalid operation exception flag,
even when the result is not NaN.

\begin{commentary}
Note that in version 2.2 of the F extension, the FMIN.S and FMAX.S
instructions were amended to implement the proposed IEEE 754-201x
minimumNumber and maximumNumber operations, rather than the IEEE 754-2008
minNum and maxNum operations.  These operations differ in their handling of
signaling NaNs.
\end{commentary}

\vspace{-0.2in}
\begin{center}
\begin{tabular}{R@{}F@{}R@{}R@{}F@{}R@{}O}
\\
\instbitrange{31}{27} &
\instbitrange{26}{25} &
\instbitrange{24}{20} &
\instbitrange{19}{15} &
\instbitrange{14}{12} &
\instbitrange{11}{7} &
\instbitrange{6}{0} \\
\hline
\multicolumn{1}{|c|}{funct5} &
\multicolumn{1}{c|}{fmt} &
\multicolumn{1}{c|}{rs2} &
\multicolumn{1}{c|}{rs1} &
\multicolumn{1}{c|}{rm} &
\multicolumn{1}{c|}{rd} &
\multicolumn{1}{c|}{opcode} \\
\hline
5 & 2 & 5 & 5 & 3 & 5 & 7 \\
FADD/FSUB & S & src2 & src1 & RM  & dest & OP-FP  \\
FMUL/FDIV & S & src2 & src1 & RM  & dest & OP-FP  \\
FSQRT     & S & 0    & src  & RM  & dest & OP-FP  \\
FMIN-MAX  & S & src2 & src1 & MIN/MAX & dest & OP-FP  \\
\end{tabular}
\end{center}

Floating-point fused multiply-add instructions require a new standard
instruction format.  R4-type instructions specify three source
registers ({\em rs1}, {\em rs2}, and {\em rs3}) and a destination
register ({\em rd}).  This format is only used by the floating-point
fused multiply-add instructions.

FMADD.S multiplies the values in {\em
rs1} and {\em rs2}, adds the value in {\em rs3}, and writes the final
result to {\em rd}.  FMADD.S computes {\em (rs1$\times$rs2)+rs3}.

FMSUB.S multiplies the values in {\em rs1} and {\em rs2}, subtracts
the value in {\em rs3}, and writes the final result to {\em rd}.
FMSUB.S computes {\em (rs1$\times$rs2)-rs3}.

FNMSUB.S multiplies the
values in {\em rs1} and {\em rs2}, negates the product, adds the value
in {\em rs3}, and writes the final result to {\em rd}. FNMSUB.S
computes {\em -(rs1$\times$rs2)+rs3}.

FNMADD.S multiplies the values
in {\em rs1} and {\em rs2}, negates the product, subtracts the value
in {\em rs3}, and writes the final result to {\em rd}. FNMADD.S
computes {\em -(rs1$\times$rs2)-rs3}.

\begin{commentary}
The FNMSUB and FNMADD instructions are counterintuitively named, owing to the
naming of the corresponding instructions in MIPS-IV.  The MIPS instructions
were defined to negate the sum, rather than negating the product as the
RISC-V instructions do, so the naming scheme was more rational at the time.
The two definitions differ with respect to signed-zero results.  The RISC-V
definition matches the behavior of the x86 and ARM fused multiply-add
instructions, but unfortunately the RISC-V FNMSUB and FNMADD instruction
names are swapped compared to x86 and ARM.
\end{commentary}

\vspace{-0.2in}
\begin{center}
\begin{tabular}{R@{}F@{}R@{}R@{}F@{}R@{}O}
\\
\instbitrange{31}{27} &
\instbitrange{26}{25} &
\instbitrange{24}{20} &
\instbitrange{19}{15} &
\instbitrange{14}{12} &
\instbitrange{11}{7} &
\instbitrange{6}{0} \\
\hline
\multicolumn{1}{|c|}{rs3} &
\multicolumn{1}{c|}{fmt} &
\multicolumn{1}{c|}{rs2} &
\multicolumn{1}{c|}{rs1} &
\multicolumn{1}{c|}{rm} &
\multicolumn{1}{c|}{rd} &
\multicolumn{1}{c|}{opcode} \\
\hline
5 & 2 & 5 & 5 & 3 & 5 & 7 \\
src3 & S & src2 & src1 & RM  & dest & F[N]MADD/F[N]MSUB  \\
\end{tabular}
\end{center}

\begin{commentary}
 The fused multiply-add (FMA) instructions consume a large part of the
 32-bit instruction encoding space.  Some alternatives considered were
 to restrict FMA to only use dynamic rounding modes, but static
 rounding modes are useful in code that exploits the lack of product
 rounding.  Another alternative would have been to use rd to provide
 rs3, but this would require additional move instructions in some
 common sequences.  The current design still leaves a large portion of
 the 32-bit encoding space open while avoiding having FMA be
 non-orthogonal.
\end{commentary}

The fused multiply-add instructions must set the invalid operation exception flag
when the multiplicands are $\infty$ and zero, even when the addend is a quiet
NaN.
\begin{commentary}
The IEEE 754-2008 standard permits, but does not require, raising the
invalid exception for the operation \mbox{$\infty\times 0\ +$ qNaN}.
\end{commentary}

\section{Single-Precision Floating-Point Conversion and Move \mbox{Instructions}}

Floating-point-to-integer and integer-to-floating-point conversion
instructions are encoded in the OP-FP major opcode space.
FCVT.W.S or FCVT.L.S converts a floating-point number
in floating-point register {\em rs1} to a signed 32-bit or 64-bit
integer, respectively, in integer register {\em rd}.  FCVT.S.W
or FCVT.S.L converts a 32-bit or 64-bit signed integer,
respectively, in integer register {\em rs1} into a floating-point
number in floating-point register {\em rd}. FCVT.WU.S,
FCVT.LU.S, FCVT.S.WU, and FCVT.S.LU variants
convert to or from unsigned integer values.
For XLEN$>32$, FCVT.W[U].S sign-extends the 32-bit result to the
destination register width.
FCVT.L[U].S and FCVT.S.L[U] are RV64-only instructions.
If the rounded result is not representable in the destination format,
it is clipped to the nearest value and the invalid flag is set.
Table~\ref{tab:int_conv} gives the range of valid inputs for FCVT.{\em int}.S
and the behavior for invalid inputs.

\begin{table}[htp]
\begin{small}
\begin{center}
\begin{tabular}{|l|r|r|r|r|}
\hline
 & FCVT.W.S & FCVT.WU.S & FCVT.L.S & FCVT.LU.S \\
\hline
Minimum valid input (after rounding) & $-2^{31}$ & 0 & $-2^{63}$ & 0 \\
Maximum valid input (after rounding) & $2^{31}-1$ & $2^{32}-1$ & $2^{63}-1$ & $2^{64}-1$ \\
\hline
Output for out-of-range negative input & $-2^{31}$ & 0 & $-2^{63}$ & 0 \\
Output for $-\infty$ & $-2^{31}$ & 0 & $-2^{63}$ & 0 \\
Output for out-of-range positive input & $2^{31}-1$ & $2^{32}-1$ & $2^{63}-1$ & $2^{64}-1$ \\
Output for $+\infty$ or NaN & $2^{31}-1$ & $2^{32}-1$ & $2^{63}-1$ & $2^{64}-1$ \\
\hline
\end{tabular}
\end{center}
\end{small}
\caption{Domains of float-to-integer conversions and behavior for invalid inputs.}
\label{tab:int_conv}
\end{table}

All floating-point to integer and integer to floating-point conversion
instructions round according to the {\em rm} field.  A floating-point register
can be initialized to floating-point positive zero using FCVT.S.W {\em rd},
{\tt x0}, which will never set any exception flags.

All floating-point conversion instructions set the Inexact exception flag if
the rounded result differs from the operand value and the Invalid exception
flag is not set.

\vspace{-0.2in}
\begin{center}
\begin{tabular}{R@{}F@{}R@{}R@{}F@{}R@{}O}
\\
\instbitrange{31}{27} &
\instbitrange{26}{25} &
\instbitrange{24}{20} &
\instbitrange{19}{15} &
\instbitrange{14}{12} &
\instbitrange{11}{7} &
\instbitrange{6}{0} \\
\hline
\multicolumn{1}{|c|}{funct5} &
\multicolumn{1}{c|}{fmt} &
\multicolumn{1}{c|}{rs2} &
\multicolumn{1}{c|}{rs1} &
\multicolumn{1}{c|}{rm} &
\multicolumn{1}{c|}{rd} &
\multicolumn{1}{c|}{opcode} \\
\hline
5 & 2 & 5 & 5 & 3 & 5 & 7 \\
FCVT.{\em int}.{\em fmt} & S & W[U]/L[U] & src & RM  & dest & OP-FP  \\
FCVT.{\em fmt}.{\em int} & S & W[U]/L[U] & src & RM  & dest & OP-FP  \\
\end{tabular}
\end{center}

Floating-point to floating-point sign-injection instructions, FSGNJ.S,
FSGNJN.S, and FSGNJX.S, produce a result that takes all bits except
the sign bit from {\em rs1}.  For FSGNJ, the result's sign bit is {\em
  rs2}'s sign bit; for FSGNJN, the result's sign bit is the opposite
of {\em rs2}'s sign bit; and for FSGNJX, the sign bit is the XOR of
the sign bits of {\em rs1} and {\em rs2}.  Sign-injection instructions
do not set floating-point exception flags, nor do they canonicalize
NaNs.  Note, FSGNJ.S {\em rx, ry,
  ry} moves {\em ry} to {\em rx} (assembler pseudoinstruction FMV.S {\em rx,
  ry}); FSGNJN.S {\em rx, ry, ry} moves the negation of {\em ry} to
{\em rx} (assembler pseudoinstruction FNEG.S {\em rx, ry}); and FSGNJX.S {\em rx,
  ry, ry} moves the absolute value of {\em ry} to {\em rx} (assembler
pseudoinstruction FABS.S {\em rx, ry}).

\vspace{-0.2in}
\begin{center}
\begin{tabular}{R@{}F@{}R@{}R@{}F@{}R@{}O}
\\
\instbitrange{31}{27} &
\instbitrange{26}{25} &
\instbitrange{24}{20} &
\instbitrange{19}{15} &
\instbitrange{14}{12} &
\instbitrange{11}{7} &
\instbitrange{6}{0} \\
\hline
\multicolumn{1}{|c|}{funct5} &
\multicolumn{1}{c|}{fmt} &
\multicolumn{1}{c|}{rs2} &
\multicolumn{1}{c|}{rs1} &
\multicolumn{1}{c|}{rm} &
\multicolumn{1}{c|}{rd} &
\multicolumn{1}{c|}{opcode} \\
\hline
5 & 2 & 5 & 5 & 3 & 5 & 7 \\
FSGNJ & S & src2 & src1 & J[N]/JX & dest & OP-FP  \\
\end{tabular}
\end{center}

\begin{commentary}
The sign-injection instructions
provide floating-point MV, ABS, and NEG,
as well as supporting a few other operations, including the IEEE copySign
operation and sign manipulation in transcendental math function
libraries.  Although MV, ABS, and NEG only need a single register
operand, whereas FSGNJ instructions need two, it is unlikely most
microarchitectures would add optimizations to benefit from the reduced
number of register reads for these relatively infrequent instructions.
Even in this case, a microarchitecture can simply detect when both
source registers are the same for FSGNJ instructions and only read a
single copy.
\end{commentary}

Instructions are provided to move bit patterns between the
floating-point and integer registers.  FMV.X.W moves the
single-precision value in floating-point register {\em rs1}
represented in IEEE 754-2008 encoding to the lower 32 bits of integer
register {\em rd}.  The bits are not
modified in the transfer, and in particular, the payloads of
non-canonical NaNs are preserved.
For RV64, the higher 32 bits of the destination
register are filled with copies of the floating-point number's sign
bit.

FMV.W.X moves the single-precision value encoded in IEEE
754-2008 standard encoding from the lower 32 bits of integer register
{\em rs1} to the floating-point register {\em rd}.  The bits are not
modified in the transfer, and in particular, the payloads of
non-canonical NaNs are preserved.

\begin{commentary}
The FMV.W.X and FMV.X.W instructions were previously called FMV.S.X
and FMV.X.S.  The use of W is more consistent with their semantics as
an instruction that moves 32 bits without interpreting them.  This
became clearer after defining NaN-boxing.  To avoid disturbing
existing code, both the W and S versions will be supported by tools.
\end{commentary}

\vspace{-0.2in}
\begin{center}
\begin{tabular}{R@{}F@{}R@{}R@{}F@{}R@{}O}
\\
\instbitrange{31}{27} &
\instbitrange{26}{25} &
\instbitrange{24}{20} &
\instbitrange{19}{15} &
\instbitrange{14}{12} &
\instbitrange{11}{7} &
\instbitrange{6}{0} \\
\hline
\multicolumn{1}{|c|}{funct5} &
\multicolumn{1}{c|}{fmt} &
\multicolumn{1}{c|}{rs2} &
\multicolumn{1}{c|}{rs1} &
\multicolumn{1}{c|}{rm} &
\multicolumn{1}{c|}{rd} &
\multicolumn{1}{c|}{opcode} \\
\hline
5 & 2 & 5 & 5 & 3 & 5 & 7 \\
FMV.X.W & S & 0    & src  & 000  & dest & OP-FP  \\
FMV.W.X & S & 0    & src  & 000  & dest & OP-FP  \\
\end{tabular}
\end{center}

\begin{commentary}
The base floating-point ISA was defined so as to allow implementations
to employ an internal recoding of the floating-point format in
registers to simplify handling of subnormal values and possibly to
reduce functional unit latency.  To this end, the F extension avoids
representing integer values in the floating-point registers by
defining conversion and comparison operations that read and write the
integer register file directly.  This also removes many of the common
cases where explicit moves between integer and floating-point
registers are required, reducing instruction count and critical paths
for common mixed-format code sequences.
\end{commentary}

\section{Single-Precision Floating-Point Compare Instructions}

Floating-point compare instructions (FEQ.S, FLT.S, FLE.S) perform the
specified comparison between floating-point registers ($\mbox{\em rs1}
= \mbox{\em rs2}$, $\mbox{\em rs1} < \mbox{\em rs2}$, $\mbox{\em rs1} \leq
\mbox{\em rs2}$) writing 1 to the integer register {\em rd} if the condition
holds, and 0 otherwise.

FLT.S and FLE.S perform what the IEEE 754-2008 standard refers to as {\em
signaling} comparisons: that is, they set the invalid operation exception flag
if either input is NaN.  FEQ.S performs a {\em quiet} comparison: it only
sets the invalid operation exception flag if either input is a signaling NaN.
For all three instructions,
the result is 0 if either operand is NaN.

\vspace{-0.2in}
\begin{center}
\begin{tabular}{S@{}F@{}R@{}R@{}F@{}R@{}O}
\\
\instbitrange{31}{27} &
\instbitrange{26}{25} &
\instbitrange{24}{20} &
\instbitrange{19}{15} &
\instbitrange{14}{12} &
\instbitrange{11}{7} &
\instbitrange{6}{0} \\
\hline
\multicolumn{1}{|c|}{funct5} &
\multicolumn{1}{c|}{fmt} &
\multicolumn{1}{c|}{rs2} &
\multicolumn{1}{c|}{rs1} &
\multicolumn{1}{c|}{rm} &
\multicolumn{1}{c|}{rd} &
\multicolumn{1}{c|}{opcode} \\
\hline
5 & 2 & 5 & 5 & 3 & 5 & 7 \\
FCMP & S & src2 & src1 & EQ/LT/LE & dest & OP-FP  \\
\end{tabular}
\end{center}

\begin{commentary}
The F extension provides a $\leq$ comparison, whereas the base ISAs provide
a $\geq$ branch comparison.  Because $\leq$ can be synthesized from $\geq$ and
vice-versa, there is no performance implication to this inconsistency, but it
is nevertheless an unfortunate incongruity in the ISA.
\end{commentary}

\section{Single-Precision Floating-Point Classify Instruction}

The FCLASS.S instruction examines the value in floating-point register {\em
rs1} and writes to integer register {\em rd} a 10-bit mask that indicates
the class of the floating-point number.  The format of the mask is
described in Table~\ref{tab:fclass}.  The corresponding bit in {\em rd} will
be set if the property is true and clear otherwise.  All other bits in
{\em rd} are cleared.  Note that exactly one bit in {\em rd} will be set.
FCLASS.S does not set the floating-point exception flags.

\vspace{-0.2in}
\begin{center}
\begin{tabular}{S@{}F@{}R@{}R@{}F@{}R@{}O}
\\
\instbitrange{31}{27} &
\instbitrange{26}{25} &
\instbitrange{24}{20} &
\instbitrange{19}{15} &
\instbitrange{14}{12} &
\instbitrange{11}{7} &
\instbitrange{6}{0} \\
\hline
\multicolumn{1}{|c|}{funct5} &
\multicolumn{1}{c|}{fmt} &
\multicolumn{1}{c|}{rs2} &
\multicolumn{1}{c|}{rs1} &
\multicolumn{1}{c|}{rm} &
\multicolumn{1}{c|}{rd} &
\multicolumn{1}{c|}{opcode} \\
\hline
5 & 2 & 5 & 5 & 3 & 5 & 7 \\
FCLASS & S & 0 & src & 001 & dest & OP-FP  \\
\end{tabular}
\end{center}

\begin{table}[htp]
\begin{small}
\begin{center}
\begin{tabular}{|c|l|}
\hline
{\em rd} bit &
Meaning \\
\hline
0 & {\em rs1} is $-\infty$. \\
1 & {\em rs1} is a negative normal number. \\
2 & {\em rs1} is a negative subnormal number. \\
3 & {\em rs1} is $-0$. \\
4 & {\em rs1} is $+0$. \\
5 & {\em rs1} is a positive subnormal number. \\
6 & {\em rs1} is a positive normal number. \\
7 & {\em rs1} is $+\infty$. \\
8 & {\em rs1} is a signaling NaN. \\
9 & {\em rs1} is a quiet NaN. \\
\hline
\end{tabular}
\end{center}
\end{small}
\caption{Format of result of FCLASS instruction.}
\label{tab:fclass}
\end{table}
