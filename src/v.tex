\chapter{``V'' Standard Extension for Vector Operations, Version 0.3-DRAFT}
\label{sec:bits}

This chapter presents a proposal for the RISC-V vector instruction set
extension.  The vector extension supports a configurable vector unit,
to tradeoff the number of architectural vector registers and supported
element widths against available maximum vector length.  The vector
extension is designed to allow the same binary code to work
efficiently across a variety of hardware implementations varying in
physical vector storage capacity and datapath spatial and/or temporal
parallelism.  The base vector extension is intended to provide general
support for data-parallel execution within the 32-bit instruction
encoding space, with later vector extensions supporting richer
functionality for certain domains.

\begin{commentary}
The vector extension is based on the style of vector register
architecture introduced by Seymour Cray in the 1970s, as opposed to
the earlier packed SIMD approach, introduced with the Lincoln Labs
TX-2 in 1957 and now adopted by most other commercial instruction
sets.

The vector instruction set contains many features developed in earlier
research projects, including the Berkeley T0~\cite{} and VIRAM~\cite{}
vector microprocessors, the MIT Scale vector-thread processor~\cite{},
and the Berkeley Maven~\cite{} and Hwacha~\cite{} projects.
\end{commentary}

\section{Vector Unit State}

The additional vector unit architectural state consists of 32 vector
data registers ({\tt v0}--{\tt v31}), 8 vector predicate registers
({\tt vp0}-{\tt vp7}), and an XLEN-bit WARL vector length CSR, {\tt
  vl}.  In addition, the current configuration of the vector unit is
held in a set of vector configuration CSRs ({\tt vdcfg0}--{\tt vdcfg7}
and {\tt vnp}), as described below.  The implementation determines an
available {\em maximum vector length} (MVL) for the current
configuration held in the {\tt vdcfg} and {\tt vnp} registers.  There
is also a 3-bit fixed-point rounding mode CSR {\tt vxrm}, and a
single-bit fixed-point saturation status CSR {\tt vxsat}.

\begin{commentary}
  Future vector extensions using wider instruction encodings can
  support more architectural vector registers. For example, 256
  architectural vector registers in a 64 bit encoding.
\end{commentary}

The {\tt vcs} CSR alias provides combined access to the {\tt vl}, {\tt
  vxrm}, {\tt vxsat}, and {\tt vnp} fields to reduce context switch
time.  The {\tt vcs} register also includes a configuration mode field
to support future extended configuration modes.

\begin{discussion}
The components of vcs might not need separate CSR addresses,
depending on how they're accessed via other non-CSR instructions.
\end{discussion}

\begin{table}
  \centering
  \begin{tabular}{|l|c|l|l|}
    \hline
    CSR name & Number & Base ISA & Description\\
    \hline
    {\tt vcs}  & TBD & RV32, RV64, RV128 & Vector control-status register\\
    {\tt vl}    & TBD & RV32, RV64, RV128 & Active vector length\\
    {\tt vxrm}  & TBD & RV32, RV64, RV128 & Vector fixed-point rounding mode\\
    {\tt vxsat} & TBD & RV32, RV64, RV128 & Vector fixed-point saturation flag \\
    \hline
    {\tt vnp} & TBD & RV32, RV64, RV128 & Number of vector predicate registers\\
    \hline
    {\tt vdcfg0} & TBD & RV32, RV64, RV128 & \multirow{8}{*}{Vector
      data register configuration}\\
    {\tt vdcfg1} & TBD & RV32 &\\
    {\tt vdcfg2} & TBD & RV32, RV64 &\\
    {\tt vdcfg3} & TBD & RV32 &\\
    {\tt vdcfg4}  & TBD & RV32, RV64, RV128 &\\
    {\tt vdcfg5} & TBD & RV32 &\\
    {\tt vdcfg6} & TBD & RV32, RV64 &\\
    {\tt vdcfg7} & TBD & RV32 &\\
    \hline
  \end{tabular}
  \caption{Vector extension CSRs.}
  \label{tab:vcsrs}
\end{table}

The vector unit must be configured before use.  Each architectural
vector data register ({\tt v0}--{\tt v31}) is configured with the bit
width and type of each element of that vector data register, or can be
disabled to free physical vector storage for other architectural
vector data registers.  The number of available vector predicate
registers can also be set independently, from 0 to 8.

\begin{commentary}
  Several earlier vector machines had the ability to configure
  physical vector register storage into a larger number of short
   vectors or a shorter number of long vectors, in particular the
  Fujitsu VP series~\cite{vp200}.
\end{commentary}

The available MVL depends on the configuration setting, but MVL must
always have the same value for the same configuration parameters on a
given implementation.  Implementations must provide an MVL of at least
four elements for all supported configuration settings.

\begin{commentary}
  Specifying a minimum MVL allows operations on known-short vectors to
  be expressed without requiring stripmining instructions.
\end{commentary}

\begin{discussion}
Both min(MVL) and max(MVL) might be better expressed as part of a
profile.
\end{discussion}

Each vector data register's current configuration is described with an
8-bit encoding split into a 3-bit current maximum-width field {\tt
  vemaxw}$n$\, and a 5-bit type field {\tt vetype}$n$, held in the
{\tt vdcfg}$x$ CSRs.  The configuration state is also accessible via
other specialized vector configuration instructions.

\section{Element Datatypes and Width}

The datatypes and operations supported by the V extension depend upon
the base scalar ISA and supported extensions, and may include 8-bit,
16-bit, 32-bit, 64-bit, and 128-bit integer and fixed-point data types
(X8/U8, X16/U16, X32/U32, X64/U64, and X128/U128 respectively,
where U indicates unsigned), and 16-bit, 32-bit, 64-bit,
and 128-bit floating-point types (F16, F32, F64, and F128
respectively).  When the V extension is added, it must support the
vector data element types implied by the supported scalar types as
defined by Table~\ref{tab:velemtypes}.  The largest element width
supported:
\[ \mbox{\em ELEN} = max(\mbox{\em XLEN}, \mbox{\em FLEN}) \]

\begin{commentary}
  Compiler support for vectorization is greatly simplified when any
  hardware-supported data types are supported by both scalar and
  vector instructions.
\end{commentary}

\begin{table}
  \centering
\begin{tabular}{|l|l|}
  \hline
  \multicolumn{2}{|c|}{Supported Fixed-Point Types} \\
  \hline
  RV32I  & X8, U8, X16, U16, X32, U32 \\
  RV64I  & X8, U8, X16, U16, X32, U32, X64, U64 \\
  RV128I & X8, U8, X16, U16, X32, U32, X64, U64, X128, U128 \\
  \hline
  \hline
  \multicolumn{2}{|c|}{Supported Floating-Point Types} \\
  \hline
  F      & F16, F32 \\
  FD     & F16, F32, F64 \\
  FDQ    & F16, F32, F64, F128 \\
  \hline
\end{tabular}
\caption{Supported data element types depending on base integer ISA
  and supported floating-point extensions.  Signed and unsigned
  integers are given separate types (e.g, X32 is signed 32-bit value,
  whereas U32 is an unsigned integer value). Note that supporting a
  given floating-point width mandates support for all narrower
  floating-point widths.}
\label{tab:velemtypes}
\end{table}

\begin{commentary}
  Future vector extensions might expand the set of supported
  datatypes, including custom application-specific datatypes.
\end{commentary}

Adding the vector extension to any machine with floating-point support
adds support for the IEEE standard half-precision 16-bit
floating-point data type.  This includes a set of scalar
half-precision instructions described in
Section~\ref{sec:scalarhalffloat}.  The scalar half-precision
instructions follow the template for other floating-point precisions,
but using the hitherto unused {\em fmt} field encoding of {\tt 10}.

\begin{commentary}
  There is interest in splitting off the scalar half-precision
  instructions into their own named extension.
\end{commentary}


\section{Vector Element Width ({\tt vemaxw}$n$)}

The current maximum element width for vector data register $n$ is held
in a three-bit field, {\tt vemaxw}$n$, encoded as shown in
Table~\ref{tab:vemaxw}.

\begin{table}[hbt]
  \centering
  \begin{tabular}{|r|c|}
    \hline
    Width & Encoding \\
    \hline
    Disabled  & 000 \\
    8         & 100  \\
    16        & 101  \\
    32        & 110  \\
    64        & 111  \\
    128       & 011  \\
%%  256       & 010  \\
%%  512       & 001  \\
    \hline
  \end{tabular}
  \caption{Encoding of vector element maximum-width fields {\tt
      vemaxw0}--{\tt vemaxw31}. All other values are reserved.}
  \label{tab:vemaxw}
\end{table}

\begin{commentary}
Future extensions might increase the supported vector element widths
beyond those of the base scalar ISA, or support smaller non-power-of-2
widths.  At least one of the remaining width values should be reserved
to support a width-encoding escape to support this larger range of
width values.
\end{commentary}

\begin{commentary}
Three broad classes of implementation can be distinguished by how they
handle {\tt vemaxw}$n$ settings.

The simplest is {\em max-width-per-implementation} (MWPI), where the
vector unit is organized in fixed ELEN-width physical lanes, and
changes to {\tt vemaxw}$n$ settings simply cause portions of the
physical registers and datapath to be disabled for operations narrower
than ELEN bits.

The next most complex implementation, {\em
  max-width-per-configuration} (MWPC), uses the maximum width across
all {\tt vemaxw}$n$ settings in a dynamic configuration to divide the
physical register storage and datapaths.  For example, a MWPC machine
with ELEN=64 might subdivide physical lanes into 32-bit datapaths if
no {\tt vemaxw}$n$ setting is greater than 32.  Operations on
sub-32-bit quantities would disable appropriate portions of the
physical registers and functional units in each 32-bit lane.  Several
early vector supercomputers, including the CDC
Star-100~\cite{cdcstart100}, provided a similar facility to divide
64-bit physical vector lanes into narrower 32-bit lanes.

The most complex implementations are {\em max-width-per-register}
(MWPR), which reduce wasted space in the physical register files by
packing elements in each vector register according to the individual
{\tt vemaxw}$n$ settings and which within one configuration can
execute instructions with narrower datatypes at higher rates than for
wider datatypes.  The Berkeley Hwacha vector
engine~\cite{hwachatr,mixedprecision} is an example microarchitecture
with this property.
\end{commentary}

Any write to any {\tt vemaxw}$n$ field configures the entire vector
unit and causes all vector data registers to be zeroed and all vector
predicate registers to be set, and the vector length register {\tt vl}
to be set to the maximum supported vector length.

\begin{commentary}
  Vector registers are zeroed on reconfiguration to prevent security
  holes and to avoid exposing differences between how different
  implementations manage physical vector register storage.

  In-order implementations will probaby use a flag bit per register to
  mux in 0 instead of garbage values on each source until it is
  overwritten.  For in-order machines, vector lengths less than MVL
  complicate this zeroing, but these cases can be handled by adding a
  zero bit per element or element group.  Machines with vector
  register renaming can just initialize the rename table to point
  entries at a physical zero register.
\end{commentary}

If a vector data register is disabled, then any vector instruction
that attempts to access that vector data register will raise an
illegal instruction exception.  Attempting to write any {\tt
  vemaxw}$n$ with an unsupported value will raise an illegal
instruction exception.

\section{Vector Element Type ({\tt vetype}$n$)}

The current element type of vector data register $n$ is held in a
five-bit {\tt vetype}$n$ field encoded as shown in
Table~\ref{tab:vetype}.  The element type {\tt vetype}$n$ of a vector
data register is constrained to have equal or lesser width than the
value in the corresponding {\tt vemaxw}$n$ field.  A write to a {\tt
  vetype}$n$ field zeros the associated vector data register {\tt
  v}$n$, but leaves other vector unit state undisturbed.  Changes to
{\tt vetype}$n$ do not alter MVL.

\begin{table}[hbt]
  \centering
  \begin{tabular}{|l|c|c|}
    \hline
    Type & {\tt vemaxw} equivalent & {\tt vetype} encoding \\
    \hline
    Disabled & 000 & 00000 \\
    \hline
    \hline
    \multicolumn{3}{|c|}{Floating-Point types} \\
    \hline
    F16      & 101 & 01101 \\
    F32      & 110 & 01110 \\
    F64      & 111 & 01111 \\
    F128     & 011 & 01011 \\
    \hline
    \hline
    \multicolumn{3}{|c|}{Signed integer and fixed-point types} \\
    \hline
    X8       & 100 & 10100  \\
    X16      & 101 & 10101  \\
    X32      & 110 & 10110  \\
    X64      & 111 & 10111  \\
    X128     & 011 & 10011  \\
    \hline
    \hline
    \multicolumn{3}{|c|}{Unsigned integer and fixed-point types} \\
    \hline
    U8      & 100 & 11100  \\
    U16     & 101 & 11101  \\
    U32     & 110 & 11110  \\
    U64     & 111 & 11111  \\
    U128    & 011 & 11011  \\
    \hline
  \end{tabular}
  \caption{Encoding of {\tt vetype} fields.  All other values are
    reserved. The middle column shows the value that will be written
    to {\tt vemaxw}$n$ for configuration instructions that write both
    {\tt vetype}$n$ and {\tt vemaxw}$n$ fields. For these standard
    types, {\tt vemaxw}$n$ follows the low three bits of {\tt
      vetype}$n$. The value of {\tt vetype}$n$ can be changed
    independently of {\tt vemaxw}$n$ provided the required element
    width is less than or equal to {\tt vemaxw}$n$.}
  \label{tab:vetype}
\end{table}

\begin{commentary}
  Vector data registers have both a maximum element width and a
  current element data type to allow the same vector data register to
  be changed to different types during execution provided the
  maximum width is not exceeded.  This reduces register pressure and
  helps support vector function calls, where the caller does not know
  the types needed by the callee, as described below.
\end{commentary}

\begin{commentary}
  The set of supported types might be greatly increased with future
  extensions.  For example (and not limited to), new scalar types in
  new number systems, a complex type with real and imaginary
  components, a key-value type, or an application-specific structure
  type with multiple consitituent fields.  Auxiliary type
  configuration state might be required in these cases.
\end{commentary}

Attempting to write an unsupported type or a type that requires more
than the current {\tt vemaxw} width to a {\tt vetype} field will raise
an illegal instruction exception.

\begin{commentary}
Implementations must still raise an exception for a {\tt vetype}$n$
setting that is greater than the architectural {\tt vemaxw}$n$ width,
even if they internally implement a larger physical {\tt vemaxw}$n$
that could accomodate the {\tt vetype}$n$ request.
\end{commentary}

\begin{discussion}
We can either have 1) implementations raise exceptions whenever
illegal values are written to {\tt vemaxw} and {\tt vetype} fields
(current design), 2) raise exceptions at use if config holds illegal
values, 3) make the fields WARL so silently reduce to supported types
with no exceptions.  Option 2 could complicate vector unit context
switch code by having more cases to check, while Option 3 could make
debugging more difficult by allowing code to run with reduced
precision or incorrect types.
\end{discussion}

\section{Vector Predicate Configuration Register ({\tt vnp})}

The {\tt vnp} CSR holds a single 4-bit value giving the number of
enabled architectural predicate registers, between 0 and 8.  Any write
to {\tt vnp} zeros all vector data registers, sets all bits in visible
vector predicate registers, and sets the vector length register {\tt
  vl} to the maximum supported vector length.  Attempting to write a
value larger than 8 to {\tt vnp} raises an illegal instruction
exception.

\begin{discussion}
The number of vector predicate registers supported in
  base ISA could be changed.  The base encoding could support up to 32
  predicate registers, but it is not clear these would be used
  frequently enough to warrant increased the architectural cost for
  all implementations.
\end{discussion}

When {\tt vnp} is 0, any instruction that reads a vector predicate
register other than {\tt vp0} will raise an illegal instruction
exception, while reads of {\tt vp0} will return all ones to provide
unpredicated execution.  When {\tt vnp} is 0, any instruction that
attempts to write any vector predicate register will raise an illegal
instruction exception.

\section{Vector Data Configuration Registers ({\tt vdcfg0}--{\tt vdcfg7})}

The vector data register configuration requires 256 bits of state (32
vector data registers each with a 3-bit {\tt vemaxw}$n$ field and a
5-bit {\tt vetype}$n$ field), and is held in the {\tt vdcfg CSRs}.

RV128 has two vector configuration CSRs: {\tt vdcfg0} holds
configuration data for {\tt v0}--{\tt v15} with bits $8n$ to $8n+4$
holding {\tt vetype}$n$ and bits $8n+5$ to $8n+7$ holding {\tt
  vemaxw}$n$, while {\tt vdcfg4} similarly holds configuration data
for {\tt v16}--{\tt v31}.

In RV64, the {\tt vdcfg2} CSR provides access to the upper 64 bits of {\tt
  vdcfg0} and {\tt vdcfg6} provides access to the upper 64 bits of
{\tt vdcfg4}.  In RV32, the {\tt vdcfg1}, {\tt vdcfg3}, {\tt vdcfg5}
and {\tt vdcfg7} CSRs provides access to the upper bits of {\tt
  vdcfg0}, {\tt vdcfg2}, {\tt vdcfg4} and {\tt vdcfg6} respectively.

Any CSR write to a {\tt vdcfg}$x$ register zeros all {\tt vdcfg}$y$
registers, for $y>x$, and also zeros the {\tt vnp} register.  As a
result configuration data should be written from the {\tt vdcfg0} CSR
upwards, followed by the {\tt vnp} setting if non-zero.

\begin{commentary}
  Zeroing higher-numbered {\tt vdcfg}$y$ registers allows more rapid
  reconfiguration of the vector register file via CSR writes, and
  provides backward-compatibility for extensions that increase the
  number of possible architectural vector registers.  This choice does
  prevent the use of CSRRW instructions to swap the configuration
  context.
\end{commentary}

\begin{commentary}
Additional instructions are provided to support more rapid changes to
the vector unit configuration as described below. These directly
affect the {\tt vemaxw}$n$ and {\tt vetype}$n$ fields and do not
necessarily have the same side effects as the CSR writes through the
{\tt vdcfg}$n$ addresses.
\end{commentary}


\section{Legal Vector Unit Configurations}

To simplify hardware configuration calculations and to reduce software
context-switch complexity, vector unit configurations are constrained
to have non-disabled architectural vector registers numbered
contiguously starting at {\tt v0}.  Also, {\tt vemaxw}$m$ must be
greater than or equal to {\tt vemaxw}$n$, for $m > n$, i.e.,
configured element widths must increase monotonically with
architectural vector register number.  An exception will be raised if
any instruction tries to change {\tt vemax}$n$ in a way that violates
this constraint.

\begin{commentary}
  During a software vector-context save, the software handler can stop
  searching for active architectural registers after encountering the
  first disabled vector register.  Hardware to calculate physical
  register allocation might be slightly simplified with this
  constraint, and might be able to pack register storage more tightly
  with monotonically increasing element size.

  In a vector-function calling convention, higher-numbered registers
  are usually made available to the callee, and must usually be a
  wider, often ELEN-width, element.  The context that configures the
  vector unit might have known-narrower element types and can save
  storage by confguring the lower-numbered architectural vector
  registers accordingly.
\end{commentary}


\section{Vector Instruction Formats}

\begin{commentary}
  The instruction encoding is a work in progress.

  An important design goal was that the base vector extension fit
  within a few major opcodes of the 32-bit encoding.  It is envisioned
  that future vector extensions will use 48-bit or 64-bit encodings to
  increase both the opcode space and the set of architectural
  registers.  The 64-bit vector encoding would support 256
  architectural vector registers and orthogonal specification of a
  predicate register in each instruction.
\end{commentary}

Vector arithmetic and vector memory instructions are encoded in new
variants of the R-format, shown in Figure~\ref{fig:vinstformats}.
Both new formats use one bit to hold a {\em vp} field, which usually
controls the predicate register in use, either {\tt vp0} or {\tt vp1}.
The VR4 form is used for fused multiply-add instructions.  The
existing RISC-V instruction formats are used for other vector-related
instructions, such as the vector configuration instructions.

\vspace{-0.2in}
\begin{figure}[h]
\begin{center}
\setlength{\tabcolsep}{4pt}
\begin{tabular}{p{0.7in}@{}p{0.4in}@{}p{0.7in}@{}p{0.7in}@{}p{0.5in}@{}p{0.4in}@{}p{0.7in}@{}p{1in}l}
\\
\instbitrange{31}{27} &
\instbitrange{26}{25} &
\instbitrange{24}{20} &
\instbitrange{19}{15} &
\instbitrange{14}{13} &
\instbit{12} &
\instbitrange{11}{7} &
\instbitrange{6}{0} \\
\cline{1-8}
\multicolumn{2}{|c|}{funct7} &
\multicolumn{1}{c|}{rs2} &
\multicolumn{1}{c|}{rs1} &
\multicolumn{1}{c|}{funct2} &
\multicolumn{1}{c|}{vp} &
\multicolumn{1}{c|}{rd} &
\multicolumn{1}{c|}{opcode} &
VR-type \\
\cline{1-8}
\\
\cline{1-8}
\multicolumn{1}{|c|}{rs3} &
\multicolumn{1}{c|}{fmt} &
\multicolumn{1}{c|}{rs2} &
\multicolumn{1}{c|}{rs1} &
\multicolumn{1}{c|}{funct2} &
\multicolumn{1}{c|}{vp} &
\multicolumn{1}{c|}{rd} &
\multicolumn{1}{c|}{opcode} &
VR4-type \\
\cline{1-8}
\end{tabular}
\end{center}
\caption{New V extension instruction formats.  }
\label{fig:vinstformats}
\end{figure}

Most vector instructions are available in both vector-vector and
vector-scalar variants.  Vector-vector instructions take the first
operand from the vector register specified by {\em rs1} and the second
operand from the vector register specified by {\em rs2}.

For vector-scalar operations, the {\em rs1} field specifies the scalar
register to be accessed.  For most vector-scalar instructions, the
type of the vector operand specified by {\em rs2} indicates whether
the integer or floating-point scalar register file is accessed using
the {\em rs1} register specifier.

Some non-commutative vector-scalar instructions (such as sub) are
provided in two forms, with the scalar value used as the second
operand.

\begin{commentary}
  The {\em rs1} field is used to provide the scalar operand because in
  the base encoding, whenever an instruction has a single scalar
  source operand, it is encoded in the {\tt rs1} field.
\end{commentary}

\section{Polymorphic Vector Instructions}

The vector extension uses a polymorphic instruction encoding where the
opcode is combined with the types of the source and destination
registers to determine the operation to be performed.  For example, an
ADD opcode will perform a 32-bit integer vector-vector add if both
vector source operands and the vector destination register are 32-bit
integers, but will perform a 16-bit floating-point vector-vector
operation if both vector source operands and the vector destination
are 16-bit floats.

The polymorphic encoding also naturally supports operations with mixed
precisions on the input and output, and also supports extending the
instruction set with new types without necessarily increasing the
opcode space.

Not all combinations of source and destination argument types need be
supported.  The base vector extension mandates only that
implementations provide a subset of combinations of types on inputs
and outputs.  Table~\ref{tab:vtypemix} shows the general rules for
integer and floating-point instructions, but the detailed instruction
listing should be consulted for accurate information.

\begin{table}
  \centering
  \begin{tabular}{|r|r|r|r|r|}
    \hline
    \multicolumn{1}{|c|}{Src1} &
    \multicolumn{1}{c|}{Src2} &
    \multicolumn{1}{c|}{Src3} &
    \multicolumn{1}{c|}{Dest} &
    \multicolumn{1}{c|}{Example} \\
    \hline
    \hline
    \multicolumn{5}{|c|}{Integer vector-scalar}\\
    \hline
    XLEN &   X & - &  X & 64b + 32b $\rightarrow$ 32b \\
    XLEN &   X & - & 2X & 64b + 8b  $\rightarrow$ 16b \\
    \hline
    \hline
    \multicolumn{5}{|c|}{Integer vector-vector}\\
    \hline
      X &  X & - &   X & 32b + 32b $\rightarrow$ 32b \\
      X &  X & - &  2X & 16b + 16b $\rightarrow$ 32b \\
     2X &  X & - &  2X & 64b + 32b $\rightarrow$ 64b \\
    \hline
    \hline
    \multicolumn{5}{|c|}{Floating-point vector-scalar}\\
    \hline
     F &  F & -  &  F &  64b + 64b $\rightarrow$ 64b \\
     F &  F & F  &  F &  32b $\times$ 32b + 32b $\rightarrow$ 32b \\
     F &  F & -  & 2F &  32b + 32b $\rightarrow$ 64b \\
     F &  F & 2F & 2F &  32b $\times$ 32b + 64b $\rightarrow$ 64b \\
    \hline
    \hline
    \multicolumn{5}{|c|}{Floating-point vector-vector}\\
    \hline
      F &  F  & - &   F & 32b + 32b $\rightarrow$ 32b \\
      F &  F  & - &  2F & 16b + 16b $\rightarrow$ 32b \\
     2F &  F  & - &  2F & 64b + 32b $\rightarrow$ 64b \\
      F &  F & F  &  F &  64b $\times$ 64b + 64b $\rightarrow$ 64b \\
      F &  F & 2F & 2F &  16b $\times$ 16b + 32b $\rightarrow$ 32b \\
    \hline
  \end{tabular}
  \caption{General rules for supported types per instruction in base
    vector extension.  X represents the number of bits in an integer
    type and F represents the number of bits in a floating-point type.
    Individual instruction types will provide more detailed listings.
    Note that the type of a scalar floating-point operand can never be
    different from that of the vector in Src2, hence the Src1=2F case
    is missing from vector-scalar operations.}
  \label{tab:vtypemix}
\end{table}

A general rule in the base vector instruction set is that the
destination precision is never less than any source operand, except
for explicit type-conversion instructions.  Another general rule is
that the input operands can only be the same width or half the width
of the destination operand except for the scalar operand in integer
vector-scalar instructions, which is always XLEN wide.  Also, src2 is
never larger than src1 or src3.

Integer computations of mixed-precision values always aligns values by
their LSB, and sign or zero-extends any smaller value according to its
type.  The result is truncated to fit in the destination type.  Note a
scalar integer value is already XLEN bits wide, and as wide as any
possible integer vector value.

Floating-point computations on mixed-precision values acts as if the
calculations are performed exactly then rounded once to the
destination format.

\section{Rapid Configuration Instructions}

It can take several CSR instructions to set up the {\tt vdcfg} and
{\tt vnp} CSRs for a given configuration.  Specialized configuration
instructions are provided to quickly set up common configurations in
the {\tt vdcfg} and {\tt vnp} CSRs.

The {\tt vsetdcfg} instruction takes a scalar register value encoded as
shown in Figure~\ref{fig:vdcfg}, and returns the corresponding MVL in
the destination register.  The {\tt vsetdcfg} and {\tt vsetdcfgi}
instructions also clear the {\tt vnp} register, so no predicate
registers are allocated.

\begin{discussion}
  For now, only a 32-bit value supporting up to three different vector
  data types is supported by the {\tt vsetdcfg} instruction.  RV64 and
  RV128 could support larger number of types, though it's not clear if
  the hardware cost (area, latency) to support a larger number of
  different types is justified.
\end{discussion}

\begin{figure}[b]
  \centering
  \begin{tabular}{p{1cm}p{1cm}ccc|c|c|c|c|c|c|c|l}
    \multicolumn{1}{c}{} &
    \multicolumn{1}{c}{} &
    \multicolumn{1}{c}{} &
    \multicolumn{1}{c}{} & 
    \multicolumn{1}{c}{} &
    \multicolumn{1}{c}{} &
    \multicolumn{1}{c}{} &
    \multicolumn{1}{c}{} &
    \multicolumn{1}{c}{} &
    \multicolumn{1}{c}{mode} &
    \multicolumn{1}{c}{} &
    \multicolumn{1}{c}{} &  \\
    \cline{6-12}
    & & & & &
    \tt type2 & \tt ntype2 &
    \tt type1 & \tt ntype1 &
    0 &
    \tt type0 & \tt ntype0 &  \\
    \cline{6-12}
    \multicolumn{1}{c}{} &
    \multicolumn{1}{c}{} &
    \multicolumn{1}{c}{} &
    \multicolumn{1}{c}{} & 
    \multicolumn{1}{c}{} &
    \multicolumn{1}{c}{5} &
    \multicolumn{1}{c}{5} &
    \multicolumn{1}{c}{5} &
    \multicolumn{1}{c}{5} &
    \multicolumn{1}{c}{2} &
    \multicolumn{1}{c}{5} &
    \multicolumn{1}{c}{5} &  \\
    %% \cline{2-12}
    %% & \multicolumn{1}{|c|}{0} & F128 &
    %% \multicolumn{1}{c|}{type3} & \multicolumn{1}{c|}{\#type3} &
    %% type2 & \#type2 & type1 & \#type1 & 0 & type0 & \#type0 & RV64 \\
    %% \cline{2-12}
    %% & & &
    %% \multicolumn{1}{c}{} &
    %% \multicolumn{1}{c}{24} &
    %% \multicolumn{1}{c}{5} &
    %% \multicolumn{1}{c}{5} &
    %% \multicolumn{1}{c}{5} &
    %% \multicolumn{1}{c}{5} &
    %% \multicolumn{1}{c}{2} &
    %% \multicolumn{1}{c}{5} &
    %% \multicolumn{1}{c}{5} &  \\
    %% \cline{1-12}
    %% \multicolumn{1}{|c|}{0} & \multicolumn{1}{c|}{X128} &
    %% \multicolumn{1}{c|}{F128} & X64 & F64 & F32 & F16 & X32 & X16 & X8 & RV128 \\
    %% \cline{1-12}
    %% \multicolumn{1}{c}{83} &
    %% \multicolumn{1}{c}{5} &
    %% \multicolumn{1}{c}{5} &
    %% \multicolumn{1}{c}{5} &
    %% \multicolumn{1}{c}{5} &
    %% \multicolumn{1}{c}{2} &
    %% \multicolumn{1}{c}{5} &
    %% \multicolumn{1}{c}{5} &  \\
  \end{tabular}
  \caption{Format of the {\tt vsetdcfg} value.  The value contains
    three pairs of a 5-bit type and a 5-bit number of registers
    to create of that type. A value of 0 for the number of a type
    indicates that 32 registers should be allocated.  A value of 0 for
    the type indicates this pair should be skipped.  The types must be
    of monotonically increasing size from type0 to type2. }
  \label{fig:vdcfg}
\end{figure}

The {\tt vsetdcfg} value specifies how many vector registers of each
datatype are allocated, and is divided into a 2-bit mode field and
pairs of 5-bit fields for each data type in the configuration.

The 2-bit mode field indicates the configuration mode of the vector
unit and is zero for the base vector extension.

\begin{commentary}
  The standard vector extension operating mode configures the vector
  unit into some number of vector registers, each with some number of
  elements of types supported by the scalar unit.

  At least one alternative mode is planned, where the vector unit is
  configured as some number of registers each holding a single large
  element, e.g., 256 bits.  This would be the base for cryptographic
  operations, or other coprocessors that operated on large structures.

  Other modes can be used to reconfigure the vector unit register file
  and functional units for other domain-specific purposes.
\end{commentary}

Each datatype pair contains a 5-bit {\tt type}$x$ value encoded as a
{\tt vetype}$n$ value, and a 5-bit {\tt ntype}$x$ for the number of
registers to allocate for that type. If the {\tt type0} field is
non-zero, the {\tt vsetdcfg} instruction will configure the first {\tt
  ntype0} vector data registers to have {\tt vetype}$n$ values of {\tt
  type0} with {\tt vemaxw}$n$ values set accordingly as shown in
Table~\ref{tab:vetype}.  If the {\tt type0} value is 0, the datatype
pair is skipped.  If the {\tt type1} field is non-zero, then the next
{\tt ntype1} vector registers are configured to be of the type given
in {\tt type1}.  Similarly for the {\tt type2} pair.

A value of zero in a {\tt type}$x$ field indicates this datatype pair
should be ignored.  A value of zero in a {\tt ntype}$x$ field
indicates 32 registers should be allocated for the corresponding type.

\begin{commentary}
Zero values are skipped to simplify setting a configuration with two
different data types, where a single LUI instruction can set the upper
20 bits leaving the low bits zero.

A single 12-bit immediate value is sufficient to create a
configuration with some number of vector registers with a single given
datatype.

A compressed C.LI with a zero-extended 5-bit immediate can create a
configuration with 32 vector registers of a given datatype.
\end{commentary}

A corresponding {\tt vsetdcfgi} instruction takes a 12-bit immediate
value to set the configuration instead of a scalar value, but
otherwise is identical to the {\tt vsetcfgd} instruction.

\begin{discussion}
It is not clear how many immediate bits will be made available for the
{\tt vsetdcfgi} instruction.  If encoding space is available for both
12 immediate bits and a source register specifier, then {\tt
  vsetdcgfi} can be defined to read the source register, OR in the
bits in the immediate, then create a configuration.  In this case,
there is no need for a separate {\tt vsetdcfg} instruction.
\end{discussion}

The configuration value given must result in a legal configuration or
else an illegal instruction exception will be raised.

If a zero argument is given to {\tt vsetdcfg} the vector unit will be
disabled and the value 0 will be returned for MVL.  This instruction
({\tt vsetdcfg x0, x0}) is given the assembly pseudo-code {\tt
  vdisable}.

Separate {\tt vsetpcfg} and {\tt vsetpcfgi} instructions are provided
that write the source value to the {\tt vnp} register and return the
new MVL.  These writes also clear the vector data registers, set all
bits in the allocated predicate registers, and set {\tt vl}=MVL. A
{\tt vsetpcfg} or {\tt vsetpcfgi} instruction can be used after a {\tt
  vsetdcfg} to complete a reconfiguration of the vector unit.

\begin{discussion}
  If {\tt vnp} is made accessible as a separate CSR, the {\tt setpcfg}
  and {\tt setpcfgi} instructions are less useful.  The only advantage
  over a CSR instruction is that they return MVL, which is rarely
  needed, and which can be obtained via that {\tt setvl} instruction.
\end{discussion}

\section{Vector-Type-Change Instructions}

To quickly change the individual types of a vector register, {\tt
  vetyperw} and {\tt vetyperwi} instructions are provided to change
the type of the specified vector data register to the given scalar
register value or 5-bit immediate value respectively, while returning
the previous type in the destination scalar register.

A vector convert instruction, described below, can simultaneously
convert a source vector register into a new type, and set that type in
the destination vector register.

\section{Vector Length}

The active vector length is held in the XLEN-bit WARL vector length
CSR {\tt vl}, which can only hold values between 0 and MVL inclusive.
Any writes to the configuration registers ({\tt vdcfg}$x$ or {\tt
  vnp}) cause {\tt vl} to be initialized with MVL. Changes to {\tt
  vetype}$n$ via vector-type-change instructions do not affect {\tt
  vl}.

The active vector length is usually set via the {\tt setvl}
instruction.  The source argument to the {\tt setvl} is the requested
application vector length (AVL) as an unsigned XLEN-bit integer. The
{\tt setvl} instruction calculates the value to assign to {\tt vl}
according to Table~\ref{tab:vlcalc}.  The result of this calculation
is also returned as the result of the {\tt setvl} instruction.

\begin{commentary}
Earlier drafts encoded {\tt setvl} using a modified CSRRW instruction
whereas it is now encoded as a separate new instruction.
\end{commentary}

\begin{table}
  \centering
  \begin{tabular}{|c|c|}
    \hline
    AVL Value & {\tt vl} setting \\
    \hline
    AVL $\geq$ 2\,MVL & MVL \\
    2\,MVL $>$ AVL $>$ MVL & $\lceil$AVL$/2\rceil$ \\
    MVL $\geq$ AVL & AVL \\
    \hline
  \end{tabular}
  \caption{Operation of {\tt setvl} instruction to set vector
    length register {\tt vl} based on requested application vector
    length (AVL) and current maximum vector length (MVL).}
  \label{tab:vlcalc}
\end{table}

\begin{commentary}
  The rules for setting the {\tt vl} register help keep vector
  pipelines full over the last two iterations of a stripmined loop.
  This version of the rules guarantees monotonically decreasing vector
  lengths. 
  Similar rules were previously used in Cray-designed machines~\cite{crayx1asm}.
\end{commentary}

\begin{discussion}
  There are multiple possible rules for setting VL, and we could give
  implementations freedom to use different VL setting rules.
\end{discussion}

\begin{commentary}
  The idea of having implementation-defined vector length dates back
  to at least the IBM 3090 Vector Facility~\cite{ibm370varch}, which
  used a special ``Load Vector Count and Update'' (VLVCU) instruction
  to control stripmine loops.  The {\tt setvl} instruction included
  here is based on the simpler {\tt setvlr} instruction introduced by
  Asanovi\'{c}~\cite{krstephd}.
\end{commentary}

The {\tt setvl} instruction is typically used at the start of every
iteration of a stripmined loop to set the number of vector elements to
operate on in the following loop iteration.  The current MVL can be
obtained from a vector configuration instruction, or by performing a
{\tt setvl} with a source argument that has all bits set (largest
unsigned integer).

When {\tt vl} is less than MVL, vector instructions will set all
elements in the range [{\tt vl}:MAXVL-1] in the destination vector
data register or destination vector predicate register to zero.

\begin{commentary}
  Requring zeroing of elements past the current active vector length
  simplifies the design of units with renamed vector data registers.
  If the specification left destination elements unchanged, renaming
  implementations would have to copy the tail of the old destination
  register to the newly allocated destination register.
  Alternatively, specifying the tail to be undefined will expose
  implementation differences and possibly cause a security hole.

  Implementations that do not support renaming, will have to zero the
  tail of a vector, but this can reuse the mechanism that is already
  required to initialize all vector data registers to zero on
  reconfiguration, for example, by having a zero bit on each element
  or element group.
\end{commentary}

No element operations are performed for any vector instruction when
{\tt vl}=0.

\begin{commentary}
  Two possible choices are to 1) require destination registers to be
  completely zeroed when {\tt vl}=0, or 2) no changes to the
  destination registers.  Option 2 is currently chosen as this will
  prevents unnecessary work in some implementations, and option 1 does
  not provide a clear advantage beyond seeming more consistent with
  {\tt vl}>0 case.
\end{commentary}

\begin{figure}[bt]
  \centering
\begin{verbatim}
                 # Vector-vector 32-bit add loop.
                 # a0 holds N
                 # a1 holds pointer to result vector
                 # a2 holds pointer to first source vector
                 # a3 holds pointer to second source vector
                 li t0, (2<<VNTYPE0|VREGF32)
                 vsetdcfg t0     # Configure with two 32-bit float vectors

          loop:  setvl t0, a0    # Set length, get how many elements in strip
                 vld v0, a2      # Load first vector
                 sll t1, t0, 2   # Multiply length by 4 to get bytes
                 add a2, t1      # Bump pointer
                 vld v1, a3      # Load second vector
                 add a3, t1      # Bump pointer
                 vadd v0, v1     # Add elements
                 sub a0, t0      # Decrement elements completed
                 vst  v0, a1     # Store result vector
                 add a1, t1      # Bump pointer
                 bnez a0, loop   # Any more?

                 vdisable        # Turn off vector unit
\end{verbatim}
\caption{Example vector-vector add loop.}
\label{fig:vvadd}
\end{figure}

\section{Predicated Execution}

All vector instructions in the base vector instruction set have a
single bit to select either {\tt vp0} or {\tt vp1} as the active
predicate register.

\begin{commentary}
  The 32-bit base encoding does not leave room for a fully orthogonal
  predicate register specifier.  A single bit is dedicated to the
  predicate register specification, and is used to select between two
  active predicate registers, {\tt vp0} or {\tt vp1}. An alternative
  scheme would have used the bit to select between {\tt vp0} and
  unpredicated (all elements active).  However, given the ease of
  setting all predicate bits in a vector predicate register with a
  single predicate instruction, the current scheme provides more
  flexibility.

  When there are no vector predicate registers enabled, {\tt vp0}
  returns all set bits when read.  So, the assembler convention is to
  assume {\tt vp0} as the predicate register when no predicate
  register is explicitly given.  The assembler can support a strict
  operands option to require the vector predicate register is
  explicitly specified.
\end{commentary}

At element positions where the selected predicate register bit is
zero, the corresponding vector element operation has no effect (does
not change architectural state or generate exceptions), except to
write a zero to the element position in the destination vector
register.

\begin{discussion}
  The previous proposal (undisturb) left the destination vector
  unchanged at element positions where the predicate bit is false,
  whereas the current plan-of-record (zero) writes zero to the
  destination where the predicate bit is false.

  The advantage of the undisturb option is that it can require fewer
  instructions and fewer architectural registers for many common code
  sequences.  For in-order machines without register renaming, the
  undisturb operation simply disables writes to the destination
  elements, except for vector registers that have not been written
  since configuration time. Typically an extra zero bit per vector
  register or element group will be added to represent a zeroed
  register instead of actually zeroing state at configuration time.
  For predicated undisturb writes to these uninitialized registers,
  the predicated false elements must be explicity written with zeros
  on each element group and the zero bit is then cleared down.
  However, in a machine with vector register renaming, undisturb does
  imply an additional read of the original destination register to
  write the value into the new physical destination register when the
  predicate is false.  This additional read port will often be cheaper
  than in a scalar machine as vector machines often time-multiplex
  read ports, and the additional read can be skipped when the
  predicate registers are disabled ({\tt vnp}=0) or when the source is
  known to be zero after configuration, but still adds complexity to a
  design.

  The advantage of the zero option is that a machine with vector
  register renaming does not need to read the original destination
  vector register and so a read port is saved.  The disadvantage of
  the zero option is that more instructions and architectural
  registers are required for common code sequences, and simpler
  microarchitectures without register renaming are penalized by
  requiring longer code sequences and greater register pressure.  In
  particular, vector merge instructions are required to collect
  results from two divergent control paths, and each vector merge has
  to read two vector values and write a vector result.  Whether the
  zero option saves total register file traffic in an register-renamed
  microarchitecture depends on the ratio of a) internal temporary
  writes, to b) writes creating values that are live out of each basic
  block, and also to the frequency of control flow merges.

  Overall, the zero option removes significant complexity from the
  renamed machines while reducing efficiency somewhat for the
  non-renamed machines, and is the current plan-of-record.
\end{discussion}

{\bf The following sections are preliminary notes.}

\section{Predicate Operations}

All the standard logical operations are defined on predicate
registers.

\begin{commentary}
  The predicate operations have effectively three inputs, the two
  register specifiers rs1 and rs2, plus the predicate specifier {\tt
    vp0} or {\tt vp1}.  There are 256 possible logic operations on
  three bits of input, so can specify with an 8-bit immediate
  providing the lookup table.
\end{commentary}

A predicate swap operation is defined to exchange the values of two
predicate registers.  This is used to work around the lack of
predicate register specifiers in the base vector ISA.

\begin{verbatim}
     vpswap vp0, vp5  # Exchange values in vp0 and vp5.
\end{verbatim}

\begin{commentary}
  The predicate swap can be performed with just rename table updates
  in a renamed architecture.  Non-renamed machines will have to
  explicitly copy the values.
\end{commentary}

\begin{discussion}
  Not clear if the swap is really needed, or if explicit moves into
  {\tt vp0} and {\tt vp1} will suffice.
\end{discussion}

The predicate operations include operations to support software
vector-length speculation for vectorization of while loops.

\begin{commentary}
  The general scheme is described in Chapter 6 of \cite{krstephd}.
\end{commentary}

\section{Vector Load/Store Instructions}

Three vector load/store addressing modes are supported, unit-stride,
constant stride, and indexed (scatter/gather).  Each addressing mode
has a 7-bit unsigned immediate offset that is scaled by the element
type.

The unit-stride address mode takes a scalar base byte address, adds
the scaled immediate, then generates a contiguous set of element
addresses for loads or stores.

\begin{commentary}
  The primary use of immediates in unit-stride loads is to generate
  overlapping unit-stride loads for convolution operations.
\end{commentary}

The constant-stride address mode takes a scalar base byte address, a
stride value encoded in bytes, and adds a scaled immediate value.

\begin{commentary}
  The stride value is in bytes to allow a single stride register to be
  used to support operations on arrays-of-structures, where not all
  elements in each structure have the same size.  The immediate value
  is still scaled by element size to increase reach, given that
  element types will be naturally aligned.
\end{commentary}

The indexed address mode takes a scalar base byte address and a vector
of byte offsets.  The scalar base address and the immediate value are
added to element of the offset vector to give a vector of addresses
used in a scatter/gather.

Indexed stores are provided in three types.  Unordered, ordered, and
reverse-ordered.  The unordered indexed stores might update the same
memory location from two different elements in an unspecified order.
The ordered stores always update memory locations in increasing vector
element order.  The reverse-ordered stores always update memory
locations in decreasing memory order.

\begin{commentary}
  The reverse-ordered stores support vectorization of software memory
  disambiguation techniques.  A reverse-ordered store of element id
  into a hash table indexed by a hash on a store access address,
  followed by a read of the hash table using a load access address and
  a comparison against the original element id, will indicate if
  there's a potential RAW hazard with an earlier loop iteration.
\end{commentary}

\begin{discussion}
  Not clear if there is sufficient realizable improvement for
  supporting unordered stores over ordered stores.
\end{discussion}

Vector loads/stores have a simple memory model, where each vector
load/store is observed to complete sequentially in program order ony
the local hart, i.e., a vector load on a hart will observe all earlier
vector stores on the same hart, and no later vector stores.

Vector loads are available in a length-speculative form that writes
predicate register {\tt vp1} in addition to the destination vector
data register.  These instructions raise an illegal instruction
exception if {\tt vp1} is not configured.  For elements that do not
generate a permissions fault, the length-speculative vector loads
operate as normally except to also clear the bit in {\tt vp1}.  If an
element encounters a permission fault, a zero is written to the
destination vector register element and the {\tt vp1} bit is set to a
1.  Implementations may treat elements past the first faulting element
as also causing a fault even if they might not cause a permissions
fault when accessed alone.

Once software determines the active vector length, it should check if
any loads within the active vector length caused a fault, and in this
case, generate a non-length-speculative load to trigger reporting of
the error.

\begin{commentary}
  Length-speculative vector loads are required to vectorize while
  loops, with data-dependent exits (e.g. strlen).

  The only faults ignored by the length-speculative vector loads are
  ones that would have resulted in a permissions violation.  Page
  faults and other virtualization-related faults should be handled
  invisibly to the user thread by the execution environment.

  A malicious program can use length-speculative vector loads to probe
  accessible address space without fear of a fatal fault.
\end{commentary}

\section{Vector Select}

A vector select produces a new result data vector by gathering
elements from one source data vector at the element locations
specified by a second source index vector.  Data source and
destination vector types must agree.  The index vector can have any
integer type.  Legal element indices can range from 0 to current
MAXVL.  Indices out of this range raise an illegal instruction
exception.

\begin{verbatim}
  # vindices holds values from 0..MAXVL
  vselect  vdest, vsrc, vindices
\end{verbatim}

\section{Reductions}

Reductions are supported via a vector extract instruction that takes
elements starting from the middle of one vector and places these at
the beginning of a second vector register.  This supports a
recursive-halving reduction approach for any binary associative
operator.

\begin{commentary}
  A similar vector register extract instruction was added to the Cray
  C90 after memory latency grew too large for the memory-memory
  reductions used in earlier Crays.

  The vector unit microarchitecture can be optimized for the
  power-of-2 sized element offsets used for reductions.
\end{commentary}


\section{Fixed-Point Support}

Clip instruction supports scaling, rounding, and clipping to
destination type.  Rounding set by CSR fixed-point rounding mode
(truncate, jam, round-up, round-nearest-even).  Clipping set by CSR
clip mode (wrap, saturate).

Add with average, rounding set by rounding mode.

Multiply with same size source and destination types, with some result
scaling values (+1, 0, -1, -8?) and rounding and clipping according to
CSR mode.

Accumulate with carry into predicate register to support larger
precise dot-products.

\section{Optional Transcendental Support}

