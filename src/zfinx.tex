\chapter{``Zfinx'' Standard Extension for Floating-Point in Integer Registers, Version 0.1}
\label{sec:zfinx}

This chapter defines the ``Zfinx'' extension (pronounced ``z-f-in-x''), which
redefines the instructions in the standard floating-point extensions
(including F, D, Q, and Zfh) to operate on the {\tt x} registers, rather than
the {\tt f} registers.
The Zfinx extension depends on the F extension.

\begin{commentary}
The baseline F extension uses separate {\tt f} registers for floating-point
computation.
This design reduces register pressure and simplifies the provision of
register-file ports for wide superscalars.
However, the additional \wunits{128}{B} of architectural state increases the
minimal implementation cost.
By eliminating the {\tt f} registers, the Zfinx extension substantially
reduces the cost of simple RISC-V implementations.

Unlike most RISC-V extensions, the addition of Zfinx is not backwards
compatible: software that uses floating-point instructions but assumes the
absence of the Zfinx extension will not, in general, execute correctly on
implementations with the Zfinx extension.
\end{commentary}

The Zfinx extension {\em removes} the floating-point load, store, and
integer-transfer instructions (FL[HWDQ], FS[HWDQ], FMV.[HWDQ].X and
FMV.X.[HWDQ]) and corresponding C-extension instructions (C.FL[WD],
C.FL[WD]SP, C.FS[WD], C.FS[WD]SP).
The opcodes corresponding to the removed instructions are {\em reserved}.

\begin{commentary}
Zfinx software uses integer loads and stores to transfer floating-point values
from and to memory.
Transfers between registers use either integer arithmetic or floating-point
sign-injection instructions.
\end{commentary}

The Zfinx extension {\em redefines} all instructions that access {\tt f}
registers to instead access the {\tt x} register with the same number.

In the standard privileged architecture defined in Volume II, the
{\tt mstatus} field FS is hardwired to 0 if the Zfinx extension is
implemented, and FS no longer affects the trapping behavior of
floating-point instructions or {\tt fcsr} accesses.

\section{NaN Boxing of Narrower Values}

Floating-point operands of width \mbox{{\em w} $<$ XLEN bits} occupy bits
\mbox{{\em w}-1:0} of an {\tt x} register.
Floating-point operations on {\em w}-bit operands ignore operand bits
\mbox{XLEN-1:{\em w}}.

Floating-point operations that produce \mbox{{\em w} $<$ XLEN-bit} results
fill bits \mbox{XLEN-1:{\em w}} of the result with ones.

\begin{commentary}
To avoid the need for dedicated floating-point load instructions that fill the
MSBs of an {\tt x} register with ones, we abandon the usual NaN-boxing
requirement for floating-point operands.
We retain the NaN-boxing of results to keep Zfinx as similar as possible and
to ease debugging.
\end{commentary}

\section{Processing of Wider Values}

Double-precision operands in RV32DZfinx and quad-precision operands
in RV64QZfinx are held in aligned {\tt x}-register pairs, i.e.,
register numbers must be even.
Use of misaligned (odd-numbered) registers for double-width floating-point
operands is {\em reserved}.

Regardless of endianness, the lower-numbered register holds the low-order
bits, and the higher-numbered register holds the high-order bits: e.g., bits
31:0 of a double-precision operand in RV32DZfinx might be held in register
{\tt x14}, with bits 63:32 of that operand held in {\tt x15}.

When a double-width floating-point result is written to {\tt x0}, the entire
write takes no effect: e.g., for RV32DZfinx, writing a double-precision result
to {\tt x0} preserves the contents of {\tt x1}.

\begin{commentary}
Load-pair and store-pair instructions are not provided, so transferring
double-precision operands in RV32DZfinx from or to memory requires
two loads or stores.
Register moves need only a single FSGNJ.D instruction, however.
\end{commentary}

Quad-precision operands in RV32QZfinx are handled analogously, with
a four-register alignment requirement.
